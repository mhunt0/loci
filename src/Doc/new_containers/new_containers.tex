\documentclass{article}

\usepackage{amsmath}
\usepackage{fullpage}
\usepackage[dvipdf]{graphicx}
\usepackage[svgnames]{xcolor}
\DeclareGraphicsExtensions{.pdf,.jpeg,.png}
\usepackage{threeparttable}

\newcommand{\lstfontfamily}{\ttfamily}
\usepackage{listings}
\lstset{%listings package
  language=c++,
  basicstyle=\lstfontfamily,
  emphstyle=\color{red}\bfseries, 
  keywordstyle=\color{DarkViolet}\bfseries,
  commentstyle=\color{DarkGreen},
  stringstyle=\color{DarkBlue},
  numberstyle=\color{DarkGrey}\lstfontfamily,
  emphstyle=\color{red},
  % get also javadoc style comments
  morecomment=[s][\color{Maroon}]{/**}{*/},
  % morekeywords=[1]{include},
  columns=fullflexible,
  % escapeinside=`',
  % escapechar=@,
  showstringspaces=false,
  numbers=left}

\lstdefinelanguage{plain}{
  keywords={}}

\usepackage{enumitem}

\begin{document}

\title{Design guideline and documentation for the new Loci fact database}

\author{The Loci Group}

\maketitle

The purpose of this document is to outline the redesign of the Loci fact
database and its associated infrastructures.  This document is meant to
be updated incrementally by all the people involved in this project and
serves as a medium to document ideas and progress along the way.  In the
end, when everything is finished and settled, this document will become
a manual for the code that have been developed for this part of the Loci
framework.

\section{Motivations and ideas}
One of the motivations for this redesign is to be able to support
extremely large meshes in the Loci framework.  Currently the mesh size
is limited by the use of the 32-bit integers in the Loci framework as
entity IDs.  We want to use a hybrid scheme where the global IDs will
use 64-bit integers (or whatever suitable type) and the local IDs will
continue to use 32-bit integers.  Such a scheme will support almost all
imaginable mesh sizes we will ever need in the future.

Another motivation is to use this as a good opportunity to reorganize
the container and fact database part of the Loci framework so that they
become: 1) better organized 2) more streamlined 3) more generic and
abstract (which allow us to build increasingly more complex and modular
codes).  We are particularly interested in having a new database
design that will also incorporate several new ideas we have been
considering (e.g, key-spaces, data distribution independent grid reader,
etc.).

Currently not all ideas and approaches are clear to us (part of the
reasons why we have such a document).  Below are some discussions of the
perceived problems and some of the current thoughts we have on them.

The current data-flow in a Loci program starts with the grid reader.
The grid reader first reads in the raw data in a mesh using the IDs
stored inside the mesh file (called the ``file number'').  Then the grid
reader redistributes the data read in using a new ID scheme (called the
``global scheme'').  Before the Loci scheduling phase, another ID scheme
(called the ``local number'') is created to facilitate the compaction
of data storage per distributed process and optimization of a
distributed memory schedule.

The current problem with this design and the code is that the entire
process is tightly integrated and hard to reuse.  In principle, Loci
does not mandate any particular parallel data distribution scheme and
can certainly use any data distribution scheme correctly provided.  In
practise though, applications almost always rely on the default grid
reader to generate a proper data distribution since it is quite hard and
involved to define any new plugins for data distribution (if not
impossible).

Another problem concerns the current design of the fact database.  The
current fact database has evolved to be a quite complex piece of code
with multiple layers of functionality involved.  Ideally the fact
database is just a central place where data containers are registered so
that they can be properly called for when a computation needs them.
However currently the fact database 1) manages the state of the data
distribution (which includes the partition information and the mapping
information between different numbering schemes). 2) The fact database
also manages the state for \textit{all} the containers at once (meaning
either all containers will be in global number scheme or the local
number scheme).  3) the containers themselves utilize a quite
sophisticated shell/handle and notification design approach that would
in some degree automatically manage the content pointers underneath.
While this is convenient at times, it will get in the way in certain
parts of the code (e.g., the multi-threaded infrastructure).

Currently we would like to use this chance of increasing mesh size to
64-bit to address all the above issues so that the data containers and
the fact database are redesigned to support future tasks.  Currently we
are considering the following ideas/approaches:

\begin{itemize}
  \item We would like to reorganize the grid reader so that every step
    in the current reader is separated and preferably better abstracted.
  \item We would also like to redesign the file number and global number
    stages so that they become more modular and flexible.  Our current
    ideas are to have the grid reader generate raw data in the form of
    file number directly obtained from the mesh files.  Then a separate
    process will supply the data distribution criteria and then perform
    the data distribution in a global numbering scheme.  Finally when
    everything (mainly data) is properly distributed/settled on each
    distributed process, we will convert everything into a local number
    scheme.  The scheduling/computation process then begins.  The local
    number part will largely stay unchanged.  We just want to change the
    part before the local number stage.
  \item We also want to incorporate the idea of ``key-space'' into
    the design.  Roughly speaking, our current understanding of the
    key-space concept is that it provides better structuring support.  A
    key-space is mostly similar to a name-space in the C++ language.  It
    allows one to define multiple things within such a space without the
    worry that they will be in conflict with others defined in different
    spaces.  However in our case, the Loci system also automatically
    provides some guarantees and utilities that will facilitate the
    interactions between different key-spaces (thus liberating the
    programmers from manually coding and checking several tedious and
    error prone tasks). The entire concept of key-space and its utility
    are still unclear to us at this moment.  We still need to discuss it
    more in the future.  Hopefully this document will provide a place
    where thoughts and ideas can be formally documented and clarified.
\end{itemize}

\section{The new container type}
For the tasks to begin, we will need to pick something to start with.
We have decided that the designing of several new data container types
will probably provide a gentle start.  This also allows us to evaluate
the situation as things progress.

As said, we want to separate the current grid reader into multiple
abstract steps. We think the starting place for this process is to
change the current grid reader so that it writes out the mesh structure
into these new data container types.

These new container types are mainly for the dynamic phase of the Loci
framework (i.e., the global numbering stage).  They will feature a new
concept.  We just wanted to express data conceptually as a list of
pairs.  For example, in the current Loci framework, a
\verb|store<string>| is an array of strings that starts with an
arbitrary index number and it supports fast random access.  In our new
container type (which we will just call \texttt{gstore} for now), it can
be \emph{viewed} as a long list of ``(ID,string)'' pairs (e.g.,
\texttt{(10,"okay")}, \texttt{(4,"yes")}, \texttt{(394,"name")},
\ldots).  These pairs are essentially the same index and value contents
in the current Loci \texttt{store}.  In this new container type,
individual pairs can be inserted (and possibly removed) from the store
dynamically.  In addition, pairs are just appended to the list and are
unordered.  Thus random access is not supported in this new container
type in general.  Only if the list of pairs is sorted, then can we have
a faster binary-search based access method (in $\log n$ time
complexity).  The main reason to have such a concept and organization is
to be flexible.  When we are creating the initial data for the Loci
framework to process/schedule, we want to be as dynamic as we would like
to.  This conceptual view also affords increased abstraction for other
parts of the framework.

To implement this concept though, there are many other details that need
to be resolved.  We will discuss several of them below.  It is very
likely that there will be more that need to be resolved.  Thus we may
need to iterate on this design.

First of all, the new container types are broadly based on the list of
pair concept.  We still need to distinguish the concrete types (i.e., a
\texttt{store} or a \texttt{map}).  Further in the list of pairs
concept, we do not necessarily care about the individual types (e.g., a
\texttt{multistore} or \texttt{multimap}).  However when we wanted to
convert these lists into array based forms, we will want to know the
exact type.  Therefore we need to tag each such list of pairs the
intended types.  So far, we will want to support the general
\texttt{store}, a \texttt{storevec} with same vector length per key, and
a \texttt{multistore} with variable value length per key.  Also needed
are a general \texttt{map} and a \texttt{multimap}.  Notice, when
creating these containers, we will only insert pairs in the form
``(ID,value)'' and not ``(ID,vector$\langle$value$\rangle$)'' for the
\texttt{storevec} and \texttt{multistore} version.  For example, the
array form of a \texttt{storevec} might look like:
\texttt{([1]->[a|b|c]}, \texttt{[2]->[e|f|g])}, i.e., there are two keys
\texttt{1} and \texttt{2} each associates a value of size three vectors.
When creating such a new container, we will just have a list of pairs
as: \texttt{[(1,a),} \texttt{(1,b),} \texttt{(2,e),} \texttt{(2,f)}
\texttt{(1,c),} \texttt{(2,g)]}.  There are two things to be noted: 1)
it is generally not possible to deduce the type automatically.  For
example, we do not know the type just by looking at the list in the
previous example.  Although apparently each key has three values
associated, it could be a special case of \texttt{multistore}.  2) the
order among the values associated per key is generally important.  In
the previous example, \texttt{[a|b|c]} and \texttt{[e|f|g]} might encode
or represent some kind of embedded information (such as a topological
structure) and therefore their insertion order in the list needs to be
preserved.  This means we want to use a stable sorting algorithm when
we sort the list.

As indicated, these container types are generally not ordered.  However
the users can choose to order them manually and thus gain the ability to
have faster access using binary search based method.  Thus we need to
distinguish the state these containers will be in.  After each sorting,
we will be in an ordered state and as soon as an insertion takes place,
we will be back in the unordered state.  And if an access request is
made in the unordered state, an exception should be thrown to indicate
the usage error.  Another problem concerns the sorting is that whether
the sorting will be a local sort on each parallel process or a global
parallel sorting.  So far, we know that we do need the global parallel
sort for all the pairs across all processes.  We do not yet know if
local sort will be needed and in what condition.  Nevertheless since
local sort is trivial (just use the \texttt{std::sort} method) it can be
added easily.  However a global parallel sort is not trivial to do.  So
far the Loci framework provides a parallel sample sort routine (in the
\texttt{System/pnn.c} file).  One issue with parallel sample sort is
that it does not scale very well.  Thus we may want to use a different
sorting approach for the global sort.  For now, we could probably use
the sample sort first and address this issue later on. 

Another issue is about what to do and how to handle a container that
violates its declared type.  For example, if we find that there are
pairs like \texttt{(1,a)} and \texttt{(1,b)} in a container created to
be a straight \texttt{store}.  What do we do?  This is clearly a
violation of the type requirements (for a straight \texttt{store}, only
one value is permitted for each key).  Note this could happen from real
data creation, e.g., if data are created in a distributed fashion.  So
it will be nice to have a graceful way to handle this problem.
Currently we think that it is probably wise to automatically remove such
duplications and keep the container type requirements.  There are
basically two cases.  In the first case, if we have identical pairs such
as if we have two \texttt{(1,a)} pairs in the list, then we would want
to drop one of them (it is a case of duplication).  Second, if we have
different pairs like \texttt{(1,a)} and \texttt{(1,b)}, then we wanted
to keep the most important pair.  For the first case, we can rely on the
default ``\texttt{=}'' operator on the value type.  For the second case,
we can rely on the default ``\texttt{<}'' operator on the value type,
i.e., we will order the values for the same key and keep the one that is
at the end of the order (either the ``smallest'' or the ``largest''
one, we just need to be consistent).  In fact, we can merge these two
cases and only rely on the single ``\texttt{<}'' operator.  For example,
in the case of pairs \texttt{(1,a)} and \texttt{(1,b)}, since
\texttt{a<b} (assuming a dictionary order), we will drop \texttt{b} and
keep \texttt{a} (i.e., we keep the smallest value).  In the case of two
\texttt{(1,a)} pairs, since \texttt{a<a} is false, we will just drop the
first \texttt{a} since it is not ``smaller'' than the second \texttt{a}.

Now this decision brings two additional questions: 1) when do we perform
such a ``fix'' and 2) does this mean that we will force the user to
supply a ``\texttt{<}'' operator for all the container types that
require such a ``fix?''  The answer to these questions are somewhat
linked together.  For the second question, we think that we do not want
to force the users to define a ``\texttt{<}'' operator for all the value
types since some value types may not be able to be ordered in a
meaningful way.  For the first question, a natural and good place to
carry out such a ``fix'' seems to be during the sorting phase (since
once we sorted all the pairs, we can easily identify a duplication
problem).  However this may not be a good idea for 1) a user may not
request any sorting at all for the list of pairs 2) this implies an
automatic check point for such a problem, if we want ``fixes'' then we
will force the users to define the ``\texttt{<}'' operator.  Thus it
seems that it is best to provide an additional method that the users can
call to make sure the container data are consistent with respect to its
type requirements.  Then if they called such a method, then they are
aware that a ``\texttt{<}'' operator should be defined for the values.
Otherwise if they do not initiate such a method, then it maybe okay to
skip the ``\texttt{<}'' operator.  The final issue is that if the users
do not want to ``fix'' possible duplication problems, then at least the
container need to fail at some point to indicate such a problem.  If we
do not support pair removal, then the sorting phase will be a place to
carry out such a check.  If a duplicate is found, then an exception will
be thrown.  However sorting may not be called and insertion (and
possibly deletion if supported) may be carried out afterwards.
Currently the best and final place to give such warning seems to be when
the list of pairs is ``frozen'' as conventional array based container.

Below is an example of what an interface of a \texttt{store} might look
like:
\begin{lstlisting}
template<typename T>
class gstore {
public: // the interface
  void insert(Entity, T);
  void sort();
  void retrieve(Entity); // throws exception if not sorted
                           // may also be defined as operator[] overload
  void make_consistent();
  template<typename CMP>
  void make_consistent(CMP cmp); // a version that overloads the < operator
private: // suggestions how data might be represented
  std::vector<std::pair<Entity,T>> table;
  bool sorted; GSTORE_TYPE gt;
};
\end{lstlisting}

We will still utilize the shell/handle approach to design the new
container type.  In this paradigm, the real data is stored in a handle
part and then multiple shells can share the same handle (thus have
access to the same data).  This has the advantage of giving the same
data multiple names as well as avoiding unnecessary copies.  The C++
language's current standard offers several additional new semantics that
might help improve (or completely remove) such design patterns.  This
needs to be investigated further.  For the time being, we will use this
shell/handle paradigm.  In the current Loci container design, in order
to remove an indirect reference access, each shell is given direct
access to the handle's internal data pointer (thus provides faster
access time and potentially improve cache utilization).  However doing
so requires that the handle notify all the shells that are connected to
change their pointers whenever the handle changes its data layout.  In
this new design, we want to remove such a mechanism.  Instead, each
shell will just ask the handle to supply any data that it needs.  This
removes the notification needs but introduces a layer of indirection.
We think at this stage this is acceptable.

Another minor requirement is to provide a way to access the range of
values associated with a single key (for the multi-container types).
This could be done by implementing an iterator interface that returns
the begin and end range of the values associated with a particular key
queried.

There are several other details concerning the container type that we
are not very clear at the moment.  They will need to be addressed in the
future when we have more experience.  Some of these issues include 1)
how to introduce the key-space concept into the container types
(existing code inside the \texttt{include/keyspace.h} and
\texttt{System/keyspace.cc} files can perhaps be borrowed or referenced,
however these codes are experimental and they have a different purpose
and are also perhaps somewhat dated and more complex than is needed in
this case) 2) we will need to have each container aware of their own
data distribution for the ultimate flexibility.  Then how do we
incorporate such information into the container type?

Besides these issues, we also need several global functions that
operate on these container types.  For now, we will want to implement
the equi-join operations on these containers.  An equi-join
implementation already exists in the current Loci codebase (the
\texttt{equiJoinFF} function inside
\verb|include/distribute_container.h| file).  We just need to adapt
the implementation to the new container type.  Since our \texttt{map}
and \texttt{store} types are only pairs and not full-blown relations,
the join operations are limited and essentially the same as multiple
nested array access as we did in the past (e.g., \texttt{m1->m2->s}
denotes \texttt{s[m2[m1[i]]]}.  The main difference between this and the
now \texttt{eqjoin(eqjoin(m1,m2),s)} is that the equi-join will actually
produce a new list that merges and flattens all these data while in the
old way, no new data is created and we are just accessing the data
indirectly.  Such data flattening may be useful as it removes the cost
of indirect reference (i.e., improves cache performance).  It could also
be beneficial to certain other parts of the Loci system where indirect
data accesses may complicate or make impossible certain analyses.
However in the future, we might consider to let the framework decide if
such data flattening is beneficial.  Conceptually the users can creation
equi-joins and ``think'' they will get a new list.  Underneath, whether
or not an actual list is created or indirect accesses happened will be
determined by the runtime system based on some analyses.

Another related task is to clean up the current \texttt{intervalSet}
type in Loci.  The current entity and entity set types are defined as
aliases of integer and the \texttt{intervalSet} types.  Since we want to
make the ID keys larger than 32-bit integers, we will want to produce
new entity and entity set types that support this feature.  First the
\texttt{intervalSet} codes are very old.  Initially it may be careful
to use a parameterized integer type.  Over the years, there may be
direct use of the C++ \texttt{int} type introduced into the code.  We
want to carefully examine the code and remove these direct use of
\texttt{int}.  Then we will want to parameterize the
\texttt{intervalSet} so that it can use any arbitrary integer types as
its components.  On top of this, we can \texttt{typedef} a bigger entity
set type by making it use the \texttt{intervalSet} with bigger integers.

Based on these discussions, the initial tasks we need may be 1) design
the prototype of these data container types as discussed 2) to test them
and to interface them gradually with the grid reader, we can change (or
rewrite) the \texttt{readGridVOG} function (inside the
\texttt{System/FVMGridReader.cc} file) to produce data in these new
containers.  Currently the \texttt{readGridVOG} function reads a grid
(in \texttt{VOG} format) into \texttt{pos}, \texttt{face2node},
\texttt{cl}, and \texttt{cr}, which are \texttt{store},
\texttt{multimap}, \texttt{map}, and \texttt{map} types respectively. 

% \begin{lstlisting}
% int main(void)
% {
%   return 0;
% }
% \end{lstlisting}
% 
% \begin{verbatim}
%   class rule: public rule_type
%   {
%     virtual prelude(sequence seq) { /* definitions */ }
%     virtual compute(sequence seq) { /* definitions */ } 
%     virtual postlude(sequence seq) { /* definitions */ }
%     // other parts
%   }
% \end{verbatim}
\section{Documentation part}
This part is for document the work that has been done and will likely
stay stable.  For example, if we commit to certain design schemes and
produced some codes, then we can put comments and documents here in this
section.

\section{Implementation progress}
\subsection{genIntevalSet}
In files \texttt{intervalSet.h} \texttt{intervalSet\_def.h} and \texttt{intervalSet\_impl.h},
class \texttt{intervalSet} is parameterized and turn into template class
\texttt{genIntervalSet}. 

\texttt{int\_type} is used to represent 32-bit integer, and \texttt{GEntity} is used to
represent 64-bit integer. The corresponding MPI data type and hdf5
data type for \texttt{GEntity} are defined using \texttt{typedef}. macro \texttt{GFORALL}
is also added for \texttt{genIntervalSet<GEntity>}.

To avoid
polluting code in large amount of files, new files
\texttt{Tools/simple\_partition\_long.h}, \texttt{hdf5\_readwrite\_long.h} and
\texttt{distribute\_long.h} are added to deal with communication and io operation of
\texttt{GEntity}.

  
\subsection{GStore}
The new container class \texttt{GStore} is partially implemented in \texttt{gstore.h},
\texttt{gstore\_def.h} and \texttt{gstore\_impl.h}. 
std iterator for \texttt{vector} are used for \texttt{GStore} so that the std iterator functions
can be used. A local binary search function is added for the \texttt{retrieve}
function, which will be implemented later.

Two member function \texttt{copy2store} and \texttt{copy2map} are added to move the
contents of \texttt{GStore} into traditional Loci stores or maps. These two
functions will be modified later so that the mapping information
between different numbering scheme can also be generated. 

Up to now, the \texttt{sort} and \texttt{make\_consistent} functions are ignored because
they are not needed yet.  

 
\subsection{FVMGridReader}  
  In the current version of \texttt{FVMGridReader\_long.cc}, \texttt{readGridVOG} calls function \texttt{readGridVOG\_raw}
  first,  which reads \texttt{pos}, \texttt{cl}, \texttt{cr}, \texttt{face2node} information into \texttt{GStore}
  containers. \texttt{readGridVOG} function then  copys the information from new
  containers into traditional Loci stores and maps.

  While reading from VOG files, the local variables need to be modified
  to fit for larger integers, so are MPI communication functions and hdf5 io
  functions. 
   
\subsection{Next step}        
 so far, Loci is tested with \texttt{GEntity} is defined as c++ int. Next step,
 partition functions will be modified in \texttt{FVMGridReader\_long.cc}.


\end{document}
