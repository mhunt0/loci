\documentclass{article}

\topmargin  0.0in
\headheight  0.15in
\headsep  0.15in
\footskip  0.2in
\textheight 8.45in

\oddsidemargin 0.56in
\evensidemargin \oddsidemargin
\textwidth 5.8in


\pagestyle{myheadings}
\markright{\bf The FVMOverset Module}

\begin{document}

\title{The Finite Volume Overset Mesh Support Module User Guide}

\author{ Edward Luke \\
\it luke@cse.msstate.edu}

\maketitle

% example of figure including code
%\begin{figure}[htbp]
%  \centerline{ \epsfxsize=5.5in \epsfbox{figures/constructs.eps}}
% \caption{Four Basic Container Categories}
% \label{constructs}
%\end{figure}

\section{Introduction}

The {\tt fvmoverset} module is provided to make it easy to add overset
mesh capability to Loci finite volume solvers.  The module provides a
number of facilities to provide for mesh hole cutting through the use
of a blanking flag, facilities for providing rigid body motion to
components , and facilities for providing interpolation both to outer
interface boundaries as well as to blanked cells.  This document will
cover how to use the overset module and how to modify your code to
support overset mesh simulations.

\section{Preliminaries}

The first step to using the {\tt fvmoverset} module is to setup data
structures needed to perform the distance computations used by the
hole cutting algorithms.  This is accomplished by calling the {\tt
  setupOverset} call sometime after the {\tt setupBoundaryConditions}
call.  Note, this call will only perform operations when the {\tt
  componentMotion} variable has been defined, so it is safe to call
this routine even for cases where the overset features are not being
used.  This will look like the following:

\begin{verbatim}
    // Setup data structures for overset computations
    setupOverset(facts) ;
\end{verbatim}


Next, load in the module for that defines finite volume overset rules.

\begin{verbatim}
    // Setup data structures for overset computations
    Loci::load_module("fvmoverset",rdb) ;
\end{verbatim}

In addition to access the overset variables one will need to add the following include files to your directory
\begin{verbatim}
#include "FVMOverset/overset.h"
$include "FVMOverset/overset.lh"
\end{verbatim}

\section{Hole Cutting and Interpolation}


There are two capabilities that need to be implemented to provide the basic overset mesh functionality in a solver.  The first of these is the implementation of interface boundary conditions that provide interpolated values from other meshes.  This boundary condition implements a Dirichlet condition and the values at the interface are provided by the following interpolation parametric variables:
\begin{verbatim}
// Interpolate Scalar variable X
$type interpolateFace(X) store<real_t> ;
// Interpolate 3d vector variable X
$type interpolateFace_v3d(X) store<vector3d<real_t> > ;
// Interpolate generic vector variable X
$type interpolateFacev(X) storeVec<real_t> ;
\end{verbatim}

So for example to interpolate the scalar variable {\tt temperature}, the vector velocity variable {\tt v} and the mixture variable {\tt Y} use the following
\begin{verbatim}

$rule pointwise(flux<-interpolateFace(temperature),
                      interpolate_v3d(v),
                      interpolateFacev(Y)) { 
...
\end{verbatim}

Next the solver time integration scheme will need to be modified such that the integrated values for some cells can be replaced by interpolated values.  This is done by establishing an inplace update rule for each variable of interest at the beginning of the time integration loop.

For example we may have the following rules for temperature that is being advanced inside of a newton iteration in the solver.
\begin{verbatim}
// The iteration advances temperature_i, but processes temperature inside of
// the loop, a dummy inplace rule is used to provide a default so that this
// can be overrided when using overset functionality.
$rule pointwise(temperature{n,it}<-temperature_i{n,it}),
      inplace(temperature{n,it}|temperature_i{n,it}), prelude {}
// This rule doesn't actually do anything except cause a rule from the overset
// module to be instantiated that will replace temperature for blanked
// cells with the appropriate interpolated values
$rule pointwise(OUTPUT<-
 scalarInterpolateToBlankedCells(temperature,temperature_i)) {}
\end{verbatim}

Similar structures can be used for 3D vectors and generic values.  These are described as follows:
\begin{verbatim}
$type scalarInterpolateToBlankedCells(Xout,Xin) store<real_t> ;
$type v3dInterpolateToBlankedCells(Xout,Xin) store<vector3d<real_t> > ;
$type vecInterpolateToBlankedCells(Xout,Xin) storeVec<real_t> ;
\end{verbatim}

Finally the code needs to be modified to disable the inclusion of
iblanked cells in the time integration process.  Any cell where the
{\tt iblank} variable is either 0 or 1 should be simulated while any
value of 2 or 3 will be filled in by the interpolation code and should
not be simulated.  How this is accomplished is solver specific, but
some examples that would be involved in modifying the solution matrix
will be as follows:

\begin{verbatim}
  // Disable the source term for the matrix solve for iblanked cells
  $rule pointwise(fluid_B<-rhs,iblank), prelude {
    $fluid_B.setVecSize(5) ;
  } compute {
    if($iblank > 1)
      for(int i=0;i<5;++i)
        $fluid_B[i] = 0 ;
    else
      for(int i=0;i<5;++i)
        $fluid_B[i] = $rhs[i] ;
  }
\end{verbatim}
\begin{verbatim}
  // Disable the lower and upper parts of the faces that border the
  // iblanked regions
  $rule pointwise(fluid_L<-fjp,(cl,cr)->iblank),constraint(fjp),
    inplace(fluid_L|fjp) {
    if($cl->$iblank > 1 || $cr->$iblank > 1) {
      for(int i=0;i<5;++i)
        for(int j=0;j<5;++j)
          $fluid_L[i][j] = 0 ;
    } else {
      for(int i=0;i<5;++i)
        for(int j=0;j<5;++j)
          $fluid_L[i][j] = -$fjp[i][j] ;
    }
  }

  $rule pointwise(fluid_U<-fjm,(cl,cr)->iblank),constraint(fjm),
    inplace(fluid_U|fjm) {
    if($cl->$iblank > 1 || $cr->$iblank > 1 ) {
      for(int i=0;i<5;++i)
        for(int j=0;j<5;++j)
          $fluid_U[i][j] = 0 ;
    } 
  }
\end{verbatim}
\begin{verbatim}
  // make the diagonal term the identity matrix so that the iblanked cells
  // don't change with time
  $rule pointwise(fluid_D<-srcJ),   constraint(geom_cells), 
     inplace(fluid_D|srcJ) {
    if($iblank > 1) {
      for(int i=0;i<5;++i)
        for(int j=0;j<5;++j)
          $fluid_D[i][j] = 0.0 ;
      for(int i=0;i<5;++i)
          $fluid_D[i][i] = 1.0 ;
    }
  }
\end{verbatim}

There may be other subtle issues that also need to be changed.  For
example, residual and other solver monitoring code should be modified
to not include iblanked cells. 

\section{ComponentMotion}

The above should work for stationary overset mesh files.  In order to
support rigid body mesh motion for the components of the overset mesh
the solver needs to provide some rules to describe how the state of
the component motion evolves in time and how when the grid positions
are updated.  The component motion facility is for prescribed mesh
movement and so can be computed once per timestep. The motion state of
each component is stored in a parameter of vector of doubles.  A
version of this state is created for each component by using rules
that are parametric on {\tt volumeTag(X)}.  First to initialize the state we can use the following rules:

\begin{verbatim}
  $type componentMotionState_X_ic param<vector<real> > ;
  $type motionBehavior_X blackbox<Loci::CPTR<Loci::motionType> > ;

  // We use a unit/apply instead of singlton for initialization so that 
  // we can override for restarting
  $rule unit(componentMotionState_X_ic<-motionBehavior_X), parametric(volumeTag(X)) {
    $componentMotionState_X_ic = vector<real>() ;
  }
  // A default apply rule to silence warnings
  $rule apply(componentMotionState_X_ic<-volumeTag(X))[Loci::NullOp], parametric(volumeTag(X)), prelude {} ;
\end{verbatim}

\begin{verbatim}
  $rule singleton(componentMotionState_X{n=0} <- componentMotionState_X_ic),
  parametric(volumeTag(X)) {
    $componentMotionState_X{n=0} = $componentMotionState_X_ic ;
  }
\end{verbatim}

\begin{verbatim}
  $type motionInfo blackbox<map<string,real> > ;

  $rule apply(motionInfo<-dtmax)[Loci::NullOp],prelude {
    (*$motionInfo)["timeStep"] = real(*$dtmax) ;
  } ;
\end{verbatim}

\begin{verbatim}
  $rule singleton(componentMotionState_X_next{n}<-componentMotionState_X{n},motionBehavior_X{n},motionInfo{n}),parametric(volumeTag(X)) {
    $componentMotionState_X_next{n} = $motionBehavior_X{n}->advanceStateInTime($componentMotionState_X{n},$motionInfo{n}) ;
  }
\end{verbatim}

\begin{verbatim}
  $rule singleton(componentMotionState_X{n+1}<-componentMotionState_X_next{n}),
  parametric(volumeTag(X)) {
    $componentMotionState_X{n+1} = $componentMotionState_X_next{n} ;
  }
\end{verbatim}

\begin{verbatim}
  $rule unit(componentMotionData<-componentMotion), prelude {
    *$componentMotionData = map<string,Loci::componentXform>() ;
  } ;

  $rule apply(componentMotionData<-componentMotionState_X_next,motionBehavior_X,componentName_X)[Loci::NullOp],
    parametric(volumeTag(X)), prelude {

    Loci::componentXform xform = (*$motionBehavior_X)->getXform(*$componentMotionState_X_next) ;
    string name = *$componentName_X ;
    (*$componentMotionData)[name] = xform ;
  } ;
\end{verbatim}

\begin{verbatim}
  $rule singleton(componentTransformsn_X{n}<-componentTransforms_X{n}),
    parametric(volumeTag(X)) {
    $componentTransformsn_X{n} = $componentTransforms_X{n} ;
  }
  

  $rule pointwise(pos{n,it}<-pos{n},pos,componentTransformsn_X{n,it}),
    constraint(componentNodes_X{n},componentMotion),
    parametric(volumeTag(X)) {
    vect3d xpos = $pos ;
    for(size_t i=0;i<$componentTransformsn_X{n,it}.size();++i)
      xpos = $componentTransformsn_X{n,it}[i].applyXform(xpos) ;
    $pos{n,it} = xpos ;
  }    
\end{verbatim}

\subsection{Restarting}

\end{document}
