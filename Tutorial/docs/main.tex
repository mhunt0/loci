%Time-stamp: <2002-09-06 12:21:41 peo>
\documentclass[10pt,epsf]{book}
\renewcommand\familydefault{ptm}
\DeclareMathAlphabet{\mathmyrm}{OT1}{ptm}{m}{n}
\DeclareMathAlphabet{\mathmybf}{OT1}{ptm}{bx}{n}
\SetMathAlphabet{\mathmyrm}{bold}{OT1}{ptm}{bx}{n}
\setlength{\textheight}{7.75in}
\setlength{\textwidth}{5.7in}
\setlength{\parskip}{3mm}
\setlength{\parindent}{0.0in}
\setlength{\topmargin}{5mm}
%\setlength{\bottommargin}{5mm}
\setlength{\headheight}{5mm}
\setlength{\headsep}{5mm}
\setlength{\oddsidemargin}{2cm}
\setlength{\evensidemargin}{2cm}
\unitlength=1in
\title { Loci : A Tutorial }

\usepackage{epsf}    % used for importing encapsulated postscript figures
\usepackage{amsmath} % used for extended formula formatting tools
\usepackage{amssymb}
\usepackage{theorem}
\usepackage{euscript}

%\includeonly{intro}

\begin{document}
%\small
\tableofcontents
%\listoffigures
\maketitle
\thispagestyle{empty}
\newpage
\setcounter{page}{1}

\chapter {Introduction}

{\it Tutorial Needs Introduction}

\include{lfvm}

\chapter{ Basic Concepts }
\section{Notation used in this document}
In this document we use the {\tt typewriter} font to distinguish
actual Loci programming keywords, classes, and data-structures.
\section{ Installing Loci}
After you untar Loci you will have a directory structure like so:

\begin{verbatim}
Loci/
Loci/src
Loci/Tutorial
\end{verbatim}

To install Loci, you need to use {\tt tmpcopy} to make a linked copy
of the source code (usually to the {\tt /usr/tmp} directory).  You
can use environment variables to change how the tmpcopy routine works.
For example, you can set the target directory (where you will be
compiling the Loci libraries), and the compiler.  For example, to set
the target directory to {\tt /home/test/OBJ} and the compiler to {\tt
  gcc3.0}, enter (NOTE: Run tmpcopy from the Loci directory!):

\begin{verbatim}
LOCI_TARGET=/home/test/OBJ LOCI_COMP=gcc3.0 ./tmpcopy
\end{verbatim}

The possible compilers that will work with Loci are:  

\begin{center}
\begin{tabular}{|l|l|}
\hline
{\tt LOCI\_COMP} & Description \\
\hline
\hline
{\tt KCC }      & Kuck and Associates C++ compiler \\
{\tt gcc }      & gcc 2.95.3 \\
{\tt gcc3.0}    & gcc 3.0.3 \\
{\tt CC\_sgi}    & SGI's CC compiler \\
\hline
\end{tabular}
\end{center}

Once you have run the {\tt tmpcopy} command, you probably should
inspect the {\tt comp.conf} and {\tt sys.conf} files in the target directory.
The {\tt comp.conf} file configures for the particular compiler that you
have selected.  You should check that the path to the compiler is
correct.  The {\tt sys.conf} file configures the particulars of any
given system.  You will need to make sure that the paths to libraries
are correct and that the installation directory ({\tt INSTALL\_DIR})
points to the location where you want to install.  To install Loci
simply type:

\begin{verbatim}
make
make install
\end{verbatim}


\section{ Compiling Loci Programs }

The most direct way to compile Loci programs is to use the Makefile
template provided in this tutorial.  It is usually as simple as
including the Loci.conf file that comes as part of your Loci
installation.  See the following makefile for an example.

\include{Makefile_ex}

\section{Loci Initialization}

Before any of the main Loci functionality can be used (that is the
components that follow this section), Loci must be initialized.  Loci
has an initialize function that must be called before executing Loci
functionality and a finalize method that must be called just before
exiting the program.  For example, see below:

\begin{verbatim}
#include <Loci.h>

int main(int argc, char *argv[]) {
   // Initialize Loci
   Loci::Init(&argc, &argv) ;

   // ...
   // Loci Program
   // ...

   // Before exiting, call finalize to let Loci clean up.
   Loci::Finalize() ;
   return 0 ;
}
\end{verbatim}



\section{Running an Example}

Most of the computations discussed in this document are part of an example
Loci implementation of the finite-volume method applied to the
two-dimensional heat conduction problem.  You will find this code
under the {\tt lfvm} directory provided with the tutorial.  If you
compile the {\tt lfvm} tutorial you will be able to run example
queries to see the computations proceed for the rule we have just
described above.  We can run the above program by executing {\tt lfvm}
with the flag {\tt -q area} to query for area.  The resulting query
produces output such as:

\begin{verbatim}
% ./lfvm -q area example_grid
reading grid file example_grid
node_alloc = ([0,24])
triangle_alloc = ([25,60])
constraint = BC_bottom nodes = ([0,3])
constraint = BC_left nodes = ([0,0][7,7][14,14][21,21])
constraint = BC_top nodes = ([21,24])
constraint = BC_right nodes = ([3,3][10,10][17,17][24,24])
found 48 internal edges, 12 on boundary
max_iteration=10000
heat flux boundary nodes = ([21,24])
temperature prescribed boundary nodes = ([0,3][7,7][10,10][14,14][17,17][21,21][24,24])
generating dependency graph...
Adding rule gradDotN(T)<-(cl,cr)->(centroid,T)
setting up variable types...
decomposing graph...
existential analysis...
creating execution schedule...
area = 
{([25,60])
0.25
....
\end{verbatim}

We can see the execution schedule produced by the Loci query by
examining the file {\tt ./schedule} provided by the {\tt lfvm} example
program.  For example:

\begin{verbatim}
% more .schedule
allocate all variables
create 1 threads
thread barrier ()

area<-triangle_nodes->pos  over sequence ([25,60])
thread barrier area

destroy threads
\end{verbatim}

As we can see, the schedule is simple:  we have called the compute
method for our area rule and passed it a sequence that includes all
triangles in our grid.




\section{Entities, Sets, and Sequences}

In Loci, computations are represented by associating (binding) values
(attributes) to entities.  Although entities can be considered in
rather abstract terms, in Loci we often will often interchange the
meaning of entity with the integer identifier that is used to label a
given entity.  Thus we may talk of entity $1$ when we are really
referring to the entity labeled $1$.  (Do not count on this remaining
true indefinitely.  We plan to change Loci to make entities
first-class objects.) 

It is useful to consider groups of entities that have similar
attributes.  In Loci we have two data structures for representing sets
of entities:  1) the {\tt interval} and 2) the {\tt entitySet}.  For
example, if we wish to represent the entities labeled from $1$ to
$100$ we would use the Loci class {\tt interval(1,100)}.  On the other
hand, the {\tt entitySet} can be used to represent arbitrary
collections of entities.  Once we have described a collection of
entities using the {\tt entitySet} class we can also create new sets
of entities using unions, intersections, and other useful set
operations.  

The {\tt entitySet} class provides a true set semantics.
That is, ordering of insertion is not preserved and there is no
duplication.  Either an entity is in the set or it is not.  If we need
to preserve the order of entities for looping or other control then we
use the {\tt sequence } class.  The {\tt sequence} class provides
operations for concatenation and reversal and can be thought
of as a list of entity labels.  It should be noted that users
generally don't create sequences in Loci, but rather the scheduler
generates sequence of entities for computations.  However, if there is
ever a need to keep track of a particular ordering of entities, then
sequences are the data-structure that accomplishes this task.

The following program segment (included with the tutorial programs)
provides examples of how to create and use the {\tt  entitySet} and
{\tt sequence} classes.  This provides examples of the most commonly
used programming techniques for these classes.

\include{entities_cc}

\section{Loci Containers}

In Loci, containers are entity based.  That is, a container provides
an association between entities and values.  There are two basic types
of containers, parameters and stores.

Parameters provide a way of associating a single value with a set of
entities.  With respect to the set of entities that they are associated
with, parameter variables behave much like global variables.  There are
two parameter containers: param and blackbox, which differ in their
scope and intended usage.  The param variables are synchronized over all
of the processors and can only be used with first class objects that Loci
knows about.  The blackbox variables are not synchronized over processors
and can hold any data type.  Blackbox containers are intended to be used
to hold data structures for third party libraries, or other data types
that Loci is not able to manage directly.

Stores provide a one-to-one correspondence between entities and values.
In shorter terms, stores look like very flexible arrays.  The stores
come in a variety of forms that allow various types of run-time selection
of the sizes of the types they contain.  For example, the storeVec
provides a store that contains vectors whose size isn't specified until
run time.

Perhaps the fastest way to learn how to apply Loci containers is to see
them in action.  The following is a short program that illustrates the
syntax of Loci containers and some basic examples of their application.

\include{containers_cc}

\section{Loci Relations}
In addition to containers, Loci provides ways of describing
relationships between entities.  The simplest of these relationships
is the constraint.  The constraint simply identifies a grouping of
entities and is used to assign attributes to entities.  For example,
a boundary condition may be specified by placing those entities in the
boundary in a boundary constraint.  The most basic of the relations is
the Map.  This provides a means of describing the relationship between
entities.  Maps are used to describe the data-structures typically
encountered in unstructured mesh computations and come in a variety of
forms in Loci.  The basic {\tt Map} class provides a one-by-one
correspondence whereas the {\tt MapVec} templated class provides a
one-to-many correspondence.  Similarly, the {\tt multiMap} class
provides the most generic interface where each entity may have 0 or
more other entities related to it.  The following source code
demonstrates the basic user interface for the Map family of classes.


\include{relations_cc}



\chapter{Representing and Manipulating Data Structures}
\section{Creating Data Structures in Loci}

In Loci, data structures are represented as relations between entities
using Maps.  These data structures, and associated initial values, are
stored in a repository called the fact database which is managed by
the {\tt fact\_db} class.  The fact database is what Loci uses to
coordinate values between computations.  All Loci containers, (e.g.
store, Map, etc.) can be registered into the fact database.

For example, we create a fact database that contains the fact ``value'' using the {\tt create\_fact} member function as follows:
\begin{verbatim}
      // Create a Fact Database
      fact_db facts ;

      // Create a new value
      store<double> value ;
      // ...
      // ... Read In Values
      // ...
   
      // Insert value into the fact database
      facts.create_fact("value",value) ;
\end{verbatim}

Once containers are registered in the fact database, we can later
extract the values that were placed in the fact database using the 
{\tt get\_fact} member function.  For example, we can obtain the
current values from the fact database as follows:
\begin{verbatim}
      store<double> value(facts.get_fact("value")) ;
\end{verbatim}

Also if we wish to allocate new entities, we can ask the fact database
for a new allocation by calling the method {\tt get\_allocation} with
an argument of the number of entities that you need allocated.  For
example:
\begin{verbatim}
      entitySet node_entities = facts.get_allocation(number_of_nodes) ;
\end{verbatim}

For a more detailed example, refer to the subroutine listed below
which reads in a file that defines a triangulated mesh.  The mesh file
consists of a list of points in 2-D space and a list of triangles that
are formed using those points.  This file reader installs the node
positions and the triangle definitions in the fact database passed
into its argument list.  The mesh is defined by two Loci containers.
One is a {\tt store} that contains 2-D vectors named ``pos'' that
defines the nodal positions and a {\tt MapVec} named
``triangle\_nodes'' that defines the three nodes that form a triangle
in a counter-clockwise ordering.

\include{grid_reader_cc}

\section{Transforming Data structures in Loci}

We may need to transform data structures from one form to another
before they are useful for computations.  For example, if we are
to use the mesh of triangles read in the previous section in a
finite-volume based algorithm, we would usually need an edge-centric
data structure rather than a cell-centric data structure that the
triangle definition provides.  So, we would like to convert this
triangle based data structure to one that consists of edges.  Each
edge is defined by two nodes and two cells on either side.  Boundary
edges will be require special treatment since there is no outer
triangle.  We will create a ``ghost'' cell in these cases to make the
data structure consistent for all edges.  Also, we will want to make
sure that each edge is represented in this data structure only once.

We create this edge-based data structure by looping over the three
edges of each triangle, and by searching neighboring triangles to find a
triangle that shares the same edge.  Once we find this triangle, then
we know the two nodes and two cells that form an edge.  In order to
make sure that we don't find the same edge twice, we mark triangles
that we visit and only insert the edge we find when none of the
triangles have been visited before.  If we can't find a matching
triangle, then we know the edge must be a boundary edge that requires
a ghost cell allocation.

In order to search only the neighboring triangles, we need to
transpose the ``triangle\_nodes'' in order to obtain a mapping from
nodes to the triangles that are defined using that node.  A transpose
of a map is found using the {\tt inverseMap} function.  This function has
four arguments.  The first is a {\tt multiMap} that is returned by the
function.  The second is the {\tt Map} that will be transposed.
Following this are the entities for which the transposed map will be
defined (in this case it will be all the nodes in the problem).  The
final argument is the entities that describe the region of the map
that will be transposed.  In this case, we are interested in
transposing the {\tt triangle\_node} map for all triangles.  Therefore
to get the map from nodes to all neighboring triangles we execute the
function 
\begin{verbatim}
  Loci::inverseMap(nodes2tri,triangle_nodes,node_set,triangle_set) ;
\end{verbatim}

Once we have the map from nodes to triangles, we can choose one of the
nodes of an edge to search for the triangle that contains the same
edge.  Once we have set up the preliminaries, we use a vector to store
each edge as we find it.  Once we know how many edges are in the mesh,
we can allocate entities for the edges and create the maps
representing the edge data structure.  For this we create three maps,
``cl'', ``cr'', and ``edge\_nodes'' representing the left and right
cells and the two nodes that define the edge.  The following code
shows how to set up these data structures using Loci.

\include{setup_edges_cc}

\chapter{Computational Rule Specification}

In Loci, computations are performed by executing rules.  At a high
level, it is useful to think of Loci as a ``make'' program for
managing simulation computations instead of compilations.  A Loci rule
includes a documentation section that describes the values that it
depends upon and the values that it produces, combined with a
computation method that can perform the documented computation when
needed.  The following sections will describe how to perform various
types of computations using Loci rules.

\section{Rule Signatures}

Before we begin describing Loci rules, we should first describe how we
name Loci rules.  Rule signatures are the names of Loci rules.  Loci
will only allow one computation of a given rule signature, so we can
use rule signatures to identify any given computational component in a
Loci program.  The simplest rule signature just shows outputs of the
rule followed by the inputs.  The inputs and outputs are divided by
the ``{\tt <- }'' symbol.  For example, a rule that inputs values {\tt
  A} and {\tt B} to produce output {\tt C} has a rule signature as
follows:

\begin{verbatim}
C<-A,B
\end{verbatim}

If relations (such as the {\tt Map} {\tt cl}
and the {\tt Map} {\tt cr} in the edge-based data structure of the
previous section) are used in a computation, then they are included 
in the rule signature.
For example, a computation that computes a value {\tt D }
by averaging value {\tt C } on both sides of any given face would
be documented by the rule signature:

\begin{verbatim}
D<-(cl,cr)->C
\end{verbatim}

What this rule signature means is that {\tt D } is computed by
accessing {\tt C } through the relations identified as 
{\tt cl } and {\tt cr }.  Note that in these rule signatures, 
the comma binds more weakly than
the mapping right arrow operator.  The
parentheses, therefore, are required.

\section{Rule Databases}

Rules are managed through the use of a rule database class called
``{\tt rule\_db }''.  Generally users put rules in the rule database
from the {\tt global\_rule\_list }, a list of rules that is created
before {\tt main} is called by {\tt register\_rule } templates.  As a
result, the usual way that a rule database is manipulated in Loci is
to insert all of the rules in the global rule list.  This is
accomplished with the following code segment: 

\begin{verbatim}
  ////////////////////////////////////////////////////////////////////
  // rule_db and global_rule_list are defined in Loci.h
  ////////////////////////////////////////////////////////////////////
  // Create a rule database called rdb
  rule_db rdb ;

  ////////////////////////////////////////////////////////////////////
  // Add all of the rules that were inserted into the global_rule_list
  // by register_rule<> types into the rule database rdb
  ////////////////////////////////////////////////////////////////////
  rdb.add_rules(global_rule_list) ;
\end{verbatim}

\section{Creating an execution schedule}

Once we have a database of facts and a database of computational
rules, Loci can use these databases to satisfy queries.  We obtain an
execution schedule that can compute these results using the function
{\tt create\_execution\_schedule} provided by Loci.  This function
takes three arguments, 1) the rule database of computations to use for
this query, 2) the fact database of data structures and initial
values, and 3) a C++ string that contains the variable name that we
wish to query for.  This function returns an {\tt executeP} which is a
pointer to the execution schedule.  {\tt executeP} is a counted
pointer that will automatically delete the memory allocated to the
schedule when the object is destructed.  If a schedule could not be
determined, {\tt create\_execution\_schedule} returns a null pointer.  

We can then execute the schedule by calling the execute member
function of the execution schedule, as in
\begin{verbatim}
  schedule->execute(facts) ;
\end{verbatim}

As an additional detail, if we wish to execute on a distributed memory
machine, we must first distribute the fact database across processors
before beginning the scheduling.  This is accomplished using the {\tt
  generate\_distribution} and {\tt distribute\_facts} functions
provided by Loci.  See the below code segment for a more complete
example.

\begin{verbatim}
  ////////////////////////////////////////////////////////////////////
  // Here we distribute the fact database, if we are running on
  // multiple processors.  If we are running serially then these
  // operations have no effect.
  // First we obtain a partition of entities based on the current
  // rules and facts
  ////////////////////////////////////////////////////////////////////
  std::vector<entitySet> partition = Loci::generate_distribution(facts,rdb) ;
  ////////////////////////////////////////////////////////////////////
  // Now we use this distribution to partition the facts to processors.
  // The assumption at this point is that every processor has 
  // identical facts and rdb.
  ////////////////////////////////////////////////////////////////////
  Loci::distribute_facts(partition, facts, rdb) ;

  ////////////////////////////////////////////////////////////////////
  // Here we ask Loci to create an execution schedule that will use
  // the rules in the rule database ``rdb'' and the data in the
  // fact database ``facts'' to obtain the variable(s) specified in
  // query.  Here query is a c++ string that contains the name of
  // the variable that we are querying for (or a comma separated
  // list if we wish to query for more than one variable).
  ////////////////////////////////////////////////////////////////////
  executeP schedule = create_execution_schedule(rdb,facts,query) ;

  ////////////////////////////////////////////////////////////////////
  // If Loci is unable to derive an execution schedule it will return
  // a null pointer.
  if(schedule == 0) {
    // Output a diagnostic if no schedule can be obtained
    cerr << "unable to produce execution schedule to satisfy query for "
         << query << endl ;
  } else {
    //////////////////////////////////////////////////////////////////
    // here we actually execute the computation schedule.  The
    // result will be placed into the fact database ``facts'' for
    // later extraction.
    //////////////////////////////////////////////////////////////////
    schedule->execute(facts) ;

    //////////////////////////////////////////////////////////////////
    // Here the execution is complete and the requested computations
    // are in facts.
  }
\end{verbatim}




\section{Pointwise Computations}

A rule is implemented as a class that provides a constructor that
documents the inputs and outputs to the computation and a virtual
compute method that provides computations for arbitrary collections of
entities.  The most fundamental of these computations is the pointwise
rule.  The pointwise rule represents a computation that can be applied
individually, entity by entity.  For example, consider the case of
computing the area of each triangle in the triangular mesh described
earlier.  In order to compute the area of any given triangle we need
to access the positions of the triangle's three nodes.  These node
positions can be accessed using the {\tt triangle\_nodes} map.  The
areas of all of the triangles in the mesh can be computed by looping
over all entities that have the {\tt triangle\_nodes} defined.  More
precisely, not only must {\tt triangle\_nodes} be defined, but the
entities that this map refers to must also have the attribute {\tt
  pos}.  This set of entities is called the context of the rule and
represents the possible entities over which the computational
subroutine may be called.  A rule that computed areas in such a
fashion would have a rule signature of
\begin{verbatim}
area<-triangle_nodes->pos
\end{verbatim}
where area would be the attribute that this computational routine
would provide when the inputs are provided.  

In order to define a pointwise rule, the user defines a class that
inherits from the class {\tt pointwise\_rule} that is provided in {\tt
  Loci.h}.  This class will contain the {\tt store}, {\tt Map}, and
{\tt param} containers that will be used for the computations.  Input
containers should use the {\tt const\_} prefix which establishes that
the container can only be accessed in a read-only mode.  The
constructor of the rule is responsible for registering the containers
used in the computations and documenting the inputs and outputs of the
rule.  The containers are registered using the rule member function
{\tt name\_store()}.  This function attaches a symbolic name (string)
to the containers so that they can be attached to containers stored in
any given fact database when an execution schedule is formed.
Additionally, {\tt input} and {\tt output} methods are provided to
document both input and output data of the given computation
encapsulated by the rule.  Below, a simple example shows how to create
a pointwise rule to compute areas and centroids of triangles in the
triangular mesh.


\include{cell_props_cc}

\section{Reduction Computations}

There is an alternative approach that we could have taken in the
computation of areas.  Instead of looping over triangles, instead we
could have looped over edges and computed the area contribution
associated with each edge as illustrated in figure \ref{fig:area}.  We
define the edge area contribution as the area of the triangle formed
by the two edge nodes and the centroid of the cell.  Obviously the
triangle area is equal to the sum of all of the edge area
contributions.  This computation method has the advantage that it can
be made to compute the area of any general polygon, whereas the
previous approach was limited only to triangles.  Also, since a
natural way of expressing this computation is to loop over edges
(which is typically what is required in finite-volume computations) we
may be able to dispense with storing the relation {\tt
  triangle\_nodes} all together.  However, if we wish to define this
computation via iteration over edges, we will need to sum up the
results incrementally.  As a result we won't be able to express this
computation as a pointwise rule since the computation of area does not
occur completely point by point, but instead is spread out over
several edge updates.

For this style of computation we need to describe a reduction.  In
Loci we describe a reduction through the use of {\tt unit\_rule} and
{\tt apply\_rule}.  The {\tt unit\_rule} defines the identity of the
operator over which we are doing the reduction, while a set of {\tt
  apply\_rule}s define the summing update that we perform as we are
adding each edges contribution to its neighboring triangles.
Following is an example program that shows how to compute areas using
this edge-centric approach.

\begin{figure}[h]
\centerline{
\epsfxsize=3.2in
\epsfbox{figures/edge_area.eps}}
\caption{Computing Triangle Area by Accumulating Edge Contributions}
\label{fig:area}
\end{figure}

Note that reductions can also be used to compute parameter values.
The specification of the reduction is the same as when the output is
a store, and the resulting parameter is the reduced (accumulated)
value.  In fact, an {\tt apply\_rule} is the only valid rule for
computing parameters from stores.  See the file {\tt stable.cc} in the
finite-volume example for an illustration of how to specify reductions
for computing parameters.

\include{area_reduce_cc}

\section{Singleton Rules and Parameter Computations}

Singleton rules are defined for computations that are performed
exclusively on parameters.  Since these computations are on single
values that are associated with a group of entities, it is not
necessary to perform any loops.  However the attribute that is
computed as a result of the rule is only associated with the entities
that are in the intersection of the entities given the attributes of
all of the inputs.  (Note:  In the parallel implementation, all
parameters are duplicated on each processor, so these singleton rules
are computed on each processor when parallel processing is used [no
communication is invoked].)

\begin{verbatim}
//////////////////////////////////////////////////////////////////////////////
//
//
//  Here we compute a new parameter C by multiplying the values contained in
//  parameters A and B.
//  
//  We define a singleton rule by creating a class that inherits from 
//  the class singleton_rule.
class singleton_example : public singleton_rule {
// Here we define the paramters involved in the computations.
// ** Note, only parameters allowed in singleton computations!
  const_param<double> A ;
  const_param<double> B ;
  param<double> C ;
public:
// Here we provide a constructor that names stores and specifies inputs and
// outputs as before.
  collapse_info() {
    name_store("A",A) ;
    name_store("B",B) ;
    name_store("C",C) ;
    input("A,B") ;
    output("C") ;
  }

  virtual void compute(const sequence &seq) {
// Now in the compute method, we don't loop since we only have a single
// value representing the entire set of entities.  Instead we use the
// dereference operator (*) to access the values.
    *C = (*A)*(*B) ;
  }
} ;

// Now we can register this rule like any other Loci Rule
register_rule<singleton_example> register_singleton_example ;
\end{verbatim}

\section{Iterative Computations}


Iteration is defined by way of three types of rule specifications:
build rules that construct the iteration, advance rules that advance
the iteration, and collapse rules that terminate the iteration.  This
specification follows an analogy to the inductive proof in that build
rules are analogous to an inductive base while advance rules are
analogous to an inductive hypothesis.


Iterations are specified by adding an iteration label to variable
identifiers.  Iteration labels are organized into a hierarchy that is
rooted at stationary time (values that don't iterate).  It is assumed
that computations can proceed at any given iteration level while
accessing values computed at either its iteration level or at parent
levels in the label hierarchy.  The iteration label is just a list of
iterator variables that follows the variable name enclosed in braces.
For example, to access value {\tt v} for iteration {\tt n} we have 
{\tt v\{n\}}.  This relationship between traditional imperative
languages' 
loop nesting and the iteration label hierarchy is shown in figure
\ref{fig:nested}. 

\begin{figure}[h]
\centerline{
\epsfxsize=3.4in
\epsfbox{figures/nested.eps}}
\caption{Nested Loops are characterized by a hierarchy of iteration labels}
\label{fig:nested}
\end{figure}

For example, an iteration where a variable named {\tt q} is iterated
to a converged solution may be described by the following three rules:
1) a build rule of the form {\tt q\{n=0\}<-initial\_condition}, 2) an
advance rule similar to {\tt q\{n+1\}<-q\{n\},delta\_q\{n\}}, and 3)
an iteration collapse rule {\tt
  solution<-q\{n\},CONDITION(converged\{n\})}.  Iteration in this
example proceeds by initializing the first iteration, {\tt q\{n=0\}},
using the build rule.  Next, termination of iteration is checked
by computing {\tt converged}.  If the test succeeds then the collapse
rule terminates the iteration.  Finally the iteration advances in time
by the repeated application of the advance rule which computes values
for {\tt q} for the next iteration ({\tt\{n+1\}}) given current
iteration values at time level {\tt\{n\}}.  Note that the completion
of these rules may require invoking other rules specified in the rule
database.  In this case, rules that compute {\tt converged\{n\}} and
{\tt dq\{n\}} will also need to be scheduled.

To support iteration, variables that exist in lower levels of the
iteration hierarchy are automatically promoted up the iteration
hierarchy.  Thus a variable that is computed in iteration {\tt\{n\}}
is communicated to iteration {\tt\{n,it\}} automatically.  In
addition, rules that are specified completely at the stationary level
will be promoted to any level of the hierarchy.  This allows for the
specification of relations that are iteration independent (for
example, $p = \rho \tilde{R} T$ implies $p^n = \rho^n \tilde{R}^n
T^n$).


\include{integration_cc}

\section{Parametric Rules}

We can also create generic, or parametric rules.  These can be any
type of rule and allows Loci to create new rules by variable
substitution.  The output of a parametric rule includes variable names
that are followed by parenthesis enclosed parameter lists.  When these
parameters occur in the inputs to the rule, they will be substituted
by the provided parameter.  See the gradient routine that follows for
an example of how to develop and use parametric rules.

\include{gradient_cc}

%\chapter { Database}

A database is the fundamental starting point for any logic 
programming systems. The definition of a problem to be solved begins
as a collection of facts stored in a database, while the result of a
rule applications is to modify the existing and/or creation of new
fact databases. Thus the database becomes a center of communication
for program derived from the specification.
%
\section { Facts and Facts database }

A Facts in Loci is a property associated with an entity. This property
could either be in the form of data or reference to some other
entities (maps). 

If the property is a map, it has to provided by the user before the
execution of the program.

If the property is a data, it could either user defined, or derived
using know facts.  In Loci, all facts are stored in the database,
hereafter called {\bf facts database} In general there are four types
of facts within the Loci system which are

\begin{enumerate}
\item Parameter
\item store
\item maps
\item constraint
\end{enumerate}

\section { Rules and Rules database }
In addition to a database of facts that includes the problem
specification, a database of rules describes transformations that can
be used to introduce new facts into the database. These rules
correspond to fundamental computations involved in solution algorithms
such as rules for evaluating areas of faces, or for solving equations
of state.  These rules are specified using text strings called rule
signatures that describe the input stores, parameters, and maps
required to perform a computation and the list of stores or parameters
that it generates.  Rule signatures are of the form {\tt head <- body}
where head consists of a list of variables that are generated by the
application of the rule, while the body contains a list of variables
that are accessed while performing the computation.  For example, the
rule signature {\tt p<-rho, T, R} represents that a value for pressure
({\tt p}) is provided when values for density ({\tt rho}), temperature
({\tt T}), and gas constant ({\tt R}) are present.

\begin{figure}[h]
\special{psfile=rule.eps vscale=50 hscale=50 voffset=-175 hoffset=125}
\vspace{2.50in}\caption { A Rule produces new facts }
\end{figure}

As shown in the above figure, a rule requires some input facts and 
produces new facts. In Loci, input facts cann't be modified, therefore,
they should be declared as read-only facts.

\par In Loci, rules can be classified in four basic rules, these are
\begin{enumerate}
\item {\bf Singleton Rule :} A rule which accepts parameters as input and 
generates a single value for an entitySet.
%
\item {\bf Pointwise Rule :}  A rule which operates on each entity independently and
produces for its own entitySet or for other entitySet over which the rule is applied.
%
\item {\bf Reduction Rule :}  A reduction rule is specified for reduction operations. 
A reduction operation works on a set of entities to produces a single value. A reduction
A reduction rule is an application of two rules
\begin{enumerate}
\item Unit Rule  A rule which initialize the values for reduction operation. 
\item Apply Rule A rule which applies operation over an entitySet.
\end{enumerate}
%
\item {\bf Iteration Rules :}  An iterative process is an set of rules, which requires
%
\begin{enumerate}
\item {\bf Build Rule(BR) :} A rule  which start an iteration hierarchy with initialization
of the values.
\item {\bf Advance Rule(AR) :} A Rule which specifies how to advance in the iterations.
\item {\bf Collapse Rule(CR):} A Rule which specifies what to do at the end of iterations.
\end{enumerate}
\end{enumerate}
%

\section {name\_store}
Every rule work on the facts to change or produce new facts. Rules and
facts are described in the next chapter. The {\em name\_store} specify the 
facts which are required by the rule. Although
no harm will be done, if you specify a fact which is not used to create
a new fact, but we should avoid to make the objective of the rule more
concise and elegant.

\par Another way we can think of class {\em name\_store} is that it opens the
files in a database. The syntax of {\em name\_store} is
\begin{center}
name\_store ("Fact", local\_store\_variable\_name)
\end{center}
where "Fact" is the variable name of the fact which is associated with the database,
and {\em local\_store\_variable\_name} is the locally defined name 
of the fact in a rule. 

\par Some conventions for defining facts name in the database.

\begin{itemize}
\item  It is case sensitive
\item  It is not a string, but a variable like any other variable, therefor
       "Fact-New", "Fact+New" are invalid names.
\item  This variable is different from locally defined variable, therefore 
       it is more elegant and acceptable to provide  name\_store("Fact", Fact)
       both the variables seems to have same name, but they are in different
       context.
\end{itemize}


\section  { A general skeleton of Loci Rule }
\begin{verbatim}
class newRule : public rule_name 
{
public:
  newRule() { 
     name_store( "Fact1", infact1);
     name_store( "Fact2", infact2);
     name_store( "Fact3", infact3);
     .
     .
     name_store( "Fact4", outfact1);
     name_store( "Fact5", outfact2);
     .
     .
     input( "Fact1");
     input( "Fact2");
     input( "Fact3");
     .
     .
     output( "Fact4");
     output( "Fact5");
     .
     .
  }
  void calculate( Entity e)
  {
     // Specify the formula to calculate new fact here.

  }
  virtual void compute( const sequence &seq)
  {
    do_loop( seq, this );
  }
private:
  const_store<datatype>    infact1, infact2;
  const_param<datatype>    infact3;

  store<datatype>          outfact1, outfact2;

};
register_rule <newRule>  register_newRule;
\end{verbatim}

\par Here datatype could be any datatype either standard or user-defined.

\section { Important things about rule specification }
\begin{itemize}
\item In Loci, any fact, which is not output of the rule specification
has to be declared const type. This allows strict checking of the intention of the fact
within a rule. For example, use const\_store instead of store.
%
\item A rule can have multiple inputs and multiple outputs facts.
%
\item All the rules must be registered before they can be used. 
%
\item The member function {\em compute} accept sequence as an 
argument which is generated by Loci schedule, by analyzing the rule signature 
%
\item The member function {\em calculate} applies rule for each entity
supplied by the sequence.
\end{itemize}
%
\section {Context of the Rule}
When we write a rule to generate a new facts, we have to apply the rule over an 
entitySet.  A context of the rule is defined as the set of entities, where a particular 
rule is applied.  
%
\section { Singleton Rule }
In Loci, computations on parameters are accomplished using Singleton Rule.
%
\begin{figure}[h]
\special{psfile=single.eps vscale=50 hscale=50 voffset=-250 hoffset=125}
\vspace{3.50in}\caption { A singleton rule produces a parameter fact}
\end{figure}
%
\par {\bf Example :} Suppose we are writing an iterative scheme and we 
would like to terminate the iterations when the iteration counter has
reached to the maximum which is specified by the user. We define a parameter named 
{\em smoothOver}, which is used as an indicator of the progress. If its value is 1, then the
iterations are stopped. Since this parameter is same for all the entities,
it is implemented as {\bf Singleton Rule }
\begin{verbatim}
class SmoothingOver_condition : public singleton_rule {
  const_param<int> n, max_iteration ;
  param<bool>      smoothingOver ;
public:
  SmoothingOver_Condition() {
    name_store("$n",n) ;
    name_store("max_iterations",max_iteration);
    name_store("SmoothingOver",smoothingOver);
    
    input("$n,max_iterations") ;
    output("SmoothingOver") ;
  }
  virtual void compute(const sequence &seq) {
    *smoothingOver = (*n >= *max_iteration);
  }
} ;
\end{verbatim}

\section{ Pointwise Rules }
A pointwise rule operates on each entity independently and produces new facts. This
rules is sometimes called {\em Rule of Production }.  
One important thing remember is that {\bf Pointwise rules they are unique
to Loci} and if you try to provide two rules for the generation of same new fact,
then Loci, will get confuse and will not proceed.
%
\begin{figure}[h]
\special{psfile=point.eps vscale=50 hscale=50 voffset=-175 hoffset=125}
\vspace{2.50in}\caption { A pointwise rule produces new facts for each entity  }
\end{figure}
%
\par {\bf Example : } In the previous example, given the node-coordinates and 
neighbor nodes information, we would like to calculate the area of each
cell. Obviously, are calculation of one triangle doesn't depend on the 
calculation for the other triangle, therefore the rule for generating
area of a triangle is a {\em pointwise rule}
\begin{verbatim}
1  class triCentroid : public pointwise_rule{
2    store<VECTOR>         centroid;
3    const_MapVec<3>       triConnect;
4    const_store<VECTOR>   pos;
5  public:
6    triCentroid ()
7    {
8      name_store( "Centroid", centroid );
9      name_store( "triConnect", triConnect);
10      name_store( "pos", pos);
11
12      input ("triConnect->pos");
13      output("Centroid");
14    }
15
16    //---------------------------------------------------------------------
17    // The following is the rule to calculate the centroid of triangle..
18    //---------------------------------------------------------------------
19    void calculate( Entity e)
20    {
21      int    n1, n2, n3;
22      n1            = triConnect[e][0];
23      n2            = triConnect[e][1];
24      n3            = triConnect[e][2];
25      centroid[e].x = (pos[n1].x + pos[n2].x + pos[n3].x)/3.0;
26      centroid[e].y = (pos[n1].y + pos[n2].y + pos[n3].y)/3.0;
27    }
28
29    virtual void compute( const sequence &seq)
30    {
31      do_loop( seq, this );
32    }
33  };
\end{verbatim}
%
\section {Recursion}
A {\bf Recursive definition} is a statement in which something is defined
in terms of {\bf smaller version of itself}. All recursive definition 
consists of two parts
\begin{enumerate}
\item {\bf Base case} is that part of the definition which {\bf cannot} 
be expressed in in terms of smaller version of itself. 
\item {\bf Recursive case} is that part of the definition which can
be expressed in terms of smaller version of itself.
\end{enumerate}

%**********************************************************************************************
%Consider a Russian Doll which consists of N nested dolls of decreasing sizes. One might 
%originally ask, "How many dolls actually exist in a given Russian Doll?"   The largest and 
%outermost doll might then reply saying, "Well, there's 1 of me and let me find out how 
%many dolls there are inside of me."   Turning to the next outermost doll, the largest doll 
%asks, "How many dolls are there?"   The answer of the outermost doll to the original 
%question is suspended while the next outermost dolls seeks an answer.   The next outermost 
%doll might reply saying, "Well there's 1 of me and let me find out how many dolls there are 
%inside of me."   Turning to the next doll, the next outermost doll asks, "How many dolls 
%are there?"   And the process continues until the questioning reaches the innermost doll 
%where that doll replies, "There's only 1 of me."   This then allows the next innermost 
%doll to reply, "There are 2 dolls."   And so on, until finally the outermost doll is 
%able to reply, "There are N dolls." 

%The corresponding recursive pseudo-function might look as follows : 
%\begin{verbatim}
%int function RussianDoll 
%{ 
%    int doll ; 
%    // base case 
%    if ( there's only one doll ) 
%       return ( 1 ) ; 
%
%    // else recursive case 
%    doll = 1 + RussianDoll ; 
%    return ( doll ) ; 
%} 
%\end{verbatim}
%After the questioning reaches the innermost doll and the process reverts back to the next 
%doll until it reaches the original inquirer, that process is referred to as back tracking . 
%**********************************************************************************************

One important thing about the recursion is the {\bf Recursion follows entities in a particular
sequence}, unlike pure pointwise rules which don't depend on the sequence order. This suggests 
us that if we are able to order the sequence for pointwise rule, we might get a correct results 
for recursion. Let us use an example to demonstrate how we can put entities in a some order. 

\par In this example, we will create some entities and enumerate them in increasing order
$(1 \dots n)$. We want to create two sequences
\begin{enumerate}
\item  {\bf Increasing Order }  Entities are traversed as $(1,2,3, \dots n-1,n)$
\item  {\bf Decreasing Order }  Entities are traversed as $(n-1,n-2,  \dots 2,1)$
\end{enumerate}

The way we implement this sequence is with the use of Map. In the first case, we define
a map {\em predecessor} which maps the immediately previous entity to the entity $e$. 
Only the first entity don't have the {\em predecessor} and this becomes the base of 
recursion.  In the second case, we define a map {\em successor } which holds the 
immediate next entity to an entity $e$. In this case, the last entity will not have
any mapping, and therefore, becomes the base of iteration. In both the cases, we will 
need two pointwise rules, one for the base, which require only one entity to participate 
in the rule invocation, and this could be enforced using the constraint environment and
the second rule will be invoked recursively till the end.

\par Now we will write programs which implements both the methods to create the sequence.

\section{ Reduction Rule }
\par A {\em Reduction Rule } is a rule, which operates on a set of entities
and returns a single value as a result. For example, the following C++
program uses operator *, to calculate the product of an array {\em someArray}
\begin{center}
\begin{verbatim}
prod = 1.0;
for( i = 0; i < arraySize; i++)
     prod  *=   someArray[i];
\end{verbatim}
\end{center}
\par In the above, a variable {\em prod }is initialized to 1, then a loop is
run over every element of an array, and its values are multiplies sequentially
to the variable {\em prod }


The target of reduction rules could be either single parameter or store for
each entity. For example, addition of 2 vectors produces an array, whose each
element is an obtained using reduction operation.
\begin{verbatim}
store<double>  prod;

// Initialzation : Unit Rule
for( i = 0; i < arraySize; i++)
     prod[i] = 1.0;

// Operation: Apply Rule
for( i = 0; i < arraySize; i++)
     prod[i]  =  a[i]*b[i];
\end{verbatim}
\par In the above, a variable {\em prod }is initialized to 1, then a loop is
run over every element of an array, and its values are multiplies sequentially

\section {Global v/s Local Reduction Rules}
There are two ways in which reduction operation are used.
\begin{itemize}
\item  {\bf Reduction for parameter datatype} When the reduction is used for
parameter datatype, then an entitySet produces a single value which is same
for that entitySet.
%
\par {\bf Example :} In CFD application, explicit schemes are frequently used,
If we are interested in the unsteady solution, we have to calculate CFL condition
for each entity, and then apply the reduction operation to get the global
minimum CFL number. Every entity uses this global minimum value to advance in
time. See Figure  A.
%
\item  {\bf Reduction for store datatype} When reduction is carried out over
a small subset of entities for each entity in the entitySet then the output
is a store datatype. Such a reduction is local to each entity.
\par {\bf Example :} If we are interested in steady state solution, we can
relax the stringent condition of global time step and allow each entity to 
proceed at its own time step. Therefore, we have to store the time step value 
for each entity. See Figure  B
\end{itemize}
\begin{figure}[h] 
\special{psfile=gloLoc.eps vscale=50 hscale=50 voffset=-175 hoffset=50}
\vspace{2.50in}\caption { Global and local reduction operation}
\end{figure}
\par In Loci, when a variable is defined by reduction rules, then its value
is the result of a set of rules applications. We have to define two rules
which are explained below.

\section { Unit Rule }
\par Before we apply reduction operation over an entitySet, it is necessary
to initialize the target value (values). In Loci, this is accomplished by
{\bf Unit Rule } For example, a summation operation requires initialization
of 0 value and a product operation requires initial value of 1 for each entity.
A {\bf unit rule} in Loci defines a unit value to each entity defined by
constraint environment. {\bf A unit rule is a single value assignment rule 
for the entities }, which mens that entities can not be initialized with
different values for each entity.
{\bf No value for a reduction can be generated without first binding a unit 
value to the target entitySet}. 
A {\bf unit\_rule} also has a constraint qualifier that limits
which entities are included in the computations.  

\section { Apply Rule }
Once the {\bf unit rule } has been applied to all the entities in an entity
set, we can apply reduction operation using the {\bf Apply Rule}

\par Loci provides four predefined reduction operators i.e. Summation, Product,
Maximum and Minimum. A user can define his/her own reduction operator, which we
will explain in the next examples.

\par A reduction operation is commutative that means $a\oplus (b \oplus c) =  (a \oplus b) \oplus c$.
This implies that the final result doesn't depend on the order of execution. 

\par Let us calculate the factorial value of an integer value using reduction operation.
In order to apply this rule, a fact consists of $(n,n-1,n-2, \dots 1)$
has to be created and stored in the fact database. A program in Loci will be 
as follows.
%
\begin{verbatim}
class Factorial_Unit : public unit_rule {
  param<int>  factorial;
public:
  Factorial_Unit() {
    name_store( "Factorial", factorial );
    constraint( "EntityVals");

    output("Factorial");
  }
  virtual void compute(const sequence &seq){
    *factorial = 1;
  }
};

register_rule<Factorial_Unit>  register_Factorial_Unit ;

class Factorial_Apply : public apply_rule<param<int>,
                                       Loci::Product<int> >
{
  param<int>         factorial;
  const_store<int>   entityVals ;
public:
  Factorial_Apply() {
    name_store("Factorial",  factorial);
    name_store("EntityVals", entityVals) ;

    input("EntityVals");
    input("Factorial");

    output("Factorial");
  }

  void calculate( Entity e) {
    join(factorial[e], entityVals[e]) ;
  }

  virtual void compute( const sequence &seq) {
    do_loop( seq, this );
  }
};

register_rule<Factorial_Apply> register_Factorial_Apply;
\end{verbatim}

%
\section { Defining your own reduction operators }
Although, Loci provides some frequently used reduction operations, but, it is fairly 
easy to create your own reduction operations. 
%
\par {\bf Example :} Although Loci provides {\bf Maximum } and {\bf Minimum} reduction
operations, we would like to create our own operation, which combines these two operation
into one operation, to demonstrate to create our own reduction operations.

\begin{verbatim}
class MinMax {
 public:
   void  operator() ( Array<float,2> &update, const Array<float,2> &newval)
   {
       update[0] = (float) min( (double)update[0],(double)newval[0] );
       update[1] = (float) max( (double)update[1],(double)newval[1] );
   }
};

class MyOp_Unit : public unit_rule {
  param<Array<float,2> > minmax;
public:
  MyOp_Unit() {
    name_store( "MinMax", minmax);
    constraint( "Density");
    
    output("MinMax");
  }

  virtual void compute(const sequence &seq){
    Array<float,2>   &newvec = *minmax;
    newvec[0]  =  1.0E+10;
    newvec[1]  = -1.0E+10;
  }

};

register_rule<MyOp_Unit>  register_myop_unit;

//*********************************************************************
// Notice that in the "apply rule" the value of PI has in-out intention.
// That is it value changes with this rule.
//
// This most important step is the "Join" statement, which combines the
// value of an entity using Reduction operator. In this example, we have
// used LOCI predefined operator "Summation" for the addition purpose. .
//*********************************************************************

class MinMax_apply : public apply_rule< param<Array<float,2> >, MinMax >
{
  param<Array<float,2> >   minmax;
  const_store<float>       array;
  Array<float,2>           newvec;
public:
  MinMax_apply() {
    name_store("MinMax",  minmax) ;
    name_store("Density", array);
    
    input("MinMax");
    input("Density") ;

    output("MinMax");
  }
  
  void calculate( Entity e) {
      newvec[0] = array[e];
      newvec[1] = array[e];
      join(minmax[e], newvec);
  }
  
  virtual void compute( const sequence &seq) {
    do_loop( seq, this );
  }
};

register_rule<MinMax_apply> register_MinMax_apply;
//*********************************************************************
\end{verbatim}
%
\section {Local Reduction v/s Pointwise Rule }
In global reduction all entities in an entity set participate to produce a single
value. In many cases, each entities participates only with few other entities in 
the reduction operation, such rules can be implemented with either reduction
or pointwise rules.
\begin{figure}[h]
\special{psfile=reduc2.eps vscale=50 hscale=50 voffset=-175 hoffset=100}
\vspace{2.50in}\caption { Applying reduction operation using local entities}
\end{figure}

\par { \bf Example :} In an unstructured grid, every face is shared by at the most 
two cells, and at each cell center, we would to like assign the maximum value 
around the neighboring cell, (including the cell itself). 
i.e.
\begin{eqnarray}
\phi[1] &=&  max ( \phi[1], \phi[2], \phi[3], \phi[4] ) \\
\phi[2] &=&  max ( \phi[1], \phi[2]) \\
\phi[3] &=&  max ( \phi[1], \phi[3])
\end{eqnarray}

\par Now there are two ways of doing this operation, which depends on choosing
the entities.

\begin{itemize}
\item { \bf Cells entities in the loop} In this case, we have to create a {\em multimap} which
store the cell-neighbors information. Since all the participating entities are 
known in advance, the final reduce value is immediately available. 
\begin{verbatim}

class Smooth_Rule : public pointwise_rule
{
  const_store<float>   density;
  const_multiMap       cellNeighs;
  store<float>         smooth;
public:
 Smooth_Rule() {
    name_store("CellDensity",  density) ;
    name_store("Smooth",       smooth);
    name_store("CellCellNeighbours", cellNeighs);

    input("CellDensity");
    input("CellCellNeighbours");

    output("Smooth");
  }

  void calculate( Entity e) {
     smooth[e]  = density[e];
     for( const int *ci = cellNeighs.begin(e); ci != cellNeighs.end(e); ++ci)
          smooth[e] +=  density[*ci];
  }

  virtual void compute( const sequence &seq) {
    do_loop( seq, this );

  }
};
register_rule<Smooth_Rule> register_Smooth_Rule;
\end{verbatim}
%
\item { \bf Face entities in the loop} In this case, we can create  a {\em map} 
for each internal face and associate two neighboring cells to it. In order to perform
the calculation, we have to define three rules as follows
\begin{enumerate}
\item  A rule to create unit value (=0.0) for each cell entity.
\item  A rule which iterative over all internal edges and add the values of associate
cell entities.
\item  A rule which iterate over all the cell entities and add its own value.
\end{enumerate}

\begin{verbatim}
class Smooth_Unit : public unit_rule
{
  store<float>   smooth;
public:
 Smooth_Unit() {
    name_store("Smooth",   smooth);
    constraint("Density");

    output("Smooth");
  }
  void calculate( Entity e) {
      smooth[e] = 0.0;
  }

  virtual void compute( const sequence &seq) {
    do_loop( seq, this );
  }
};

register_rule<Smooth_Unit> register_Smooth_Unit;

//*********************************************************************

class Smooth_Rule1 : public apply_rule<store<float>, Loci::Summation<float> >
{
  const_store<float>   density;
  const_MapVec<2>      edgeCells;
  store<float>         smooth;
public:
 Smooth_Rule1() {
    name_store("Density",  density) ;
    name_store("Smooth",   smooth);
    name_store("Cell_Edge_Neighbours", edgeCells);

    input("Cell_Edge_Neighbours->Density");
    input("Cell_Edge_Neighbours->Smooth");

    output("Cell_Edge_Neighbours->Smooth");
  }

  void calculate( Entity e) {
     Entity e1    =  edgeCells[e][0];
     Entity e2    =  edgeCells[e][1];
     join( smooth[e2], density[e1] );
     join( smooth[e1], density[e2] );
  }

  virtual void compute( const sequence &seq) {
    do_loop( seq, this );

  }
};

register_rule<Smooth_Rule1> register_Smooth_Rule1;

//*********************************************************************

class Smooth_Rule2 : public apply_rule<store<float>, Loci::Summation<float> >
{
  const_store<float>   density;
  const_MapVec<2>      edgeCells;
  store<float>         smooth;
public:
 Smooth_Rule2() {
    name_store("Density",  density) ;
    name_store("Smooth",   smooth);

    input( "Density");
    input( "Smooth");

    output("Smooth");
  }

  void calculate( Entity e) {
     join( smooth[e], density[e] );
  }

  virtual void compute( const sequence &seq) {
    do_loop( seq, this );
  }
};

register_rule<Smooth_Rule2> register_Smooth_Rule2;

//*********************************************************************
\end{verbatim}
\end{itemize}
\section { Every local reduction rule can be implemented as pointwise rule }
%
Whenever, the reduction operation target is store datatype, we can either
use pointwise rule or reduction rules to generate the desired result. But before
that, let us review the concept of {\bf Context of Rule}. 
\begin{center}
target $\leftarrow$ context $\rightarrow$ source 
\end{center}
When the context of the rule and the target entitySet is same, then we have all 
all the information required to generate the target data. but when the
context entitySet and target entitySet are not same, then, we don't have
sufficient information to generate the final target value for an entity. In
this case, we have to apply local reduction. 

%In general all reduction rules, could be implemented with pointwise rule also.
%Let us see one example, where we have used pointwise rule to calculate the
%global sum of an array.
%
%\begin{verbatim}
%class SumApply : public  pointwise_rule {
%{
%public:
%   SumApply() {
%     name_store("sum",  mysum);
%     name_store("Array",someArray) ;
%
%     input("Array");
%     input("sum") ;
%     output("sum");
%   }
%
%    void calculate( Entity e) {
%      sum[e] += someArray[e];
%    }
%
%    virtual void compute( const sequence &seq) {
%      sum = 0.0;
%      do_loop( seq, this );
%    }
%  };
%private:
%  param<double> sum ;
%  const_store<double> someArray;
%}
%\end{verbatim}

Notice that the initialization is done just before the {\em do\_loop}. Now
the question comes in mind, then why it is necessary to provide a different
rule. The reasons are as follows
\begin{itemize}
\item pointwise rules can be applied only when all the information is 
already available, and just for the reduction operation, it might be prohibitive
to create additional facts.
\item Reduction rules could be implemented in an efficient ways.
\end{itemize}
%
\par Another way to remember this rule is as follows: Thinks about for which
entities the rule is being applied, and if you are producing data for yourself,
you can always implement using this rule as pointwise, but if you looping over
some entities, but producing data for other entities, you have to apply 
reduction operation, because you do not know in advance how many entities, and
the sequence of entities that will contribute to the final result.

\section { Iteration Rule }
The mathematical foundation of iteration rules is the {\bf Theory of Induction},
which is used to verify the correctness of a theorem $p(n)$ for all non-negative
integers.
It has two component
\begin{enumerate}
\item {\bf Basic Step}  Prove the truth of $P(0)$. This is usually done by direct
verification. In general, the basic step merely involves demonstrating that the
the theorem is true for some smallest integer. It need not be $0$.
\item {\bf Inductive Step} Prove that for any integer n, if the $p(n)$ is true,
$p(n+1)$ must also be true. 
\end{enumerate}

\section{Naming Convention}
\par In the iteration context, any variable is identified by name, 
entity identifier, and iteration identifier. Perhaps many times
we have come across $u_i^n$ symbol, which represent variable
name $u$ associated with entity $i$ at iteration $n$. The iteration
identifier of a variable represents an iteration hierarchy. In Loci,
brace delimited iteration identifier are used to signify the iteration
identifier of a variable, thus $u^{n-1}$ is represented as $u\{n-1\}$

\par In Loci, an iteration counter is accessed as {\bf \$n} where $n$ is
the positive integer value.

\subsection { Two classical examples of iterative scheme }
Let us consider two typical iterative schemes to solve system
of linear equations $Ax=b$.
\begin{itemize}
\item {\bf Jacobi Scheme} In the Jacobi method we write for the $n^{th}$ step of
iteration 
\begin{equation}
 D. x^n = -(L+U). x^{n-1} + b
\end{equation}
\item {\bf Gauss-Seidel Scheme}
\begin{equation}
 (L+D).x^n = -U.x^{n-1} + b
\end{equation}
\end{itemize}
\par where $D$ is the diagonal part of A, $L$ is the lower triangle of A with 
zeros on the diagonal,  and $U$ is the upper triangle of $A$ with zeros on the
diagonal.
%
In the above examples, using the existing facts at previous level, or newly 
created facts, we are creating ( producing ) new facts, which is the precise
definition of pointwise rule.
%
\subsection { General Steps in an iterative scheme }
Again consider the solving $Ax=b$ described in {\em point-wise rule} section.
All iterative scheme follows the steps
\begin{itemize}
\item Initialize the values of unknown at the start of iterations (i.e. n = 0)
\item Produce new facts for the next level of iteration i.e. (n+1)
\item Check the convergence. If converged, then break the loop.
\item Update values i.e. new facts becomes old facts.
\end{itemize}

\par In Loci,  iterations are performed by set of rules. One for each step 
described earlier.  
%
\begin{figure}[h]
\special{psfile=iter2.eps vscale=55 hscale=50 voffset=-200 hoffset=5}
\vspace{2.70in}\caption { Build, Advance and Collapse Rules constitute iteration Rules}
\label {FigIteration}
\end{figure}
%
\section {Build Rules}
\par Build rules are rules that formulates the initial values for the iteration.
They are identified as build rules because they construct a new foundation level for
the iteration hierarchy. In Loci, a build rule is implemented just like any other
pointwise rule, but with one difference, that is the brace limiter around the
variable, which is initialized. 

\par Why are they called Build rules and not initialization ?  For starting
iteration, Loci creates hierarchy, so in order to start the calculation we have
to build a foundation for the hierarchy, which is at $n=0$. 

\par {\bf Example :}
\begin{verbatim}
1  class Initialization: public pointwise_rule {
2    store<double>          finit ;
3    const_store<double>    noisyData;
4  public:
5    laplaceSmoothing_initial() {
6      name_store("smooth{n=0}",finit) ;
7      name_store("noisyData",noisyData) ;
8
9      input( "noisyData" );
10     output("smooth{n=0}") ;
11    }
12    void calculate(Entity e) {
13      finit[e] = noisyData[e] ;
14    }
15    virtual void compute(const sequence &seq) {
16      do_loop(seq,this ) ;
17    }
18 } ;
\end{verbatim}
\par Notice the line 6 and 10. The $\{ n= 0 \}$ around the fact smooth signifies that
it is a build rule. If there were no braces, then it would have been simply a stationary
rule.

\par {\bf At present Loci has a limitation that iterations can start only from $n=0$,
which means that we cannot restart the calculations from some intermediate stage.} 
%
\section {Advance Rule}
\par Advance rules specify how values changes with the iteration. These rules are 
implemented using pointwise rules. A prototype of advance rule is as follow
\begin{verbatim}
1 class laplaceSmoothing_advance : public pointwise_rule{
2  store<double>         fnew;
3  const_store<double>   fold;
4  const_MapVec<4>       neighConnect;
5 public:
6  laplaceSmoothing_advance() {
7    name_store( "data{n}",  fold );
8    name_store( "data{n+1}",fnew );
9    name_store( "neighConnect", neighConnect);
10   input ("neighConnect->data{n}");
11   output("data{n+1}");
12  }
13  void calculate( Entity e){
14    int    n1, n2, n3, n4;
15    n1      = neighConnect[e][0];
16    n2      = neighConnect[e][1];
17    n3      = neighConnect[e][2];
18    n4      = neighConnect[e][3];
19    fnew[e] = (fold[n1] + fold[n2] + fold[n3] + fold[n4])/4.0;
20  }
21  virtual void compute( const sequence &seq)
22  {
23    do_loop( seq, this );
24  }
25 };
26 register_rule<laplaceSmoothing_advance>
27             register_laplaceSmoothing_advance;
\end{verbatim}

In the above example values are advanced to $(n+1)$th iteration using the
value at $n$th level. In line 7, we specify the facts at $n$th level which
is input to the input and line 8, we specify the output fact for the $n+1)$
th level, which is generated by the application of the rule. 
\section {Collapse Rule}
\par  Collapse rule specify {\em how } iterations are terminated and what values are generated
as a result of the iteration process. The criterion of terminating an iterative process is
always user defined. In Loci, a collapse rule takes input from the current iteration
and generates values at the lower iteration level. The word "collapse" signifies that we 
are breaking the hierarchy of the variable, which was raised during the iterations from 
the stationary time at $n=0$. As a result of this rule, hierachial facts are brought 
down to stationary level, which were promoted by the build rules. After the execution of 
collapse rule, Loci, may delete the iteration fact, therefore, we may not be able to use 
them later.
\par The word {\em how } in collapse rule definition, can sometimes, creates confusion and 
ambiguity in the definition. Let
us clarify it more. This word {\em how }prescribe {\bf only} the  condition to stop the iterations, 
which is just a boolean operation. This rule doesn't requires {\em how} this condition was set to 
1, in fact, the rule, which calculates the condition is a different rule, and is 
implemented as a reductionrule (because all the entities should have the same value of
the condition). 
%
\begin{verbatim}
1 class ResultSmoothing: public pointwise_rule {
2  store<double>        result ;
3  const_store<double>  smooth ;
4  public:
5   ResultSmoothing() {
6    name_store("smooth{n}",smooth) ;
7    name_store("smooth",result) ;
8    input("smooth{n}") ;
9    output("smooth") ;
10   conditional("IterationOver{n}") ;
11  }
12  void calculate(Entity e) {
13    result[e] = smooth[e] ;
14  }
15  virtual void compute(const sequence &seq) {
16    do_loop(seq,this ) ;
17  }
18 } ;
\end{verbatim}
\par Notice the line 10, this signifies that this rule will be invoked only when the
{\em IterationOver${n}$} is true, i.e. when the iterations are over. Notice in the 
line 6, we have specified {\em smooth \{ n \}} which is the variable associated with the
iteration rules, and in line 7 {\em smooth} specifies the stationary level fact. The
collapse rule makes a copy of {\em smooth\{n\}} to {\em smooth} when the iteration are 
terminated. 

\par The collapse rule will be invoked when the conditional fact value is TRUE. It is
a singleton value, and it is up-to user to specify the criteria using {\bf Singleton Rule}.
Let us write a sample program to do this operation.
%
\par {\bf Example :} In solving $Ax=b$ using iterative schemes, we choose to terminate
iteration under the following conditions
\begin{itemize}
\item  The maximum residual $\parallel Ax-b \parallel$ $< \epsilon$.
\item  Iteration counter has reached to the Maximum number of iteration specified by the
       user.
\end{itemize}
\begin{verbatim}
1  class IterationOver_Condition : public singleton_rule {
3    const_param<int>      n, max_iteration ;
4    param<bool>           iterationOver ;
5    const_store<double>   tolerance;
6    const_store<double>   residue;
7  public:
8    IterationOver_Condition() {
9      name_store("$n",n) ;
10      name_store("maxIterations",max_iteration);
11      name_store("tolerance", tolerance);
12      name_store("maxResidue",residue);
13      name_store("IterationOver",iterationOver);
14
15     input("$n,maxIterations");
16       input("tolerance");
17       input("residue");
18
19      output("iterationOverr") ;
20    }
21    virtual void compute(const sequence &seq) {
22      *iterationOver = (*n >= *max_iteration) || ( residue < tolerance);
23    }
24 } ;
\end{verbatim}

\par One thing, which we should emphasize here that facts {\em smooth} and {\em smooth$\{n\}$}
are two totally different facts, one is stationary and the other is bind to iterative
rules. Sometime, at the end of collapse rule, you may want to copy the {\em smooth$\{n\}$} to
a result fact, which could be {\em smooth} also. Let us see how we do it.

\section { Priority Rules }
\par Let us take an example of solving 1D Laplace equation using finite difference
scheme with fixed boundary conditions (i.e. Dirichlet conditions). 
A sequential code in C++ may be as follows

\begin{verbatim}
  u[0]      = 1.0;  // Applying boundary condition at the left.
  u[nmax-1] = 1.0;  // Applying boundary condition at the right

  for( i = 1; i < nmax-1; i++) 
       u[i] = 0.0;  // Initializing the values at internal nodes,

  for( i = 0; i < nmax; i++) 
       uold[i] = u[i];
   
  // Start the iterations.
  for( iter = 0; iter < maxIter; iter++) {
      // Calculate the new values ...
       for( i = 1; i < nmax-1; i++) 
            unew[i] = 0.50*( uold[i-1] + uold[i+1]);

      // Update the values ...
      for( i = 1; i < nmax-1; i++) 
          uold[i] = unew[i];
  }
\end{verbatim}

\par In this program, the boundary conditions remains the same throughout
the computations, so it was unnecessary to copy them for the next iteration.
But the same method will not work in Loci, where each variable name $u$ is
associated with entity $i$ at iteration $n$. In Loci, things are fine only 
at the first iteration, but successive iteration will not have values at 
the boundary, so how do we proceed ?
%
\par Here comes the role of {\em priority rules}. A {\em priority rule} is
very similar to {\em point-wise rules} described earlier. By specifying
priority rules, we can ensure that boundary values are promoted at the next
iteration level, by just copying the values from previous level or updating
them differently than internal nodes. In other words, we are forcing that 
certain entities should have higher priority than the other ones. 
\par {\em In Loci, we can increase the priority of certain rules, but cannot
decrease it}

The following is the program which implements both priority rules and 
normal pointwise rules.

\begin{verbatim}
class laplaceSmoothing_default : public pointwise_rule 
{
  store<double>      fnew;
  const_store<double>      fold;
public:
  laplaceSmoothing_default() {
    name_store( "smooth{n}",  fold );
    name_store( "smooth{n+1}",fnew );
    
    input ("smooth{n}");
    output("smooth{n+1}");
  }
  void calculate( Entity e){
    fnew[e] = fold[e];
  }

  virtual void compute( const sequence &seq)
  {
    do_loop( seq, this );
  }
};

register_rule<laplaceSmoothing_default> register_laplaceSmoothing_default;

class laplaceSmoothing_advance : public pointwise_rule{
  store<double>         fnew;
  const_store<double>   fold;
  const_Map             leftmap, rightmap;
public:
  laplaceSmoothing_advance() {
    name_store( "smooth{n}",  fold );
    name_store( "interior::smooth{n+1}",fnew );
    name_store( "LeftConnect",  leftmap);
    name_store( "RightConnect", rightmap);
    
    input ("LeftConnect->smooth{n}");
    input ("RightConnect->smooth{n}");

    output("interior::smooth{n+1}");
  }
  void calculate( Entity e){
    int    n1, n2, n3, n4;
    n1      = leftmap[e];
    n2      = rightmap[e];
    fnew[e] = (fold[n1] + fold[n2])/2.0;
  }

  virtual void compute( const sequence &seq)
  {
    do_loop( seq, this );
  }
};

register_rule<laplaceSmoothing_advance>
              register_laplaceSmoothing_advance;

\end{verbatim}

\par Notice that priority rules are specified by using interior::smooth{n+1}.
Here the name interior is immaterial, we can choose any other name. Without
the use of :: this program will not even compile, because Loci will notice
two rules for computing the new facts, which is prohibited by Loci. 

{\par \em  Whether the priority rules are executed before the normal rules
or after, it is up-to Loci to decide about it, which is immaterial from users' point of view }

\section {Can we avoid priority rules ? }

In the above example, a priority rule was a must in order to produce the
desired result, but could we avoid priority rule altogether, and if yes,
how ? If we closely examine the above code, a priority rule was required
because we never distinguished between the internal nodes and boundary nodes,
and because of this, we had to provide a priority rules, which applies 
a default behavior to all the entities, before (or after) it apply rules
for the internal nodes. But we could have classified nodes into two 
groups and could have created two pointwise rules one for the internal
nodes, and other for the boundary nodes. So we can get the same results
with on the expense of creating two group. 
\par Which method is better, it depends on the user's liking.

\section{ Stationary Rule}
Consider the ideal gas law for the mixture of thermally perfect gases
\begin{equation}
p = \rho R T
\end{equation}
This equation establishes a relationship of pressure to density, temperature
and mixture gas constant. This relationship is universal, it applied to
any place where these properties are simultaneously available at any 
instant to time. This law can be represented by the rule signature
\begin{equation}
p = \leftarrow \rho,  R , T
\end{equation}
\par This rule signature signifies that a value for the pressure can be
generated from values for density, temperature and density.  A Rule which is 
independent of iteration identifier is called a {\bf Stationary Rule}
\par In reality the scope of stationary rules is much wider than what is
described earlier.  All rules which are not in any iteration could also be
termed as {\em Stationary Rules }
%
\section{ Time Promotion Rule}
As explained in earlier section, Stationary rules are general rules in
space and time, which means that they can be used either as stand-alone
rules or in iteration hierarchy. If these rules are applied with {}, which
signifies the iterations, Loci will automatically promote these rules
in time. A time promoted facts don't require initialization of the values.
%
\section { Understanding operators $\rightarrow$ and $\leftarrow$ }
Consider Loci as a fine grain programming model, where each rule require need to 
know only those facts from the database, which it really requires. When we specify
rules, which iterative over an entities, which have map, then there must be 
some mechanism to access the data associated with the mapped entities. In Loci,
this is done using $\rightarrow$ operator, which is sometimes termed as {\em fetch
operator}

\par When we write "triConnect $\rightarrow$ pos", it means that triConnect is a map (
any kind of map i.e. Map, MapVec or Multimap) and its mapped values will be used
to fetch the values of {\em pos} from the fact database.

As oppose to fetch operator $\rightarrow$, which is just used as to fetch the 
stored data from the database, $\leftarrow$ is an assignment operators, which
means a new facts is created and need to be store in the fact database. 
\par For example, we would like to calculate centroid of a cell which 
requires a map, which stores the neighboring nodes, which in turn store the
coordinates of the nodes. We can specify the rule as
\begin{center}
Centroid $\leftarrow$ CellConnect $\rightarrow$ pos
\end{center}

There is no upper limit of using these indirection operators to assign or 
fetch the data.

The $\rightarrow$ is a powerful operator which allows recursion. A recursive
statements call themselves. For example a simple execution of statement 
$ a \rightarrow b$ is deferred till $b$ can be generated by some rule, the
generation of $b$ is another rule, which is again deferred till we find
one rule which can generate the $b$. Once one $b$ is available, {\bf backtracking}
takes over and we get the final result. In fact, many algorithms are inherently
recursive (even if the user is unaware of them), Loci will generate a correct
sequence if there is one rule, which can generate the base value.

\section { Queries }
Loci is {\bf Goal Oriented } language. At the end of any simulation, you want to
generate or transform some facts. In Loci, {\bf Query} is used to specify the outcome
facts. Loci, will generate a correct code with the supplied facts and rules, if
the query facts can be generated. A used may provide many rules and facts, but if 
they are not needed to generate the query facts, then Loci
will prune those rules and a scheduler will not execute those rule. This is unlike
C/C++ programming, where every procedure is called, whether its execution is 
required or not, to produce the final results.
\section { Memory Management in Loci }
If you notice carefully, when we define rules, which create new facts, we
have not not allocated any memory for the new facts. It is not required
from the user to allocate memory for the new facts, Loci will internally
allocate necessary memory for you. 

\section { Writing programs in Loci } 
Developing new application using Loci is very easy.  The most important 
step in developing Loci program is the identification of entities and
relationship. There are many ways to do this. Once you have decided
judiciously, an application uses the following five steps.
\begin{enumerate}
\item { Specify query variables }
\item { Create facts and store them in facts database }
\item { Create Rules and store them in rules database }
\item { Create and execute schedule}
\item { Query the fact database }
\end{enumerate}

\section { Before you write your own program}
\begin{itemize}
\item Decide about the entities and their relationship with other entities.
\item Specify the queries before you run the program.
\item All entities within the program has unique identifier. It is user's 
responsibility to provide unique numbering.  
\item All reduction operations are broken into two rules, ie. {\em unit\_rule} and 
{\em apply\_rule}
\item Stationary rules don't require brace limiters.
\item Time promotion in Loci is automatic.
\item All iteration require three rules i.e. build, advance and collapse.
\item Currently, Loci doesn't support Polymorphism for the rules, so you 
have to create rules for each datatype.
\end{itemize}

\par When you write rules, consider the followings 
\begin{itemize}
\item  What is the context of the rule, or in other words, over which
entities, the rule will be created. 
\item  For the entities in the rule context, what is the mapped entities.
\item  How the value are accessed. If there are maps involved, the 
       values are accessed using indirection
\end{itemize}

\section { Loci Programs }
\subsection { Program 1 }
\begin{verbatim}
int main(int argc, char *argv[]) 
{
  
  string query = "Data";

  param<int>  n;
  fact_db        facts ;                      // Facts Database
  
  *n = 10;
  facts.create_fact( "N", n);

  entitySet  aset = interval(1,*n);

  entitySet  start;
  store<int> startEntity;

  start += 1;
  startEntity.allocate( start );
  facts.create_fact( "StartEntity", startEntity);

  Map       predecessor;
  predecessor.allocate(aset);

  for( int i = 1; i <= *n; i++)
      predecessor[i] = i-1;

  facts.create_fact( "Predecessor", predecessor);

  rule_db    rdb ;                  // Rule Database
  
  rdb.add_rules(global_rule_list) ;

  executeP schedule = create_execution_schedule(rdb,facts,query ) ;

  if(schedule != 0) schedule->execute(facts) ;

  schedule->Print(cout) ;
  
  Loci::variableSet query_vars( Loci::expression::create(query));

  return 0 ;
}

//********************************************************************

class Recursive_Base : public pointwise_rule {
  const_Map    predecessor;
  store<int>   data;
public:
  Recursive_Base() {
    constraint( "StartEntity" );        // Apply this rule only one entity.
    name_store( "Predecessor", predecessor );
    name_store( "Data",  data );

    input( "Predecessor" );

    output( "Data" );
  }

  void calculate( Entity e) {
      data[e]   = e;
  }

  virtual void compute(const sequence &seq){
    do_loop( seq, this );
  }
};

register_rule<Recursive_Base>  register_Recursive_Base;

//********************************************************************

class ForwardSequence : public pointwise_rule {
  store<int>    data;
  const_Map     predecessor;
public:
  ForwardSequence() {
    name_store( "Predecessor", predecessor );
    name_store( "Data",        data );

    input( "Predecessor->Data" );        // Recursive definition.
    output( "Data" );
  }

  void calculate( Entity e) {
     data[e]   = e;
  }

  virtual void compute(const sequence &seq){
    do_loop( seq, this );
  }
};

register_rule<ForwardSequence>  register_Forward_Sequence;

//********************************************************************
\end{verbatim}

\subsection { Program 2 }

\begin{verbatim}
int main(int argc, char *argv[]) 
{
  
  string query = "Data";

  param<int>  n;
  fact_db        facts ;                      // Facts Database
  
  *n = 10;
  facts.create_fact( "N", n);

  entitySet  aset = interval(1,*n);

  entitySet  start;
  store<int> startEntity;

  start += *n;
  startEntity.allocate( start );
  facts.create_fact( "StartEntity", startEntity);

  Map    successor;
  successor.allocate(aset);

  for( int i = 1; i < *n; i++)
      successor[i] = i+1;

  facts.create_fact( "Successor", successor);

  rule_db    rdb ;                  // Rule Database
  
  rdb.add_rules(global_rule_list) ;

  executeP schedule = create_execution_schedule(rdb,facts,query ) ;

  if(schedule != 0) schedule->execute(facts) ;

  schedule->Print(cout) ;
  
  Loci::variableSet query_vars( Loci::expression::create(query));
  
  return 0 ;
}

//*********************************************************************

class Recursive_Base : public pointwise_rule {
  const_Map    successor;
  store<int>   data;
public:
  Recursive_Base() {
    constraint( "StartEntity" );
    name_store( "Successor", successor );
    name_store( "Data",      data );

    input( "Successor" );

    output( "Data" );
  }

  void calculate( Entity e) {
      data[e]   = e;
  }

  virtual void compute(const sequence &seq){
    do_loop( seq, this );
  }
};

register_rule<Recursive_Base>  register_Recursive_Base;

//********************************************************************

class BackwardSequence : public pointwise_rule {
  store<int>    data;
  const_Map     successor;
public:
  BackwardSequence() {
    name_store( "Successor", successor );
    name_store( "Data",        data );

    input( "Successor->Data" );
    output( "Data" );
  }

  void calculate( Entity e) {
     data[e]   = e;
  }

  virtual void compute(const sequence &seq){
    do_loop( seq, this );
  }
};

register_rule<BackwardSequence>  register_BackwardSequence;

\end{verbatim}


%\chapter { Sample Programs }
In this chapter, we will extremely simple programs to demonstrate the usage
of Loci. Each example, contains a salient feature of Loci, which are required
to write real applications. This will help in better understanding the underline
concepts described in earlier chapters.

\section { Calculation of $\pi$ using numerical integration }
%
The exact value of ${\pi}$ is given by the integration
%
\begin{equation}
\pi = {1 \over 4 r^2}\int_{0}^{r} {(r^2-x^2)^{1\over2} dx}
\end{equation}
%
\par \par To do this integration numerically, we divide the interval from $0$ to $R$ into 
some number $n$  of subintervals, then we use Euler method to calculate the summation
of rectangle formed under the curve. In this case coordinates $x$ is independent
term and the height of the curve is given by $y(=h) = (R^2-x^2)^{1/2}$. 
Larger number of intervals, will produce thin rectangles, which will minimize the
errors associated with overshoot or undershoot area formed because of the rectangular
approximation of the curve and therefore, we will get more accurate results of
the value of $\pi$. Because of finite precision of the machines, we may not be
compute the result to the theoretical value.  This is not, in fact, a very good way 
of compute $\pi$, and more accurate approximation such as Simpson's or Trapezoidal
rules should be applied, but our emphasize is to explain Loci features.
\begin{equation}
\pi \approx {1 \over 4 r^2}\sum_{i=0}^{n}(r^2-x_i^2)dx
\end{equation}

where  $n$ is the user defined number of intervals in the domain and 
$dx = {1\over n}$. 

In the above formulation $x$ coordinate is an independent variable and the
height of the rectangle depends on the value of $x$, we can say, that height
is an attribute to the variable $x$, therefore $x$ is an entity,and the
each subinterval give rise to one entity. Also since the width of subinterval
is constant across all the entities, it is best represented as parameter value.
In Loci, Entities are identified as integer values, so that we might assign value
$(0,1,2, \dots n)$ to them.  The spatial location of an entity is given
by $x_i = x_0 + i*dx$ and the attribute value $\phi_i$ is $(r*r-x_i*x_i)^{1/2}dx$
The area of the curve is given by the summation of each rectangle. Which we 
will perform using the reduction operation.


To facilitate describing discretization process, the
$N$ subintervals, or cells, are enumerated as by $(0, \dots N)$. 
%
\begin{figure}[h]
\special{psfile=pi.eps vscale=50 hscale=50 voffset=-250 hoffset=100}
\vspace*{3.50in}\caption {Calculation of $\pi$ using numerical integration}
\end{figure}
%
\subsection { What you will learn from this example }
\par This simple examples, will reveal many important aspect of Loci system. 
At the end of the example, you should be able to learn 
\begin{enumerate}
\item  How to identify entities and associate maps on each entity.
\item  How to apply reduction rule.
\end{enumerate}
%
\subsection { Solving using C/C++ }
\begin{verbatim}
1
2  #include <stdio.h>
3  #include <iostream>
4  #include <iomanip>
5
6  int main(int argc, char *argv[])
7  {
8
9     int     i, N = 100000;
10    double  x, dx, sum = 0, radius=1.0;
11    double  *pi_funct;
12
13    pi_funct = new double[N];
14
15    dx =  1.0/N;
16    for( i = 0; i < N; i++) {
17      x           = 0.5*dx + i*dx;
18      pi_funct[i] = sqrt(radius*radius-x*x);
19    }
20
21    cout << setprecision(10);
22    sum = 0.0;
23    for( i = 0; i < N; i++)
24         sum = sum + pi_funct[i];
25    pi = sum/(4.0*radius*radius);
26    cout << " The value of PI is " << sum << endl;
27
28    delete [] pi_funct;
29
30    return 0 ;
31  }
\end{verbatim}

\subsection { Solving using Loci }
\subsubsection { Creating fact database }
%
When we look at the problem, we observe that we need to calculate the
area of each rectangle lying between $x_{i-1/2}$ and $x_{i+1/2}$. We
denote $x_i$ as the center of the cell and calculate the value of the
function at that location. That mean, we would like to store the
value of the function at the center of the cell. Therefore, we should
create an entity for the cell center. So, in this example {\bf Cell-Center
is an entity}. and the collection of $n$ cells will constitute an
entitySet.

\begin{verbatim}
1  #include <Loci.h>
2  #include <Tools/stream.h>
3
4  int main(int argc, char *argv[])
5  {
6
7  //-------------------------------------------------------------------
8     string query = "PI";
9
10 //-------------------------------------------------------------------
11    fact_db        facts ;               // Facts Database
12
13    int            i, N = 1000;
14    double         x, dx, radius = 1.0;
15    store<double>  pi_funct; // At at entity we are going to store the value
16
17    entitySet  cells=interval(0,N);
18    pi_funct.allocate(cells);
19
20    dx =  1.0/N;
21    for( i = 0; i < N; i++) {
22      x           = 0.5*dx + i*dx;
23      pi_funct[i] = sqrt(radius*radius-x*x)*dx;
24    }
25
26    facts.create_fact( "pi_funct", pi_funct);
27
28    //-------------------------------------------------------------------
29    rule_db    rdb ;                  // Rule Database
30
31    rdb.add_rules(global_rule_list) ;
32
33    //-------------------------------------------------------------------
34
35    executeP schedule = create_execution_schedule(rdb,facts,query ) ;
36
37    if(schedule != 0) schedule->execute(facts) ;
38
39    //-------------------------------------------------------------------
40
41    Loci::variableSet query_vars( Loci::expression::create(query));
42
43    //-------------------------------------------------------------------
44
45    Loci::variableSet::const_iterator vi ;
46
47    cout << " The value oi PI is " << endl;
48
49    for(vi=query_vars.begin();vi!=query_vars.end();++vi) {
50      Loci::storeRepP sr = facts.get_variable(*vi) ;
51      if(sr == 0) {
52        cout << "variable " << *vi << " does not exist in fact database."
53             << endl ;
54      } else {
55        sr->Print(cout) ;
56      }
57    }
58
59    return 0 ;
60  }
\end{verbatim}

%
\subsubsection { Creating rules database }
\begin{verbatim}
1  #include <Loci.h>
2  class pi_sum_unit : public unit_rule {
3    param<double>  sum;
4  public:
5    pi_sum_unit() {
6      name_store( "PI", sum );
7      constraint( "UNIVERSE");
8
9      output("PI");
10    }
11    virtual void compute(const sequence &seq){
12      *sum = 0;
13    }
14  };
15
16  register_rule<pi_sum_unit>  register_pi_sum_unit ;
17
18  //*********************************************************************
19
20  class pi_sum_apply : public apply_rule<param<double>,
21                                         Loci::Summation<double> >
22  {
23    param<double> sum ;
24    const_store<double> pi_funct ;
25  public:
26    pi_sum_apply() {
27      name_store("PI", sum);
28      name_store("pi_funct",pi_funct) ;
29
30      input("pi_funct");
31      input("PI") ;
32      output("PI");
33    }
34
35    void calculate( Entity e) {
36      join(sum[e],pi_funct[e]) ;
37    }
38
39    virtual void compute( const sequence &seq) {
40      do_loop( seq, this );
41      sum *= 0.25;
41    }
42  };
43
44  register_rule<pi_sum_apply> register_pi_sum_apply;
45
46  //*********************************************************************
\end{verbatim}

\section { 1D Finite Volume Calculation}
Consider a 1D linear diffusion given by
\begin{equation}
u_t = \nu u_{xx}, \hspace{5mm}
\end{equation}
with initial and boundary conditions given by
\begin{eqnarray}
u(x,0) = f(x)    \\
u_x(0,t) = g(t)  \\
u_x(1,t) = h(t)
\end{eqnarray}

\begin{figure}[h]
\special{psfile=heat.eps vscale=50 hscale=50 voffset=-100 hoffset=1}
\vspace*{1.60in}\caption {Problem Specification of 1D Diffusion }
\end{figure}

\subsection{ What you will learn from this example }
\begin{itemize}
\item  How they entities and relationship are identified.
\item  This program make use of stationary and time promotion rules,
       which are automatically generated by Loci, but you should
       know about them.
\end{itemize}
      

\subsection{A Finite Volume Solution}

The first step in numerically approximating the function $u(x,t)$ is
the discretization of the spatial domain (in this case the interval
$[0,1]$).  For this example, the finite volume discretization method
is chosen.\footnote{Other discretization schemes follow similar lines.
  See Appendix \ref{app:numerical} for alternative discretization
  examples.} Using this discretization approach, the interval $[0,1]$
is divided into $N$ sub-intervals, as illustrated in figure
\ref{fig3:oned}.  To facilitate describing the discretization process,
the $N$ sub-intervals, or cells, are labeled by $c = N+1, \cdots, 2N$,
while the interfaces at the boundaries of sub-intervals are labeled $i
= 0, \cdots, N$.  Note that the typical labeling used for theoretical
purposes would include half step labels for the interfaces, while a
typical unstructured application code might label both cells and
interfaces starting from zero and use context to distinguish between
the two cases.  However, for the purposes of automating reasoning
about these entities of computations it is assumed that these labels
are integers and that independent computational sites (in this case,
cells and interfaces) are labeled distinctly.  The proposed labeling
satisfies both of these constraints.

%\begin{figure}[htbp]
% \centerline{
%  \epsfxsize=5.50in
%  \epsfbox{one-d.eps}}
% \caption{A Discretization of the Interval $[0,1]$}
% \label{fig3:oned}
%\end{figure}

As illustrated in figure \ref{fig3:oned}, the discretization yields
$N+1$ interfaces which have the positions given by
\begin{equation}
x = \lbrace (i,x_i) | i \in [0, \cdots, N], x_i = i/N \rbrace.
\label{eq3:interfacex}
\end{equation}
Notice that the variable $x$ in this equation is described by a set of
ordered pairs where the first entry is the entity identifier, whereas
the second entry is the value bound to that entity.  This is a more
general abstraction of the array.  For example {\tt x[i]} is
represented abstractly as $\lbrace x_i | (i,x_i) \in x \rbrace$.

In addition, this discretization yields $N$ intervals, or cells, which
are represented by the mappings between cells and interfaces by way of
the following relationships
\begin{equation}
\begin{array}{rcl}
il & = & \lbrace (c,l) | c \in [N+1, \cdots, 2N], l = c-N-1 \rbrace,\\
ir & = & \lbrace (c,r) | c \in [N+1, \cdots, 2N], r = c-N \rbrace.\\
\end{array}
\label{eq3:cellmaps}
\end{equation}
The mappings $il$ and $ir$ provide mappings from every cell to their
left and right interfaces.  The domain of $ir$ and $il$ is $[N+1,
\cdots, 2N]$, or the cells in the discretization, while the ranges are
$\mathrm{ran}(ir) = [0, \cdots, N-1]$ and $\mathrm{ran}(il) = [1, \cdots, N]$.  This
mapping is used to conveniently describe subscripts, {\it i.e.}
$x_{c-N} = ir \rightarrow x$, where the composition operator,
$\rightarrow$, defines the application of the mapping, as in
\begin{equation}
il\rightarrow x = \lbrace (c,x_l) | (c,l) \in il, (l,x_l) \in x \rbrace.
\end{equation}
Using this notation, it is possible to conveniently describe cell
based calculations.  For example, a generic description of each cell
center is given by
\begin{equation}
\label{eq3:cellcenter}
x = (ir \rightarrow x + il \rightarrow x)/2.
\end{equation}
Note that the definition of $x$ provided by equation
(\ref{eq3:cellcenter}) is only applicable to cells since only cells are in
the domain of maps $ir$ and $il$; however, this does not prevent the
definition of $x$ for other entities (for example, interfaces) via
other rules.

The mappings $il$ and $ir$ are used to describe the first
step of the finite volume discretization, where integration of
equation (\ref{eq3:diffuse}) over each cell produces the equation
\begin{equation}
\int_{t_n}^{t_{n+1}} \int_{il \rightarrow x}^{ir \rightarrow x} u_t dx
dt = \int_{t_n}^{t_{n+1}} \nu(ir \rightarrow u_x - il \rightarrow
u_x) dt.
\label{eq3:diffinteg}
\end{equation}

Equation (\ref{eq3:diffinteg}) is an exact equation, which can be
integrated numerically to obtain a numerical solution algorithm.
For example, a first order end-point rule is applied to the time
integrations while a second order mid-point rule is applied to the space
integrations to obtain a finite-volume numerical method, expressed as
\begin{equation}
u^{n+1} = u^n + \nu \Delta t
\left[ \frac{ir \rightarrow u^n_x - il \rightarrow u^n_x}
          {ir\rightarrow x - il\rightarrow x}\right]
\label{eq3:timeadvance}
\end{equation}

Equation (\ref{eq3:timeadvance}) describes the numerical method for
advancing the time step, but it is not complete.  The gradient term,
$u^n_x$, located at the interfaces has not been defined as a numerical
approximation.  The most straightforward approximation for $u_x$ is a
central difference formula using the values at the cell centers at
either side of the interface.  In order to perform this calculation it
will be convenient to have a mapping from interfaces to cells similar to
the development of $il$ and $ir$.  These mappings are defined by
the relations
\begin{equation}
\begin{array}{rcl}
cl & = & \lbrace (i,l) | i \in [1, \cdots, N], l = i+N \rbrace,\\
cr & = & \lbrace (i,r) | i \in [0, \cdots, N-1], r = i+N+1 \rbrace.\\
\end{array}
\label{eq3:facemaps}
\end{equation}

Using the definitions of $cl$ and $cr$ of (\ref{eq3:facemaps}), a
numerical approximation to the gradient can be given as
\begin{equation}
u_x^n = \frac{cr\rightarrow u^n - cl\rightarrow u^n}
           {cr\rightarrow x - cl\rightarrow x}.
\label{eq3:ux}
\end{equation}
Notice that this equation uses the x-coordinate at the cell centers
that is computed by equation (\ref{eq3:cellcenter}).  In addition,
since this rule uses both maps $cr$ and $cl$, it only defines $u_x$ on
the intersection of the domains of $cr$ and $cl$, given by $[1, \cdots,
N-1]$.  By this reasoning, equation (\ref{eq3:ux}) only provides
gradients at the internal faces of the domain.  The gradient at the
boundary faces is provided by the boundary conditions given in
equations (\ref{eq3:diffuseb0}) and (\ref{eq3:diffuseb1}).  The
question is, how do these boundary conditions specify $u_x$ at the
boundaries without specifying $u_x$ everywhere in the domain?
Obviously additional information must be provided that constrains
the application of boundary condition gradients only to the boundary
interfaces.  A solution to this problem can be found with the
observation that the boundary interfaces have the distinction that
either $cl$ is defined or $cr$ is defined, but not both.  Using this
fact, the rules for calculating the boundary gradients can be given by
\begin{eqnarray}
u_x^n & = g(n \Delta t), \mbox{constraint}\lbrace \neg \mathrm{dom}(cl) \wedge
\mathrm{dom}(cr) \rbrace, \label{eq3:brule0}\\
u_x^n & = h(n \Delta t), \mbox{constraint}\lbrace \mathrm{dom}(cl)
\wedge \neg \mathrm{dom}(cr) \rbrace.\label{eq3:brule1}
\end{eqnarray}
Here the constraint term added to the rule indicates a constraint on
the application of the rule.  In this case it constrains the
application of the boundary conditions to the appropriate boundary faces.

At this point, the computation of $u^{n+1}$ from $u^n$ is completely
specified.  However, before any such iteration can begin, an initial
value, or $u^{n=0}$, must be given.  To be consistent with the finite
volume formulation, the derivation of the initial conditions begins
with the integral form of equation (\ref{eq3:diffuseinitial}), given by
\begin{equation}
\int^{ir\rightarrow x}_{il\rightarrow x} u^{n=0} dx =
\int^{ir\rightarrow x}_{il\rightarrow x} f(x) dx.
\end{equation}
Using a midpoint rule to numerically integrate this equation one
obtains the rule
\begin{equation}
u^{n=0} = f(x), \mbox{constraint}\lbrace (il,ir)\rightarrow x\rbrace.
\label{eq3:ic}
\end{equation}
For this rule, the constraint is used to indicate that although the
coordinates of the interfaces cancel in the derivation, their
existence is predicated by the integration.  In other words, the
derivation assumed a cell perspective that includes left and right
interface positions.

\subsection{On Problem Specification}

For an analytic solution method, equations (\ref{eq3:diffuse}) through
(\ref{eq3:diffuseb1}) are sufficient to define the problem at hand.
For numerical solution methods, additional definitions are required,
due to the fact that these are inexact methods.  For example, there
are often tradeoffs between discretization and accuracy that require
additional specification.  In addition, since discretization for
complex geometries (grid generation) is not a completely automatic
process, the discretization becomes part of the problem definition for
numerical solution methods.  For the example diffusion problem already
introduced, the definition of the numerical problem consists of
spatially independent information such as the diffusion constant
$\nu$, the initial condition function $f(x)$, the numerical time step
$\Delta t$, and a representation of the discretization of space.  The
discretization of space is given by a set of positions,
(\ref{eq3:interfacex}), and the collection of mappings given in
(\ref{eq3:cellmaps}) and (\ref{eq3:facemaps}).  Table
\ref{table3:facts} summarizes these formal definitions for the example
diffusion problem.


\begin{table}[htbp]
\caption{ A Summary of Definitions for the Example Diffusion
  Problem}
\label{table3:facts}
\begin{center}
  \begin{tabular}{|l|l|}
    \hline
    fact      & meaning \\
    \hline
    $\nu$     & given diffusion constant  \\
    $f(x)$     & given initial condition  \\
    $g(t)$     & given left bc \\
    $h(t)$     & given right bc  \\
    $\Delta t$& given time-step  \\
    $x$       & $\lbrace (i,x_i) | i \in [0, \cdots, N], x_i = i/N    \rbrace$\\
    $il$      & $\lbrace (c,l)   | c \in [N+1, \cdots, 2N], l = c-N-1 \rbrace$\\
    $ir$      & $\lbrace (c,r)   | c \in [N+1, \cdots, 2N], r = c-N   \rbrace$\\
    $cl$      & $\lbrace (i,l)   | i \in [1, \cdots, N], l = i+N      \rbrace$\\
    $cr$      & $\lbrace (i,r)   | i \in [0, \cdots, N-1], r = i+N+1  \rbrace$\\
    \hline
  \end{tabular}
\end{center}
\end{table}

\subsection{On Specification of Process}

Given the definition of the problem, the process of solving the
problem is dictated by a prescribed set of transformations.  For
example, consider equation (\ref{eq3:cellcenter}) as an example of a
transformation that transforms $x$ located at $il$ and $ir$ into a
cell $x$.  To simplify discussions of the structure of the
calculations, the transformation rules are represented by a rule
signature that is denoted by a list of targets of the transformation
delineated from the sources of the transformation by the left arrow
symbol, '$\leftarrow$'.  Thus the cell center position calculation is
represented by the rule signature $x \leftarrow (ir,il)\rightarrow x$.
This rule signature represents the augmentation of the set of ordered
pairs defined in equation (\ref{eq3:interfacex}) with the additional
set given as
\begin{equation}
x\leftarrow\lbrace (c, x_c) |  x_c = (x_l + x_r)/2,
                               (l,x_l) \in x, (r,x_r) \in x, 
                               (c,l) \in il, (c,r) \in ir \rbrace.
\end{equation}
For the moment, the augmentation of $x$ with this set can be
considered as a set union operation, with the caveat that it will
become more complex once issues of specification consistency are
considered.  Given this notation, the specification of the finite
volume scheme derived in this section can be summarized by six rules
given in table \ref{table3:rules}.

\begin{table}[htbp]
\caption{ A Summary of Rules Describing the Solution of the Example
    Diffusion Problem.}
\label{table3:rules}
\begin{center}
  \begin{tabular}{|l|l|l|}
    \hline
    Rule  & Rule Signature & Equation\\
    \hline
    Rule 1 & $x \leftarrow (ir,il)\rightarrow x $ &
    (\ref{eq3:cellcenter})\\
    Rule 2 & $u^{n+1} \leftarrow u^n,(ir,il)\rightarrow(u_x^n,x)$ &
    (\ref{eq3:timeadvance})\\
    Rule 3 & $u_x^n \leftarrow (cr,cl)\rightarrow(u^n,x)$ &
    (\ref{eq3:ux})\\
    Rule 4 & $u_x^n \leftarrow  g, n, \Delta t, \mbox{constraint}\lbrace 
    \neg \mathrm{dom}(cl) \wedge \mathrm{dom}(cr) \rbrace$&
    (\ref{eq3:brule0})\\
    Rule 5 & $u_x^n \leftarrow  h, n, \Delta t, \mbox{constraint}\lbrace 
    \mathrm{dom}(cl) \wedge \neg \mathrm{dom}(cr) \rbrace$ &
    (\ref{eq3:brule1})\\
    Rule 6 & $u^{n=0} \leftarrow f,x,\mbox{constraint}\lbrace(il,ir)\rightarrow
    x\rbrace $ &
    (\ref{eq3:ic})\\
    \hline
  \end{tabular}
\end{center}
\end{table}

\subsection{Implementing the  Problem Specification}

How does one translate the definition of the problem given in table
\ref{table3:facts} and the specification of the solution method given
in table \ref{table3:rules} into an implementation that can solve for
$u^n, n=0,1, \cdots$?  One accomplishes this implementation by
starting from what is known and using the rules of table
\ref{table3:rules} to incrementally derive the specified goal.  As in
Prolog\cite{Clocksin.87}\cite{Sterling.86}, rule resolution, a
generalization of modus ponens, is used to produce these incremental
derivations.  In other words, rules are applied where their sources
are satisfied.  The set of entities that satisfy a rules sources is
the context of the rule.  For example, Rule 1 from table
\ref{table3:rules} can be resolved with the definitions of $il$, $ir$,
and $x$ given in table \ref{table3:facts} for the entities numbered
$N+1, \cdots, 2N$.  Similarly, once Rule 1 is resolved, Rule 6 can be
resolved using the values of $x$ generated by Rule 1.  Iteration is
recovered by way of induction.  For example, if a rule generates
$u^{n=0}$ while another rule generates $u^{n+1}$, then these two rules
can be used to iteratively generate $u^n$ for $n=0,1, \cdots$.  Thus
by resolving the rules that provide the interface and boundary
gradients $u_x^n$ a complete schedule as shown in table
\ref{table3:schedule} can be derived from the given specification.

\begin{table}[htbp]
\caption{ A Deduced Execution Schedule for the Example
  Diffusion Problem}
\begin{center}
  \begin{tabular}{|l|l|l|l|}
    \hline
    Rule Used & variable computed  & context          & comment \\
    \hline
    Rule 1 & compute $x_c$         & $c=N+1, \cdots, 2N$ & cell centers\\
    Rule 6 & compute $u^{n=0}_c$   & $c=N+1, \cdots, 2N$ & initial conditions \\
    Loop   &                       & define $n=0$     & for $n=0,\cdots$ \\
    Rule 4 & compute $(u_x)^n_i$   & $i=0$       & left boundary condition\\
    Rule 5 & compute $(u_x)^n_i$   & $i=N$       & right boundary condition\\
    Rule 3 & compute $(u_x)^n_i$   & $i=1..N-1$  & diffusion flux at time $n$\\
    Rule 2 & compute $u^{n+1}_i$   & $i=0..N$    & advance time-step\\
    End Loop &                     & loop to Rule 4 & replace $n=n+1$, repeat\\
    \hline
  \end{tabular}
\end{center}
\label{table3:schedule}
\end{table}

\subsection { Implementation }

\begin{verbatim}
1  #include <Loci.h>
2  #include <iostream>
3
4  using namespace std ;
5
6  int main()
7  {
8    const int N = 50 ; // Number of points in grid.
9
10    //-----------------------------------------------------------------
11    // Create a 1-d unstructured grid ; Node and Cells are identified
12    // separately.
13    //-----------------------------------------------------------------
14    entitySet nodes  = interval(0,N) ;
15    entitySet cells  = interval(N+1,2*N);
16
17    //-----------------------------------------------------------------
18    // Generate 1D grid positions at the nodes.
19    //-----------------------------------------------------------------
20    store<float> x;
21    x.allocate(nodes);
22
23    entitySet::const_iterator ei ; // Generic iterator
24
25    for(ei=nodes.begin();ei!=nodes.end();++ei)
26      x[*ei] = float(*ei)/float(N);
27    //-----------------------------------------------------------------
28    // Create mapping from interface to cells
29    // cl = cell left , cr = cell right
30    //-----------------------------------------------------------------
31
32    Map cl,cr ;
33    cl.allocate(nodes-interval(0,0)) ; // do not allocate for leftmost interface
34    cr.allocate(nodes-interval(N,N)) ; // do not allocate for rightmost interface
35
36    // Assign maps from nodes to cells
37    // cl = {(i,l) | i \in [1,N], l = i+N}
38    // cr = {(i,r) | i \in [0,N-1], r = i+N+1}
39    for(ei=cl.domain().begin();ei!=cl.domain().end();++ei)
40      cl[*ei] = *ei + N;
41    for(ei=cr.domain().begin();ei!=cr.domain().end();++ei)
42      cr[*ei] = *ei + N + 1;
43
44    //-----------------------------------------------------------------
45    // Create mapping from cells to interface
46    // il = interface left, ir = interface right
47    // il = {(c,l) | c \in cells, l = c-N-1},
48    // ir = {(c,r) | c \in cells, l = c-N}
49    //-----------------------------------------------------------------
50    Map il,ir ;
51    il.allocate(cells) ;
52    ir.allocate(cells) ;
53
54    for(ei=cells.begin();ei!=cells.end();++ei) {
55      il[*ei] = *ei - N - 1 ;
56      ir[*ei] = *ei - N ;
57    }
58
59    //-----------------------------------------------------------------
60    // Create Facts Database
61    //-----------------------------------------------------------------
62    fact_db facts ;
63
64    facts.create_fact("il",il) ;
65    facts.create_fact("ir",ir) ;
66    facts.create_fact("x", x) ;
67    facts.create_fact("cl",cl) ;
68    facts.create_fact("cr",cr);
69
70    // Diffusion constant
71    param<float> nu ;
72    *nu = 1.0 ;
73    facts.create_fact("nu",nu) ;
74
75    // Number of iterations to run simulation
76    param<int> max_iteration ;
77    *max_iteration = 100 ;
78    facts.create_fact("max_iteration",max_iteration) ;
79
80    // Minimum L1 norm for convergence test
81    param<double> error_tolerance;
82    *error_tolerance = 1.0E-03;
83    facts.create_fact( "error_tolerance", error_tolerance);
84
85    // Identify boundary conditions
86    constraint left_boundary ;
87    constraint right_boundary ;
88    *right_boundary = cl.domain() - cr.domain() ;
89    *left_boundary = cr.domain() - cl.domain() ;
90
91    facts.create_fact("left_boundary",left_boundary) ;
92    facts.create_fact("right_boundary",right_boundary) ;
93
94    //-----------------------------------------------------------------
95    // Create Rule database ...
96    //-----------------------------------------------------------------
97
98    rule_db rdb ;
99    rdb.add_rules(global_rule_list) ;
100
101    //-----------------------------------------------------------------
102    // Create and execute the schedule to  obtains the solution
103    //-----------------------------------------------------------------
104    executeP schedule = create_execution_schedule(rdb,facts,"solution") ;
105
106    //schedule->Print(cout) ;                   // Display schedule
107
108    schedule->execute(facts) ;           // Execute the schedule
109
110    //-----------------------------------------------------------------
111    // Final Step: Query the database for solution:
112    //-----------------------------------------------------------------
113
114    store<float> usol ;
115    usol = facts.get_variable("solution") ;
116
117    cout << "The solution is : " <<endl;
118    for(ei=cells.begin();ei!=cells.end();++ei)
119      cout << ""<< *ei<<" "<<usol[*ei]<<endl ;
120
121    //-----------------------------------------------------------------
122    // End of computations::
123    //-----------------------------------------------------------------
124
125    return(0);
126
127  }
128  //*******************************************************************
\end{verbatim}

\begin{verbatim}
1  class cell_center : public pointwise_rule {
2    const_store<float> x ;
3    const_Map il,ir ;
4    store<float> xc ;
5  public:
6    cell_center() {
7      name_store("x",x) ;
8      name_store("il",il) ;
9      name_store("ir",ir) ;
10      name_store("xc",xc) ;
11      input("(il,ir)->x") ;
12      output("xc") ;         // xc <- (il,ir)->x
13    }
14    void calculate(Entity e) {
15      xc[e] = 0.5*(x[il[e]]+x[ir[e]]);
16    }
17  
18    virtual void compute(const sequence &seq) {
19      do_loop(seq,this,&cell_center::calculate) ;
20    }
21  } ;
22  
23  register_rule<cell_center> register_cell_center ;
24  
25
//*******************************************************************
\end{verbatim}

\begin{verbatim}
1  //*******************************************************************
2  //Initializing values at time= 0 using u(x,0) = f(x). Since "xc" is
3  //used, it will calculate the values are cell-centers.
4  //*******************************************************************
5
6  float f(float x) {
7    return 0 ;
8  }
9
10  class initial_condition : public pointwise_rule {
11    const_store<float> xc ;
12    store<float> u ;
13  public:
14    initial_condition() {
15      name_store("xc",xc) ;
16      name_store("u{n=0}",u) ;
17      input("xc") ;
18      output("u{n=0}") ; // u{n=0}<-xc
19    }
20    void calculate(Entity e) {
21      u[e] = f(xc[e]) ;
22    }
23
24    virtual void compute(const sequence &seq) {
25      do_loop(seq,this,&initial_condition::calculate) ;
26    }
27  } ;
28
29  register_rule<initial_condition> register_initial_condition ;
\end{verbatim}

\begin{verbatim}
1  //*******************************************************************
2  // Compute boundary condition at leftmost interface. At left we are
3  // imposing Neumann Boundary Condition.
4  //*******************************************************************
5
6  class left_bc : public pointwise_rule {
7    store<float> ux ;
8  public:
9    left_bc() {
10      name_store("ux",ux) ;
11      constraint("left_boundary") ;
12      output("ux") ;
13    }
14    void calculate(Entity e) {
15      ux[e] = -1 ;
16    }
17
18    virtual void compute(const sequence &seq) {
19      do_loop(seq,this,&left_bc::calculate) ;
20    }
21  } ;
22
23  register_rule<left_bc> register_left_bc ;
24
25  //*******************************************************************
26  // Compute boundary condition at rightmost interface. At right we are
27  // imposing Neumann Boundary Condition.
28  //*******************************************************************
29
30  class right_bc : public pointwise_rule {
31    const_store<float> u,xc,x ;
32    const_Map cl ;
33    store<float> ux ;
34  public:
35    right_bc() {
36      name_store("u",u) ;
37      name_store("cl",cl) ;
38      name_store("x",x) ;
39      name_store("xc",xc) ;
40      name_store("ux",ux) ;
41      input("x,cl->(u,xc)") ;
42      output("ux") ;
43      constraint("right_boundary") ;
44    }
45    void calculate(Entity e) {
46      ux[e] = (u[cl[e]])/(xc[cl[e]]-x[e]) ;
47    }
48
49    virtual void compute(const sequence &seq) {
50      do_loop(seq,this,&right_bc::calculate) ;
51    }
52  } ;
53
54  register_rule<right_bc> register_right_bc ;
55
56  //*******************************************************************
\end{verbatim}

\begin{verbatim}
1     class interface_gradient : public pointwise_rule {
2    const_store<float> u ;
3    const_Map cl,cr ;
4    const_store<float> xc ;
5    store<float> ux ;
6  public:
7    interface_gradient() {
8      name_store("u",u) ;
9      name_store("cl",cl) ;
10      name_store("cr",cr) ;
11      name_store("xc",xc) ;
12      name_store("ux",ux) ;
13      input("(cl,cr)->(u,xc)") ;
14      output("ux") ; 
15    }
16    void calculate(Entity e) {
17      ux[e] = (u[cl[e]]-u[cr[e]])/(xc[cl[e]]-xc[cr[e]]) ;
18    }
19  
20    virtual void compute(const sequence &seq) {
21      do_loop(seq,this,&interface_gradient::calculate) ;
22    }
23  } ;
24  
25  register_rule<interface_gradient> register_interface_gradient ;
\end{verbatim}

\begin{verbatim}
1  //*******************************************************************
2  // Compute maximum stable timestep for simulation.  Use reduction rule
3  // and calculate timestep as a function of local conditions.  The global
4  // timestep is the minimum of these local timesteps
5  //*******************************************************************
6
7  class timestep_unit : public unit_rule {
8    param<float> dt ;
9  public:
10    timestep_unit() {
11      name_store("dt",dt) ;
12      output("dt") ;
13      constraint("UNIVERSE") ;  // This property applies to all entities
14    }
15    virtual void compute(const sequence &seq) {
16      *dt = 1e30 ;              // Largest allowable timestep
17    }
18  } ;
19
20  register_rule<timestep_unit> register_timestep_unit ;
21
22  //*******************************************************************
23
24  class timestep_apply : public apply_rule<param<float>,
25                         Loci::Minimum<float> > {
26    const_store<float> xc ;
27    const_Map cl,cr ;
28    const_param<float> nu ;
29    param<float> dt ;
30  public:
31    timestep_apply() {
32      name_store("xc",xc) ;
33      name_store("cl",cl) ;
34      name_store("cr",cr) ;
35      name_store("nu",nu) ;
36      name_store("dt",dt) ;
37      input("dt,(cl,cr)->xc,nu") ;
38      output("dt") ;
39    }
40    void calculate(Entity e) {
41      float dx = abs(xc[cr[e]]-xc[cl[e]]) ;
42      // Compute timestep as 1/2 of maximum stable timestep
43      float local_dt = dx*dx*nu[e]/4. ;
44
45      join(dt[e],local_dt) ; // Set dt = min(dt,local_dt)
46    }
47
48    virtual void compute(const sequence &seq) {
49      do_loop(seq,this,&timestep_apply::calculate) ;
50    }
51
52  } ;
53
54  register_rule<timestep_apply> register_timestep_apply ;
55
56  //*******************************************************************
\end{verbatim}

\begin{verbatim}

1  //-------------------------------------------------------------------
2  // Objective :  Compute u{n+1} based on explicit euler time integration method
3  //-------------------------------------------------------------------
4
5  class advance_time : public pointwise_rule {
6    const_store<float> x,ux,un ;
7    const_Map il,ir ;
8    const_param<float> dt,nu ;
9    store<float> unp1 ;
10  public:
11    advance_time() {
12      name_store("x{n}",x) ;
13      name_store("ux{n}",ux) ;
14      name_store("u{n}",un) ;
15      name_store("il{n}",il) ;
16      name_store("ir{n}",ir) ;
17      name_store("nu{n}",nu) ;
18      name_store("dt{n}",dt) ;
19      name_store("u{n+1}",unp1) ;
20      input("u{n},nu{n},dt{n},(ir{n},il{n})->(ux{n},x{n})") ;
21      output("u{n+1}") ;
22
23      constraint("(il{n},ir{n})->x{n}") ; // Assert update for all valid cells
24    }
25
26    void calculate(Entity e) {
27      unp1[e] = un[e] + nu[e]*dt[e]*(ux[ir[e]]-ux[il[e]])/(x[ir[e]]-x[il[e]]) ;
28    }
29
30    virtual void compute(const sequence &seq) {
31      do_loop(seq,this,&advance_time::calculate) ;
32    }
33  } ;
34
35  register_rule<advance_time> register_advance_time ;
36
37  //*******************************************************************
\end{verbatim}
\begin{verbatim}

1  //*******************************************************************
2  // When simulation is finished, copy current iteration results to
3  // solution
4  //*******************************************************************
5
6  class collapse_time : public pointwise_rule {
7    const_store<float> u ;
8    store<float> solution ;
9  public:
10    collapse_time() {
11      name_store("u{n}",u) ;
12      name_store("solution",solution) ;
13      input("u{n}") ;
14      output("solution") ;
15      conditional("simulation_finished{n}") ;
16    }
17    void calculate(Entity e) {
18      solution[e] = u[e] ;
19    }
20
21    virtual void compute(const sequence &seq) {
22      do_loop(seq,this,&collapse_time::calculate) ;
23    }
24  } ;
25
26  register_rule<collapse_time> register_collapse_time ;
27
28  //*******************************************************************
29  // Condition that determines when iteration is complete
30  //*******************************************************************
31
32  class collapse_condition : public singleton_rule {
33    const_param<int> n, max_iteration ;
34    param<bool> simulation_finished ;
35  public:
36    collapse_condition() {
37      name_store("$n",n) ;
38      name_store("max_iteration",max_iteration) ;
39      name_store("simulation_finished",simulation_finished) ;
40      input("$n,max_iteration") ;
41      output("simulation_finished") ;
42    }
43    virtual void compute(const sequence &seq) {
44      *simulation_finished = (*n >= *max_iteration) ;
45    }
46  } ;
47
48  register_rule<collapse_condition> register_collapse_condition ;
49
50  //*******************************************************************
~
\end{verbatim}

\section {Unstructured Grid Cell Area and Centroid Calculation}
For the solution of PDE in complex geometries using unstructured grid
provides greater flexibility. In this example, we will read triangular
grid ( generated from other Mesh Generation Program ) and calculate
area and centroid of each cell.

\begin{figure}[h]
\special{psfile=sq.eps vscale=50 hscale=50 voffset=-325 hoffset=50}
\vspace{3.75in}\caption {Triangular Grid in a square domain}
\end{figure}

\subsection { What will you learn from this example}
\begin{itemize}
\item  Creating two different entitySet.
\item  Using constant map vector.
\end{itemize}


\subsection { Creating Facts Database }
\begin{verbatim} 
1  #include <Loci.h>
2  #include <Tools/stream.h>
3  #include "2dvector.h"
4
5  typedef Loci::vector2d<double> VECTOR;
6
7  bool read_grid(fact_db &facts, char *filename);
8
9   //*********************************************************************
10  // Objective :  Find the Centroid and Area of each 2D Cell ( Triangles )
11  //*********************************************************************
12  int main(int argc, char *argv[])
13  {
14
15    //-------------------------------------------------------------------
16    // First Step: Creating queries ..
17    // In this examples, we will just create rules for centroid and area
18    // only and query the results. Specify what you would like to query
19    // from the fact data base..
20    //-------------------------------------------------------------------
21
22    string query = "Centroid, Area";
23
24    //-------------------------------------------------------------------
25    // Second Step : Create fact data base
26    //
27    // In this example, we import the grids using some other software, which
28    // generates the triangular grids.
29    //
30    //-------------------------------------------------------------------
31    fact_db        facts ;               // Facts Database
32
33    if(!read_grid(facts,"grid.dat")) {
34      cerr << "unable to read 'grid' file" << endl ;
35      exit(-1) ;
36    }
37
38 
39  //-------------------------------------------------------------------
40  // Third Step : Create Rules and register in Rule Database..
41  //-------------------------------------------------------------------
42  rule_db    rdb ;                  // Rule Database
43
44  rdb.add_rules(global_rule_list) ;
45
46  //-------------------------------------------------------------------
47  // Fourth Step: Create and execute the scheduler. A scheduler uses
48  // fact and rule database to generate query database.
49  //-------------------------------------------------------------------
50
51  executeP schedule = create_execution_schedule(rdb,facts,query ) ;
52  
53  if(schedule != 0) {
54    cout << "schedule = " << endl ;
55    schedule->Print(cout) ;
56    schedule->execute(facts) ;
57  }
58
59  //-------------------------------------------------------------------
60  // Fifth Step : Execution is over,  Query the database. 
61  //-------------------------------------------------------------------
62  
63  Loci::variableSet query_vars( Loci::expression::create(query));
64  
65  Loci::variableSet::const_iterator vi ;
66
67  for(vi=query_vars.begin();vi!=query_vars.end();++vi) {
68    Loci::storeRepP sr = facts.get_variable(*vi) ;
69    if(sr == 0) {
70      cout << "variable " << *vi << " does not exist in fact database."
71           << endl ;
72    } else {
73      sr->Print(cout) ;
74    }
75  }
76  //-------------------------------------------------------------------
77  // Everything was successful. Execution is over 
78  //-------------------------------------------------------------------
79  
80  return 0 ;
81 }
\end{verbatim}

\begin{verbatim}
1    bool read_grid(fact_db &facts, char *filename)
2    {
3
4    //-------------------------------------------------------------------
5    // Open the File in which Grid information is stored ...
6    //-------------------------------------------------------------------
7
8    ifstream in(filename, ios::in) ;
9    if(in.fail()) {
10      cerr << "can't open file '" << filename << "'." << endl ;
11      return false ;
12    }
13
14    //-------------------------------------------------------------------
15    // How many nodes and starting nodes (C Style=0 or Fortran=1) ...
16    //-------------------------------------------------------------------
17
18    int nnodes, nstart ;
19    in >> nnodes >> nstart ;
20
21    entitySet nodes = interval(nstart,nstart+nnodes-1) ;
22    entitySet::const_iterator ei ;
23    store<VECTOR> pos ;
24    pos.allocate(nodes) ;
25    for(ei=nodes.begin();ei!=nodes.end();++ei)
26      in >> pos[*ei] ;
27
28    facts.create_fact("pos",pos) ;
29
30    //-------------------------------------------------------------------
31    // Read triangular cell connectivity ...
32    //-------------------------------------------------------------------
33
34    int  ncells;
35
36    in >> ncells;
37
38    entitySet cells = interval( nstart+nnodes, nstart+nnodes+ncells-1);
39
40    MapVec<3>    triConnect;
41
42    triConnect.allocate( cells );
43
44    int  n1, n2, n3;
45    for( entitySet::const_iterator ei=cells.begin(); ei != cells.end(); ei++
){
46      in >> n1 >> n2 >> n3;
47      triConnect[*ei][0]   = n1;
48      triConnect[*ei][1]   = n2;
49      triConnect[*ei][2]   = n3;
50    }
51
52    facts.create_fact("triConnect", triConnect);
53
54    //-------------------------------------------------------------------
55    // That is all we need for this example.
56    //-------------------------------------------------------------------
57    cout << "Information: Reading grid file over " << endl;
58
59    return true;
60  }
61
62  //*********************************************************************
\end{verbatim}

\subsection { Creating Rule Database }
There are two rules which we need to define, one for Centroid and one
for Area of each triangle. These are defined as follows.

\begin{verbatim}
1  class triCentroid : public pointwise_rule{
2    store<VECTOR>         centroid;
3    const_MapVec<3>       triConnect;
4    const_store<VECTOR>   pos;
5  public:
6    triCentroid ()
7    {
8      name_store( "Centroid", centroid );
9      name_store( "triConnect", triConnect);
10      name_store( "pos", pos);
11
12      input ("triConnect->pos");
13      output("Centroid");
14    }
15
16    //---------------------------------------------------------------------
17    // The following is the rule to calculate the centroid of triangle..
18    //---------------------------------------------------------------------
19    void calculate( Entity e)
20    {
21      int    n1, n2, n3;
22      n1            = triConnect[e][0];
23      n2            = triConnect[e][1];
24      n3            = triConnect[e][2];
25      centroid[e].x = (pos[n1].x + pos[n2].x + pos[n3].x)/3.0;
26      centroid[e].y = (pos[n1].y + pos[n2].y + pos[n3].y)/3.0;
27    }
28
29    virtual void compute( const sequence &seq)
30    {
31      do_loop( seq, this );
32    }
33  };
34
35  register_rule<triCentroid> register_triCentroid ;
36
37  //********************************************************************
38  // The following modules calculates the area of a triangle
39  // using Green's theorem
40  //********************************************************************
41
42  class triArea : public pointwise_rule{
43    store<double>         area;
44    const_MapVec<3>       triConnect;
45    const_store<VECTOR>   pos;
46  public:
47    triArea()
48    {
49      name_store( "Area", area);
50      name_store( "triConnect", triConnect);
51      name_store( "pos", pos);
52
53      input ("triConnect->pos");
54      output("Area");
55    }
56
57
58    void calculate( Entity e)
59    {
60      int    nodeList[3], inode, nodeID, nnodes;
61      double  x[5], y[5], sum;
62
63      nnodes        = 3;
64      nodeList[0]   = triConnect[e][0];
65      nodeList[1]   = triConnect[e][1];
66      nodeList[2]   = triConnect[e][2];
67
68      for( inode = 0; inode < nnodes; inode++){
69        nodeID     = nodeList[inode];
70        x[inode+1] = pos[nodeID].x;
71        y[inode+1] = pos[nodeID].y;
72      }
73
74      // Complete the node cycle.
75      x[nnodes+1] = x[1];
76      y[nnodes+1] = y[1];
77      x[0]        = x[nnodes];
78      y[0]        = y[nnodes];
79
80      sum = 0.0;
81      for( inode = 1; inode <= nnodes; inode++)
82        sum +=  x[inode]*(y[inode+1]-y[inode-1]);
83      area[e] = 0.5*sum;
84    }
85
86    virtual void compute( const sequence &seq)
87    {
88      do_loop( seq, this );
89    }
90  };
91  register_rule<triArea> register_triArea;
92
\end{verbatim}

\section { Laplace Smoothing }
Laplacian Smoothing is a common technique for smoothing the noisy signal.
Its formula is given as 
\begin{equation}
\phi^s = { 1 \over n} { \sum_{i=1}^{i=n} } \alpha_i \phi_i^n
\end{equation}

Where $\alpha_i$ is the weight associated to $\phi_i$. For simple case it can be taken
as unity.

\begin{figure}[h]
\special{psfile=lap.eps vscale=40 hscale=40 voffset=-215 hoffset=50}
\vspace{2.96in}\caption { Laplacian Smoothing over internal nodes }
\end{figure}

For simplicity we take square domain  consisting of the points $(x_i, y_i)$ given by
\begin{eqnarray}
x_i =  i / (n+1), i = 0, \dots n+1  \\
y_i =  j / (n+1), j = 0, \dots n+1  \\
\end{eqnarray}

where there are $n+2$ points along each edge of the mesh. We can approximate the 
operator at each of these points with the formula
\begin{equation}
{u_{i-1,j} + u_{i,j+1} + u_{i,j-1} + u_{i+1,j} - 4 u_{i,j}} = 0
\end{equation}

Since the formula involves $u$ at five points, we must find some wait to solve
for $u$ everywhere,. One approach is to rewrite above equation as 

\begin{equation}
u_{i,j} = {1 \over 4}(u_{i-1,j} + u_{i,j+1} + u_{i,j-1} + u_{i+1,j})
\end{equation}

iterate by choosing values for all mesh points $u_{i,j}$ and then replace them by
using

\begin{equation}
u_{i,j}^{k+1} = {1 \over 4}(u_{i-1,j}^k + u_{i,j+1}^k + u_{i,j-1}^k + u_{i+1,j}^k)
\end{equation}

\subsection { What you will learn from this example }
At the end of this examples, you should be able to know the following.
\begin{enumerate}
\item How to read facts from the file.
\item How to use mapVec.
\item How to write operation that works on only subsets.
\item How to use priority rules.
\end{enumerate}

\subsection { How to identify internal nodes }

In order to clarify the use of entitySets, we would like to apply smoothing
only on the internal nodes. There are many ways we can make use of entitySet
in this example, we will explain only two.  

\begin{enumerate}
\item  The first ways is very simple, we can create one entitySet of 
{\em mapVec$<4>$} and store the four neighbors for all internal nodes. This
map could be supplied to the rule.
\item The second way is more elegant, we will create four maps 
\begin{itemize}
\item {\bf eastmap} :  All Entities which have east as their neighbor. Only the
rightmost boundary entities will not be included in this map.
\item {\bf westmap} :  All Entities which have west as their neighbor. Only the
leftmost boundary entities will not be included in this map.
\item {\bf northmap}:  All Entities which have north as their neighbor. Only the
uppermost boundary entities will not be included in this map.
\item {\bf southmap}:  All Entities which have north as their neighbor. Only the
lowermost boundary entities will not be included in this map.
\end{itemize}
\par Using the second approach, an internal entity is identified as an entity
which have all four neighbors. 
\end{enumerate}
               
\subsection { Using C/C++ }
\begin{verbatim}
1  int main()
2  {
3
4    double     noisyData[100], smooth[100], noise[100];
5    int        xnodes, ynodes;
6    int        i,j, n1, n2, n3, n4, iter;
7    ofstream   outfile("noise.dat", ios::out);
8
9    outfile << xnodes << " " << ynodes << endl;
10    for( i = 0; i < xnodes*ynodes; i++) {
11      noisyData[i]  = drand48();
12      outfile << noisyData[i] << endl;
13    }
14
15    for( i = 0; i < xnodes*ynodes; i++) {
16      noise[i]  = noisyData[i];
17      smooth[i] = noisyData[i];
18    }
19
20    int  offset, numIters = 100;
21    for( iter = 0; iter < numIters; iter++) {
22      cout << iter << endl;
23      for( j = 1; j < ynodes-1; j++) {
24        for( i = 1; i < xnodes-1; i++) {
25          offset         = j*xnodes + i;
26          n1             = offset-1;      // Left
27          n2             = offset+1;      // Right
28          n3             = offset+xnodes; // Top
29          n4             = offset-xnodes; // Bottom
30          smooth[offset] = 0.25*(noise[n1]+noise[n2]+noise[n3]+noise[n4]);
31        }
32      }
33      for( i = 0; i < xnodes*ynodes; i++)
34        noise[i] = smooth[i];
35    }
36
37    cout.setf( ios::fixed);
38    cout << " Original Noisy Data " << endl;
39    for( j = 0; j < ynodes; j++) {
40      for( i = 0; i < xnodes; i++)
41        cout << noisyData[j*xnodes+i] << " ";
42      cout << endl;
43
44    }
45
46    cout << " Smooth Data " << endl;
47    for( j = 0; j < ynodes; j++) {
48        for( i = 0; i < xnodes; i++)
49          cout << smooth[j*xnodes+i] << " ";
50        cout << endl;
51    }
52
53  }
54
55  //*********************************************************************
\end{verbatim}


\subsection{ Using Loci : Method 1}
\subsubsection { Creating Facts }
\begin{verbatim}
1  int main(int argc, char *argv[])
2  {
3
4    //-------------------------------------------------------------------
5    // Step 1: Specify queries ...
6    //-------------------------------------------------------------------
7
8    string query = "smooth" ;
9
10    //-------------------------------------------------------------------
11    // Step 2: Create fact database
12    //-------------------------------------------------------------------
13
14    fact_db     facts ;                 // Facts Database
15    param<int>     max_iterations;
16    *max_iterations = 100;
17
18    facts.create_fact( "max_iterations", max_iterations);
19
20    //-------------------------------------------------------------------
21    // Create entities. In this example, all the internal nodes are our
22    // entities where calculation will be performed i.e. where Laplacian
23    // operator will be applied. For this example, we are creating data
24    // statically, but can be modified to read through a file.
25    //-------------------------------------------------------------------
26
27    int  xnodes, ynodes;
28
29    ifstream infile("noise.dat", ios::in);
30    if( infile.fail()) {
31        cout << "Error: Cann't open file " << endl;
32        exit(0);
33    }
34
35    infile >> xnodes >> ynodes;
36
37    int    i,j;
38    entitySet internalNodes;
39    for( j = 1; j < ynodes-1; j++) {
40      for( i = 1; i < xnodes-1; i++) {
41        internalNodes += j*xnodes + i;
42      }
43    }
44
45    store<double> noisyData;
46    entitySet     nodes = interval(0,xnodes*ynodes-1);
47    noisyData.allocate(nodes);
48
49    for( i = 0; i < xnodes*ynodes; i++)
50        infile >> noisyData[i];
51    infile.close();
52
53    facts.create_fact( "noisyData",noisyData);
54
55    //-------------------------------------------------------------------
56    // In 2D structured grid, all internal nodes have four neighbors.
57    //-------------------------------------------------------------------
58
59    MapVec<4>      neighConnect;
60    neighConnect.allocate(internalNodes);
61
62    size_t offset;
63    for( j = 1; j < ynodes-1; j++) {
64      for( i = 1; i < xnodes-1; i++) {
65        offset                   = j*xnodes + i;
66        neighConnect[offset][0]  = offset-1;      // Left
67        neighConnect[offset][1]  = offset+1;      // Right
68        neighConnect[offset][2]  = offset+xnodes; // Top
69        neighConnect[offset][3]  = offset-xnodes; // Bottom
70      }
71    }
72    facts.create_fact( "neighConnect", neighConnect);
73
74
75
76    //-------------------------------------------------------------------
77    // Step 3: Create Rules and register then into Rule database
78    //-------------------------------------------------------------------
79    rule_db    rdb ;                  // Rule Database
80
81    rdb.add_rules(global_rule_list) ;
82
83    //-------------------------------------------------------------------
84    // Step 4: Create and Execute the Schedule
85    //-------------------------------------------------------------------
86
87    facts.write(cout) ;
88
89    executeP schedule = create_execution_schedule(rdb,facts,query ) ;
90
91    if(schedule != 0) {
92      cout << "schedule = " << endl ;
93      schedule->Print(cout) ;
94      schedule->execute(facts) ;
95    }
96
97    //-------------------------------------------------------------------
98    // Step 5: Execution Over, Query the database
99    //-------------------------------------------------------------------
100
101    Loci::variableSet query_vars( Loci::expression::create(query));
102
103    Loci::variableSet::const_iterator vi ;
104
105    for(vi=query_vars.begin();vi!=query_vars.end();++vi) {
106      Loci::storeRepP sr = facts.get_variable(*vi) ;
107      if(sr == 0) {
108         cout << "variable " << *vi << " does not exist in fact database."
109      << endl ;
110      } else {
111      sr->Print(cout) ;
112      }
113    }
114
115    //-------------------------------------------------------------------
116    // Execution over successfully
117    //-------------------------------------------------------------------
118
119    return 0 ;
120  }
121
122  //*********************************************************************
\end{verbatim}

\subsubsection { Creating Rules }
\begin{verbatim}
1  class laplaceConvergence_unit : public unit_rule {
2    param<double>    maxError;
3  public:
4    laplaceConvergence_unit() {
5      name_store( "maxError", maxError );
6      constraint( "UNIVERSE");
7
8      output("maxError");
9    }
10    virtual void compute(const sequence &seq){
11      *maxError = 0.0;
12    }
13
14  };
15  register_rule<laplaceConvergence_unit>
16                register_laplaceConvergence_unit;
17
18  //*********************************************************************
19
20  class laplaceConvergence_apply : public apply_rule<param<double>,
21                                                 Loci::Maximum<double> >
22  {
23    param<double>       maxError;
24    const_store<double> fnew, fold ;
25  public:
26    laplaceConvergence_apply() {
27      name_store("maxError", maxError);
28      name_store("smooth{n}",  fold);
29      name_store("smooth{n+1}",fnew);
30
31      input( "smooth{n}");
32      input( "smooth{n+1}");
33      output("maxError");
34    }
35    void calculate( Entity e) {
36      join(maxError[e], fabs(fold[e]-fnew[e])) ;
37    }
38    virtual void compute( const sequence &seq) {
39      do_loop( seq, this );
40    }
41  };
42
43  register_rule<laplaceConvergence_apply>
44    register_laplaceConvergence_apply;
45
46  //*********************************************************************
47
48  class laplaceSmoothing_initial : public pointwise_rule {
49    store<double>          finit ;
50    const_store<double>    noisyData;
51  public:
52    laplaceSmoothing_initial() {
53      name_store("smooth{n=0}",finit) ;
54      name_store("noisyData",noisyData) ;
55
56      input( "noisyData" );
57      output("smooth{n=0}") ;
58    }
59    void calculate(Entity e) {
60      finit[e] = noisyData[e] ;
61    }
62    virtual void compute(const sequence &seq) {
63      do_loop(seq,this ) ;
64    }
65  } ;
66
67  register_rule<laplaceSmoothing_initial>
68               register_laplaceSmoothing_initial ;
69
70  //*********************************************************************
71
72  class laplaceSmoothing_default : public pointwise_rule
73  {
74    store<double>      fnew;
75    const_store<double>      fold;
76  public:
77    laplaceSmoothing_default() {
78      name_store( "smooth{n}",  fold );
79      name_store( "smooth{n+1}",fnew );
80
81      input ("smooth{n}");
82      output("smooth{n+1}");
83    }
84    void calculate( Entity e){
85      fnew[e] = fold[e];
86    }
87
88    virtual void compute( const sequence &seq)
89    {
90      do_loop( seq, this );
91    }
92  };
93
94  register_rule<laplaceSmoothing_default>
95               register_laplaceSmoothing_default;
96
97  //*********************************************************************
98
99  class laplaceSmoothing_advance : public pointwise_rule{
100    store<double>         fnew;
101    const_store<double>   fold;
102    const_MapVec<4>       neighConnect;
103  public:
104    laplaceSmoothing_advance() {
105      name_store( "smooth{n}",  fold );
106      name_store( "interior::smooth{n+1}",fnew );
107      name_store( "neighConnect", neighConnect);
108
109      input ("neighConnect->smooth{n}");
110      output("interior::smooth{n+1}");
111    }
112    void calculate( Entity e){
113      int    n1, n2, n3, n4;
114      n1      = neighConnect[e][0];
115      n2      = neighConnect[e][1];
116      n3      = neighConnect[e][2];
117      n4      = neighConnect[e][3];
118      fnew[e] = (fold[n1] + fold[n2] + fold[n3] + fold[n4])/4.0;
119    }
120
121    virtual void compute( const sequence &seq)
122    {
123      do_loop( seq, this );
124    }
125  };
126
127  register_rule<laplaceSmoothing_advance>
128                register_laplaceSmoothing_advance;
129
130  //*********************************************************************
131
132  class laplaceSmoothingOver_condition : public singleton_rule {
133    const_param<int> n, max_iteration ;
134    param<bool>      smoothingOver ;
135  public:
136    laplaceSmoothingOver_condition() {
137      name_store("$n",n) ;
138      name_store("max_iterations",max_iteration);
139      name_store("SmoothingOver",smoothingOver);
140
141      input("$n,max_iterations") ;
142      output("SmoothingOver") ;
143    }
144    virtual void compute(const sequence &seq) {
145      *smoothingOver = (*n >= *max_iteration);
146    }
147  } ;
148
149  register_rule<laplaceSmoothingOver_condition>
150                register_laplaceSmoothingOver_condition;
151
152  //*********************************************************************
153
154  class collapse_smoothing : public pointwise_rule {
155    store<double>        result ;
156    const_store<double>  smooth ;
157
158  public:
159    collapse_smoothing() {
160      name_store("smooth{n}",smooth) ;
161      name_store("smooth",result) ;
162
163      input("smooth{n}") ;
164      output("smooth") ;
165      conditional("SmoothingOver{n}") ;
166    }
167    void calculate(Entity e) {
168      result[e] = smooth[e] ;
169    }
170    virtual void compute(const sequence &seq) {
171      do_loop(seq,this ) ;
172    }
173  } ;
174
175  register_rule<collapse_smoothing>
176                register_collapse_smoothing ;
177
178  //*********************************************************************
\end{verbatim}

\subsection{ Using Loci : Method 2}

\subsubsection { Creating fact database }
\begin{verbatim}
     1
     2  int main(int argc, char *argv[])
     3  {
     4
     5    //-------------------------------------------------------------------
     6    // Step 1: Specify queries ...
     7    //-------------------------------------------------------------------
     8
     9    string query = "smooth" ;
    10
    11    //-------------------------------------------------------------------
    12    // Step 2: Create fact database
    13    //-------------------------------------------------------------------
    14
    15    fact_db     facts ;               // Facts Database
    16    param<int>     max_iterations;
    17    *max_iterations = 100;
    18
    19    facts.create_fact( "max_iterations", max_iterations);
    20
    21    //-------------------------------------------------------------------
    22    // Create entities. In this example, all the internal nodes are our
    23    // entities where calculation will be performed i.e. where Laplacian
    24    // operator will be applied. For this example, we are creating data
    25    // statically, but can be modified to read through a file.
    26    //-------------------------------------------------------------------
    27
    28    int  xnodes, ynodes;
    29
    30    ifstream infile("noise.dat", ios::in);
    31    if( infile.fail()) {
    32      cout << "Error: Cann't open file " << endl;
    33      exit(0);
    34    }
    35
    36    infile >> xnodes >> ynodes;
    37
    38    store<double> noisyData;
    39    entitySet     nodes = interval(0,xnodes*ynodes-1);
    40    noisyData.allocate(nodes);
    41
    42    for( i = 0; i < xnodes*ynodes; i++)
    43      infile >> noisyData[i];
    44    infile.close();
    45
    46    facts.create_fact( "noisyData",noisyData);
    47
    48    //-------------------------------------------------------------------
    49    // In 2D structured grid, all internal nodes have four neighbors.
    50    //-------------------------------------------------------------------
    51
    52    size_t offset;
    53    // Create East map
    54    Map      eastmap;
    55
    56    for( i = 1; i < xnodes-2; i++) {
    57      for( j = 1; j < ynodes-1; j++) {
    58        offset          = j*xnodes + i;
    59        eastmap[offset] = offset+1;     // rightConnect
    60      }
    61    }
    62    facts.create_fact( "EastConnect", eastmap);
    63
    64    // Create West map
    65    Map      westmap;
    66    for( i = 2; i < xnodes-1; i++) {
    67      for( j = 1; j < ynodes-1; j++) {
    68        offset          = j*xnodes + i;
    69        westmap[offset] = offset-1;     // rightConnect
    70      }
    71    }
    72    facts.create_fact( "WestConnect", westmap);
    73
    74    // Create North map
    75    Map      northmap;
    76    for( j = 1; j < ynodes-2; j++) {
    77      for( i = 1; i < xnodes-1; i++) {
    78        offset          = j*xnodes + i;
    79        northmap[offset] = offset+xnodes;
    80      }
    81    }
    82    facts.create_fact( "NorthConnect", northmap);
    83
    84    // Create South map
    85    Map      southmap;
    86    for( j = 2; j < ynodes-1; j++) {
    87      for( i = 2; i < xnodes-1; i++) {
    88        offset          = j*xnodes + i;
    89        southmap[offset] = offset-xnodes;     // rightConnect
    90      }
    91    }
    92    facts.create_fact( "SouthConnect", southmap);
    93
    94    //-------------------------------------------------------------------
    95    // Step 3: Create Rules and register then into Rule database
    96    //-------------------------------------------------------------------
    97    rule_db    rdb ;                  // Rule Database
    98
    99    rdb.add_rules(global_rule_list) ;
   100
   101    //-------------------------------------------------------------------
   102    // Step 4: Create Scheduler
   103    //-------------------------------------------------------------------
   104
   105    facts.write(cout) ;
   106
   107    executeP schedule = create_execution_schedule(rdb,facts,query ) ;
   108
   109    if(schedule != 0) {
   110      cout << "schedule = " << endl ;
   111      schedule->Print(cout) ;
   112      schedule->execute(facts) ;
   113    }
   114
   115    //-------------------------------------------------------------------
   116    // Step 5: Execute the scheduler
   117    //-------------------------------------------------------------------
   118
   119    Loci::variableSet query_vars( Loci::expression::create(query));
   120
   121    //-------------------------------------------------------------------
   122    // Step 6: Query the database ...
   123    //-------------------------------------------------------------------
   124
   125    Loci::variableSet::const_iterator vi ;
   126
   127    for(vi=query_vars.begin();vi!=query_vars.end();++vi) {
   128      Loci::storeRepP sr = facts.get_variable(*vi) ;
   129      if(sr == 0) {
   130        cout << "variable " << *vi << " does not exist in fact database."
   131             << endl ;
   132      } else {
   133        sr->Print(cout) ;
   134      }
   135    }
   136
   137    //-------------------------------------------------------------------
   138    // Execution over successfully
   139    //-------------------------------------------------------------------
   140
   141    return 0 ;
   142  }
\end{verbatim}

\subsection { Creating rules }
\begin{verbatim}
     1  class laplaceConvergence_unit : public unit_rule {
     2    param<double>    maxError;
     3  public:
     4    laplaceConvergence_unit() {
     5      name_store( "maxError", maxError );
     6      constraint( "UNIVERSE");
     7
     8      output("maxError");
     9    }
    10    virtual void compute(const sequence &seq){
    11      *maxError = 0.0;
    12    }
    13
    14  };
    15  register_rule<laplaceConvergence_unit>
    16                register_laplaceConvergence_unit;
    17
    18  //*********************************************************************
    19
    20  class laplaceConvergence_apply : public apply_rule<param<double>,
    21                                                 Loci::Maximum<double> >
    22  {
    23    param<double>       maxError;
    24    const_store<double> fnew, fold ;
    25  public:
    26    laplaceConvergence_apply() {
    27      name_store("maxError", maxError);
    28      name_store("smooth{n}",  fold);
    29      name_store("smooth{n+1}",fnew);
    30
    31      input( "smooth{n}");
    32      input( "smooth{n+1}");
    33      output("maxError");
    34    }
    35    void calculate( Entity e) {
    36      join(maxError[e], fabs(fold[e]-fnew[e])) ;
    37    }
    38    virtual void compute( const sequence &seq) {
    39      do_loop( seq, this );
    40    }
    41  };
    42
    43  register_rule<laplaceConvergence_apply>
    44    register_laplaceConvergence_apply;
    45
    46  //*********************************************************************
    47
    48  class laplaceSmoothing_initial : public pointwise_rule {
    49    store<double>          finit ;
    50    const_store<double>    noisyData;
    51  public:
    52    laplaceSmoothing_initial() {
    53      name_store("smooth{n=0}",finit) ;
    54      name_store("noisyData",noisyData) ;
    55
    56      input( "noisyData" );
    57      output("smooth{n=0}") ;
    58    }
    59    void calculate(Entity e) {
    60      finit[e] = noisyData[e] ;
    61    }
    62    virtual void compute(const sequence &seq) {
    63      do_loop(seq,this ) ;
    64    }
    65  } ;
    66
    67  register_rule<laplaceSmoothing_initial>
    68               register_laplaceSmoothing_initial ;
    69
    70  //*********************************************************************
    71
    72  class laplaceSmoothing_default : public pointwise_rule
    73  {
    74    store<double>      fnew;
    75    const_store<double>      fold;
    76  public:
    77    laplaceSmoothing_default() {
    78      name_store( "smooth{n}",  fold );
    79      name_store( "smooth{n+1}",fnew );
    80
    81      input ("smooth{n}");
    82      output("smooth{n+1}");
    83    }
    84    void calculate( Entity e){
    85      fnew[e] = fold[e];
    86    }
    87
    88    virtual void compute( const sequence &seq)
    89    {
    90      do_loop( seq, this );
    91    }
    92  };
    93
    94  register_rule<laplaceSmoothing_default>
    95               register_laplaceSmoothing_default;
    96
    97  //*********************************************************************
    98
    99  class laplaceSmoothing_advance : public pointwise_rule{
   100    store<double>         fnew;
   101    const_store<double>   fold;
   102    const_Map             eastmap, westmap, northmap, southmap;
   103  public:
   104    laplaceSmoothing_advance() {
   105      name_store( "smooth{n}",  fold );
   106      name_store( "interior::smooth{n+1}",fnew );
   107      name_store( "EastConnect",  eastmap);
   108      name_store( "WestConnect",  westmap);
   109      name_store( "NorthConnect", northmap);
   110      name_store( "SouthConnect", southmap);
   111
   112      input ("eastmap->smooth{n}");
   113      input ("westmap->smooth{n}");
   114      input ("northmap->smooth{n}");
   115      input ("southmap->smooth{n}");
   116
   117      output("interior::smooth{n+1}");
   118    }
   119    void calculate( Entity e){
   120      int    n1, n2, n3, n4;
   121      n1      = eastmap[e];
   122      n2      = westmap[e];
   123      n3      = northmap[e];
   124      n4      = southmap[e];
   125      fnew[e] = (fold[n1] + fold[n2] + fold[n3] + fold[n4])/4.0;
   126    }
   127
   128    virtual void compute( const sequence &seq)
   129    {
   130      do_loop( seq, this );
   131    }
   132  };
   133
   134  register_rule<laplaceSmoothing_advance>
   135                register_laplaceSmoothing_advance;
   136
   137  //*********************************************************************
   138
   139  class laplaceSmoothingOver_condition : public singleton_rule {
   140    const_param<int> n, max_iteration ;
   141    param<bool>      smoothingOver ;
   142  public:
   143    laplaceSmoothingOver_condition() {
   144      name_store("$n",n) ;
   145      name_store("max_iterations",max_iteration);
   146      name_store("SmoothingOver",smoothingOver);
   147
   148      input("$n,max_iterations") ;
   149      output("SmoothingOver") ;
   150    }
   151    virtual void compute(const sequence &seq) {
   152      *smoothingOver = (*n >= *max_iteration);
   153    }
   154  } ;
   155
   156  register_rule<laplaceSmoothingOver_condition>
   157                register_laplaceSmoothingOver_condition;
   158
   159  //*********************************************************************
   160  class collapse_smoothing : public pointwise_rule {
   161    store<double>        result ;
   162    const_store<double>  smooth ;
   163
   164  public:
   165    collapse_smoothing() {
   166      name_store("smooth{n}",smooth) ;
   167      name_store("smooth",result) ;
   168
   169      input("smooth{n}") ;
   170      output("smooth") ;
   171      conditional("SmoothingOver{n}") ;
   172    }
   173    void calculate(Entity e) {
   174      result[e] = smooth[e] ;
   175    }
   176    virtual void compute(const sequence &seq) {
   177      do_loop(seq,this ) ;
   178    }
   179  } ;
   180
   181  register_rule<collapse_smoothing>
   182                register_collapse_smoothing ;
   183
   184  //*********************************************************************
\end{verbatim}
~


%\section { Sample Makefile }

\begin{verbatim}
#This is the Loci base directory.  The one listed here defines the
# debugging version for the sgi platform

LOCI_BASE = /ccs/mssl/general/loci/opt/Loci-sgi-rel-0-3-4f

# List the name of your compiled program here
TARGET = fire

# List your object files here
OBJS =  main.o advance_time.o boundary_conditions.o cell_center.o \
        collapse_time.o gradient.o initial_conditions.o stable_timestep.o


include $(LOCI_BASE)/sys.conf
include $(LOCI_BASE)/comp.conf

DEFINES = $(SYSTEM_DEFINES) $(MACHINE_SPECIFIC) $(DEBUG) -DBOUNDS_CHECK

INCLUDES  = -I$(LOCI_BASE)/include
LIBS      = -L$(LOCI_BASE)/lib -lLoci -lTools $(BASE_LIBS)

CP  = $(CPP) $(CPPOPT) -w
CC  = $(CPP) $(CPPOPT) -w

COPT     = $(CC_DEBUG) $(CC_OPT)
COPTLESS = $(CC_DEBUG) $(CC_OPT1)

LOPT     = $(CC_DEBUG) $(COPT)

JUNK = *~  core ti_files ii_files make.depend make.depend.bak

.SUFFIXES: .o .c .cc

.c.o:    ;  $(CC) $(COPT) $(DEFINES) $(INCLUDES) -c $*.c
.cc.o:   ;  $(CP) $(COPT) $(DEFINES) $(INCLUDES) -c $*.cc

default:
   $(MAKE) $(TARGET)

$(TARGET): $(OBJS) FRC
   $(LD) -o $(TARGET) $(OBJS) $(LOCAL_LIBS) $(LIBS) $(LDFLAGS)

FRC :

clean:
   rm -fr $(OBJS) $(TARGET) $(JUNK)  rii_files ii_files

\end{verbatim}

~


%\appendix
\chapter { Basic Classes}

\section { EntitySet Class }
\begin{verbatim}

class EntitySet 
{
public:
  void  friend  Union( const interval &);
  void  friend  Union( const entitySet &e);
  static friend Union( const entitySet &set1, const entitySet &set2);

  // Returns true if the integer value is present.
  bool  inSet( int_type indx ) const;

  // Returns the number of intervals.
  int   num_itervals() const;

  // A.Equal(B)
  // Returns true if both A and B point to the entitySet 
  //
  bool  Equal( const entitySet &p);

  // Provides a canonical ordering for entitySet objects so that
  // they can be used as keys to associate containers such as STL map<>

  bool  less_than( const entitySet &p);

  // Provides a canonical ordering for entitySet objects so that they
  // can be used as keys to associate containers such as STL map<> 

   bool  greater_than( const entitySet &p);

  //  A.min() returns the minimum value of the integer identifiers in 
  //  the entitySet A.
  int   Min() const;

  //  A.max() returns the maximum value of the integer identifiers in 
  //  the entitySet A.

  int   Max() const;
  void  Complement();

  inline ostream &operator << ( ostream &s, const entitySet &e);
  inline istream &operator << ( istream &s, entitySet &e);

  inline &operator += ( entitySet &e, int_type val);
inline &operator |= ( entitySet &e1, const entitySet &e2);
inline &operator &= ( entitySet &e, const interval &in);
inline &operator ^  ( const entitySet &e1, const entitySet &e2);
}
\end{verbatim}


\section { Store Class }

\begin{verbatim}

template <class T>
class store
{
public:
   // After declaring a store the next step is to allocate memory.
   // Member function allocate() allocates memory for all entities
   // in the specified entitySet.
   void        allocate(const entitySet &p);

   // Returns the domain of the store. It is nothing but the entitySet
   // over which the store is defined.
   entitySet   domain() const;
   
   // The operator [] is used to access the elements associated with 
   // a store. 
   T       &operator[](int indx);
}

\end{verbatim}

\section { storeVec $\&$ storeMat }
\begin{verbatim}
template <class T>
class store
{
public:

 // Specify the size of the vector.
  void setVecSize( int size );

 // Returns the size of the vector associated with each entity
  int vecSize() const;

  inline &operator []( int_type val);

  inline ostream &operator << ( ostream &s, const store &e);
  inline istream &operator << ( istream &s, store &e);

  inline &operator  = ( entitySet &e, int_type val);
  inline &operator += ( entitySet &e, int_type val);
  inline &operator -= ( entitySet &e, int_type val);
  inline &operator *= ( entitySet &e1, const entitySet &e2);
  inline &operator /= ( entitySet &e1, const entitySet &e2);
}
\end{verbatim}

\section { Parameter }

\begin{verbatim}


\end{verbatim}

\section { Maps }

\begin{verbatim}

class Map : public store_instance
{
public:
    Map();

}

class multiMap : public store_instance
{
public:
    multiMap();

}


\end{verbatim}


\begin{verbatim}


%\appendix{datatype}
\chapter{Datatypes }
%
\section {Introduction}
For the purpose of I/O and communication, datatypes can be considered
as abstract representations of the state of an object.  Usually this
is represented by the memory locations where data is stored.  In the
Loci framework, we gave to specify how to save and restore the state
of new object types explicitly to facilitate interprocessor
communication in heterogeneous environments or portable file I/O.
This information is provided to Loci using the traits mechanism.

In Loci, the datatype information is encapsulated in {\tt
  data\_schema\_traits} template class.  A user will need to provide a
specialized template for his/her own datatypes before they can be
used with Loci containers such as {\tt store}.
%
\section {Classification of Datatypes} 

Datatypes could be classified according to the their relationship
between computer memory representation and data which they hold.  The
basic distinction depends on the way C++ stores the object in memory.
If the object is represented in a continuous segment of memory with a
fixed layout, then then the memory layout is consistent with the
schemas used to send messages or write data to files.  If, instead the
objects data is scattered through memory or has a size that is
determined at run time, then the object must be serialized before it
can be sent as a message or placed in file storage.  An object whose
memory layout is compatible with the contiguous memory schema uses an
{\tt IDENTITY\_SCHEMA\_CONVERTER} for serialization.  Other objects
must also specify a schema converter.  The specifics of this will be
specified in the below examples:
%
\begin {verbatim}
typedef struct My_Type_t {
    int       iScalar;
    float     fScalar;
    char      cScalar;
    double    dScalar;
} My_Type1;

typedef struct My_Type_t {
    int        iScalar;
    float      fScalar;:
    char       cScalar;
    double    *dScalar;
} My_Type2;
\end{verbatim}
%
For {\em My\_Type1} C/C++ guarantees that the members of a structure
are stored in contiguous memory locations with some padding, so the
bitwise copy of {\em My\_Type1} will pack the data correctly, whereas
for {\em My\_Type2} the address of {\em dScalar} and not the data will
be copied. In the first case, the memory schema and the file schema is
handled by an {\bf Identity Function} called the {\tt
  IDENTITY\_SCHEMA\_CONVERTER }

In the second case, we need to explicitly convert between memory
schema and the file schema by specifying {\tt USER\_DEFINED\_CONVERTER}.

It should be mentioned here, that pointers are not the only source of
difference between these two datatype. STL container like vector,
list, queues, maps and virtual classes, all would need user to
serialize the state they contains, therefore, they also fall under the
category of user defined converter type.


\section {Predefined Datatypes in Loci}
Loci provides some of the frequently used datatypes so that user need
not write for themselves.These are
%
%
\begin{enumerate}
\item {\bf Atomic Datatype}
As the name implies, these datatypes are indivisible types. These type
are supported by all machines. These are building blocks for all other
datatypes. All native C/C++ datatypes are atomic datatypes
in Loci. The following table provides all the atomic datatypes support
by the Loci.
%
\begin{center}
\begin{tabular} [h] {|l|l|} \hline
Loci Datatype & Native C++ Datatype \\ \hline \hline
BOOL & bool \\ \hline
CHAR & signed char \\\hline
UNSIGNED\_CHAR & unsigned char \\\hline
SHORT & short \\\hline
UNSIGNED\_SHORT & unsigned short \\\hline
INT & int \\\hline
UNSIGNED & unsigned \\\hline
LONG & long \\\hline
UNSIGNED\_LONG & unsigned long \\\hline
FLOAT & float \\\hline
DOUBLE & double \\\hline
\end{tabular}
\end{center}
\item {\bf Array class}
Loci, provides template version of constant size array
\begin{verbatim}
template <class T, unsigned int n>
class Array {
   public:
   .
   .
   private:
      T  x[n];
};
\end{verbatim}
\item {\bf Vector class}
Loci, has 2D/3D vesions of vector class(mathematical vectors)
\begin{verbatim}
 template<T>
 struct vector3d {
      T x,y,z ;
 }

 template<T>
 struct vector2d {
      T x,y ;
 }
\end{verbatim}
\item {\bf STL containers parameterized by types using the {\tt
      IDENTITY\_SCHEMA\_CONVERTER}} 

All standard STL containers with predefined identity schema types are supported.
\begin{enumerate}
\item vector
\item list
\item queue
\item set
\item map
\end{enumerate}
\end{enumerate}
\section {Creating your own compound datatypes}
%
Loci, uses {\tt data\_schema\_traits} template class to determine
the datatype of an object. The generic behavior of template class is
not suitable for identifying or creating new datatype, so we use {\bf
  template Specialization } technique to customize or create new
datatypes . This {\tt data\_schema\_traits} class has one static
member function {\tt get\_type()} in which user specifies the
information about new datatype. A general skeleton for creating new
datatype look as follows
\begin{verbatim}
namespace Loci {
   // Skelton for datatype having identity schema
   template <>
   struct data_schema_traits <My_New_Type1 > {
         typedef IDENTITY_CONVERTER Schema_Converter;
         static DatatypeP get_type() {
                CompoundDatatypeP cmpd = CompoundFactory(My_New_Type1());
                LOCI_INSERT_TYPE(cmpd, My_New_Type1, member);
                   .
                   .
                return DatatypeP(cmpd);
         }
   };
}
\end{verbatim}
%
Where {\bf CompoundFactory} is one of the software design pattern for creating a 
new compound datatype object. Since the allocation of this object is done inside
the functions, the question will always arise, who is responsible for deleting
the object ? To destroy the objects when they are not needed, we use reference counting and the 
{\tt CompoundDatatypeP} class stands for {\em Reference Counting} version of
{\tt CompoundDatatype} class.
%
\par If your C/C++ structures contains members of predefined Loci datatypes, then it
is fairly easy to create corresponding Loci datatype. For example
\begin{verbatim}
 typedef struct My_Compound_Type_t {
        float                       fScalar;      
        vector3d<double>            vect3d;
        Array<double,2>             array1d;
        Array<Array<double,2>,4>    array2d;
 } My_Compound_Type;

 namespace Loci {
  template <> struct data_schema_traits <My_Compound_Type > {
     typedef IDENTITY_CONVERTER Schema_Converter;
     static DatatypeP get_type() {
         CompoundDatatypeP cmpd = CompoundFactory(My_Compound_Type());
         LOCI_INSERT_TYPE(cmpd, My_Compound_Type, fScalar);
         LOCI_INSERT_TYPE(cmpd, My_Compound_Type, vect3d);
         LOCI_INSERT_TYPE(cmpd, My_Compound_Type, array1d);
         LOCI_INSERT_TYPE(cmpd, My_Compound_Type, array2d);
         return DatatypeP(cmpd);
      }
    };
  }
\end{verbatim}
\section{Creating User Defined Datatype}
\par As explained earlier, whenever the memory allocation in any object is
not contiguous, it is the users responsibility to serialize the
data contained in the objects.  The skeleton of user defined schema type
will look as follows
\begin{verbatim}
   // Skelton for datatype having user defined schema
   template <>
   struct data_schema_traits <My_New_Type2 > {
        typedef USER_DEFINED_CONVERTER Schema_Converter;
        typedef char  Converter_Base_Type;
        typedef MyObject_SchemaConverter   Converter_Type;
   };
\end{verbatim}
\par {\em Note : {\tt Converter\_Base\_Type} could be any datatype with identity schema}
\par Following steps must be taken in order to define your own datatype
for Loci
\begin{enumerate} 
\item Specialize the {\tt data\_schema\_traits} class
\begin{itemize}
\item Specialize {\tt data\_schema\_traits} template class with your
class
\item \par declare in {\tt data\_schema\_traits} class
\begin{center}
{\tt typedef USER\_DEFINED\_CONVERTER Schema\_Converter}
\end{center}
\item specify what datatype will be used in conversion. This should be 
datatype with identity schema defined which means, that we can use any 
valid Loci atomic or compound datatype.
\item specify the class which has the responsibility of conversion (
Serialize class )
\end{itemize}
\item Specifying Serialize class
\begin{itemize}
\item Specify object reference in the constructor.
\item {\tt getSize()} member function returns the number of atomic
  datatypes used in this object.(It is not the size of the object in
  bytes)
\item {\tt getState()} member function gets the state of an
object into a contiguous buffer.
\item {\tt setState()} member function sets the state of an
object from a contiguous buffer.
\end{itemize}
\item Overload input/output stream functions.
Both atomic and compound datatypes have already been overloaded with
input/output streams in Loci.{\bf It is required that these function 
are overloaded even if the a user doesn't have intention of
using them.}
\end{enumerate}
%
Now we shall give one simple example to show how things work. We define
one structure with STL list inside it. Since list may not have contiguous
memory, we define it is defined with user defined schema
\begin{verbatim}

  namespace Loci {
  /////////////////////////////////////////////////////////////////////////////////
  // This is an example of conventional C/C++ structure 
  /////////////////////////////////////////////////////////////////////////////////

     struct My_Type {
      list<int>  alist;
      friend ostream& operator << (ostream &, const My_Type &);
      friend istream& operator >> (istream &, My_Type &);
  };

  //------------------------------------------------------------------------------/
  class My_Type_SchemaConverter;    // Forward Declaration of class
  //------------------------------------------------------------------------------/
  // Specialize the data_schema_traits class with "My_Type" class
  //------------------------------------------------------------------------------/

  template <>
  struct data_schema_traits<My_Type> {
    // This class has user defined schema
    typedef USER_DEFINED_CONVERTER Schema_Converter ;

    // Since list contains "int" we use it directly for our converion
    typedef int Converter_Base_Type ;
   
    // Here we specify the class used for serialization/deserialization 
    // purpose
    typedef My_Type_SchemaConverter Converter_Type ;
  };

  //------------------------------------------------------------------------------/
  // Define a class which has the responsibity of serialization and deserialization 
  // of "My_Type" class
  //------------------------------------------------------------------------------/

  class My_Type_SchemaConverter {
    // For the schema converter, we always store a reference to the object
    // we are converting schmata for.
    My_Type &RefObj ;
    public:
        explicit My_Type_SchemaConverter(My_Type &new_obj): RefObj(new_obj) {}
        //
        // This member function returns number of elements of type defined
        // in Converter_Base_Type. It is not the size in bytes.
        //
        int getSize() const {
            return RefObj.alist.size() ;
        }
      
        // Get the state of an object "RefObj" into an array and also size of
        // array. This is a serialization step. 
        void getState(int *buf, int &size) {
             size = getSize() ;
             int ii=0;

             list<int> :: const_iterator ci;
             list<int> :: const_iterator begin = RefObj.alist.begin();
             list<int> :: const_iterator end   = RefObj.alist.end();
             for(ci = begin; ci != end; ++ci) 
                 buf[ii++] = *ci;
        }
        //
        // From a given array, construct the object. This is "Deserialization Step"
        //
        void setState(int *buf, int size) {
             RefObj.alist.clear();
             list<int> :: iterator ci;
             for(int i=0;i<size;++i)
                 RefObj.alist.push_back(buf[i]);
        }
  };
  }
  //------------------------------------------------------------------------------/
\end{verbatim}
%
%
\section{Inner Details about Compound Datatype}
Compound datatypes are similar to structures in C/C++. These datatypes
are a collection of heterogeneous atomic or fixed sized array datatypes. 
Every member of these datatype has a unique name within the datatype and 
they occupy  non-overlapping memory locations.
%
%
\par In Loci, this datatype is declared in {\tt CompoundType}
class. The corresponding counted pointer class is {\tt
CompoundDatatypeP}.
%
\par A new member can be inserted into the new datatype in either way
\begin{itemize}
\item Using member function of CompoundType class
\begin{center}
insert( member\_name, offsetof(type, member-designator), member\_datatype);
\end{center}
where {\tt offsetof} is a standard C/C++ function which provides offset of any
member (designated by member-designator) in C/C++ structure (designated by type).
{\em member\_datatype} could be any valid Loci datatype.
%
\item Using predefined macro
\begin{center}
LOCI\_INSERT\_TYPE( compound\_object, compound\_class, insert\_member);
\end{center}
\par where {\em compound\_object} is the compound datatype for {\em compound\_class}
and {\em insert\_member} is the required member of {\em compound\_class} which
is inserted into new datatype. 
\par In order to use the macro {\em insert\_member} must be a first class object
and and its own type should be identified by {\tt data\_schema\_traits} class.
\end{itemize}
\par In the following sections we shall gives some examples of
creating different Loci datatypes.
%
\subsection {Creating compound datatype with only atomic datatypes}
The following is a very simple C/C++ structure, which contains only native datatypes.
For this structure, we would like to create Loci datatype, which is also described
below.
\begin{verbatim}
 typedef struct My_Compound_Type_t {
        int        iScalar;
        float      fScalar;
        char       cScalar;
        double     dScalar;
 } My_Compound_Type;
\end{verbatim}
\begin{verbatim}
  // data_scheme_traits should be defined in Loci namespace
 namespace Loci {
   // Specialize the class with the new class
   template <> 
   struct data_schema_traits <My_Compound_Type > {

        // Specify that ideneity Schema will be used for this class
        typedef IDENTITY_CONVERTER Schema_Converter;

        // define the member function
        static DatatypeP get_type() {
              // Create a new product "cmpd" from the factory pattern 
              CompoundDatatypeP cmpd = CompoundFactory(My_Compound_Type());
              // Insert a new member into the new compound datatype
              LOCI_INSERT_TYPE(cmpd, My_Compound_Type, iScalar);
              LOCI_INSERT_TYPE(cmpd, My_Compound_Type, fScalar);
              LOCI_INSERT_TYPE(cmpd, My_Compound_Type, cScalar);
              LOCI_INSERT_TYPE(cmpd, My_Compound_Type, dScalar);
             // return pointer to the base class
             return DatatypeP(cmpd);
         }
      };
  }
\end{verbatim}
\subsection{Creating compound datatype with arrays}
In the following example, we have inserted a two dimensional array
into the structure and define corresponding Loci datatype.
\begin{verbatim}
 namespace Loci {
     typedef struct My_Compound_Type_t {
         int     iScalar;
         float   fScalar;
         double  dScalar;
         Array<double,10>          dArray1D;
         Array<Array<double,3>,5>  dArray2D;
     } My_Compound_Type;
    
     template<>
     struct data_schema_traits<My_Compound_Type> {
          typedef IDENTITY_CONVERTER Schema_Converter ;
          static DatatypeP get_type() {
            CompoundDatatypeP ct = CompoundFactory(My_Compound_Type()) ;
            LOCI_INSERT_TYPE(ct,My_Compound_Type, iScalar) ;
            LOCI_INSERT_TYPE(ct,My_Compound_Type, fScalar) ;
            LOCI_INSERT_TYPE(ct,My_Compound_Type, dScalar) ;
            LOCI_INSERT_TYPE(ct,My_Compound_Type, dArray1D) ;
            LOCI_INSERT_TYPE(ct,My_Compound_Type, dArray2D) ;
            return DatatypeP(ct) ;
          }
      } ;
 }
\end{verbatim}
\subsection{Creating compound datatype with nested compound
datatypes}
In this example, we would demonstrate that any member with valid
datatype could be inserted into compound datatype in an hierarchal fashion. There
is no restriction on number of levels used to define a datatype.
\begin{verbatim}
namespace Loci {
   struct Velocity {
     Array<double,3>  comp;
   };
   // Define traits for "Velocity" structure
   template <>
   struct data_schema_traits<Velocity> {
     typedef IDENTITY_CONVERTER Schema_Converter ;
     static DatatypeP get_type() {
       Velocity v;
       return getLociType(v.comp);
     }
   };
   // A structure contains another structure
   struct CellAttrib {
     int      local_id;
     double   density;
     double   pressure;
     Velocity vel;
   };
 
   // Define traits for "CellAttrib" class. Notice that "vel" is a structure
   // and since its type is already defined, we can insert it similar to other
   // members.
   template <>
   struct data_schema_traits<CellAttrib> {
     typedef IDENTITY_CONVERTER Schema_Converter ;
     static DatatypeP get_type() {
       CompoundDatatypeP  cmpd = CompoundFactory( CellAttrib() );
       LOCI_INSERT_TYPE(cmpd, CellAttrib, local_id );
       LOCI_INSERT_TYPE(cmpd, CellAttrib, density  );
       LOCI_INSERT_TYPE(cmpd, CellAttrib, pressure );
       LOCI_INSERT_TYPE(cmpd, CellAttrib, vel );
       return DatatypeP(cmpd);
     }
   };
 }
\end{verbatim}

\section{Array Datatype}
Array datatype consists of homogeneous collection of both compound and 
atomic datatypes. We can defined array datatypes for standard C/C++
arrays. In order to create array datatype, a user need to provide
\begin{itemize}
\item the rank of the array, i.e. number of dimensions 
\par {\em Note At present, Loci can support arrays with maximum rank of
4. This limitation comes from HDF5 library. If the user wants higher ranked
arrays, using {\bf Array class} is one solution}
\item the size of each dimension
\item the datatype of each element of the array
\end{itemize}
\par In Loci, this datatype is declared in {\tt ArrayType} class. The 
corresponding counted pointer class is {\tt ArrayDatatypeP}.
%
\begin{itemize}
\item Create 1 D dimensional array datatype
\begin{verbatim}
dims[0]= 100;
ArrayType  atype(Loci::DOUBLE, 1, dims);
\end{verbatim}
%
\item Create 2 D dimensional array datatype
\begin{verbatim}
dims[0] = 10;
dims[1] = 20;
ArrayType  atype(Loci::DOUBLE, 2, dims);
\end{verbatim}
%
\item Create 3 D dimensional array datatype
\begin{verbatim}
dims[0] = 10;
dims[1] = 20;
dims[2] = 30;
ArrayType  atype(Loci::DOUBLE, 3, dims);
\end{verbatim}
%
\item Create 4 D dimensional array datatype
\begin{verbatim}
dims[0] = 10;
dims[1] = 20;
dims[2] = 30;
dims[3] = 40;
ArrayType  atype(Loci::DOUBLE, 4, dims);
\end{verbatim}
\end{itemize}
%
%
\par For example
\begin{verbatim}
typedef struct My_Compound_Type_t {
    int       iScalar;
    float     fScalar;
    char      cScalar;
    double    dScalar[10][5][2];
}My_Compound_Type;
\end{verbatim}
%
This datatype has multidimensional array, which is not first class
objects. We can use {\tt ArrayDatatype} to specify the Loci Datatype as
\begin{verbatim}
int     rank  = 3;
int     dim[] = {10, 5, 2};
int     sz    = 100*sizeof(double);

My_Compound_Type     type;
CompoundDatatypeP cmpd    = CompoundFactory(My_Compound_Type());
DatatypeP         atom    = getLociType(type.dScalar[0][0][0]);
ArrayDatatypeP    array_t = ArrayFactory(atom, sz, rank, dim);
cmpd->insert("dScalar", offsetof(My_Compound_Type, dScalar), DatatypeP(array_t));
\end{verbatim}

This is definitely cumbersome, Instead of using array in this way, if
we had used
\begin{verbatim}
typedef  Array < double, 10 > Array1D;
typedef  Array < Array1D, 5 > Array2D;
typedef  Array < Array2D, 2 > Array3D;
typedef struct My_Compound_Type_t {
   int             iScalar;
   float           fScalar;
   char            cScalar;
   Array3D         dScalar;
} My_Compound_Type;
\end{verbatim}
then we can use {\tt LOCI\_INSERT\_TYPE} macro to insert {\tt
dScalar} into new compound datatype.

%

%\appendix{io}
\chapter {Input Output}
Loci used HDF5 (Hierarchical Data Format)  to read-write data from its 
containers. All the containers of Loci has the following member virtual functions
%
\begin{itemize} 
\item virtual void readhdf5( hid\_t group, entitySet \&  eset);
\item virtual void writehdf5( hid\_t group, entitySet \& eset) const ;
\end{itemize}
%
{\em group\_id} is the valid HDF5 group in which data is written. {\em eset} is
the entitySet for which data is written. One simple example of
creating a group is given below
%
\begin{verbatim} 
hid_t file_id  = H5Fopen(filename, H5F_ACC_RDONLY, H5P_DEFAULT);
hid_t group_id = H5Gopen(file_id, groupname.c_str() );
\end{verbatim} 

\section {Rules for writing facts}
The following conventions are adopted for writing containers in files
All files written in HDF5 ends with .hdf5
All the facts are written in a single file, but in different groups.
\subsection {Distributed facts}
\begin{enumerate}
\item they are written in {\em filename\_p0.hdf5}, {\em filename\_p1.hdf5}, \dots
\item they use local numbering.
\item local to global numbering is provided in each file and this
mapping is shared by all facts in the fact database.
\item each process is responsible for writing only its own data, and
no clone data is written.
\item if there is only process, facts are written with global numbering.
\end{enumerate}

\subsection {Undistributed facts}

\begin{enumerate}
\item they are written in {\em filename\_p0.hdf5}
\item facts are written with global numbering.
\end{enumerate}
\section {Rules for reading data from files}

\begin{enumerate}
\item A processor can read any number of files in round-robin fashion. 
\item If the facts are distributed then the objects move to their
correct location after reading the file. 
\item If the number of processes is only one, facts are stored with
global numbering
\end{enumerate}

\section {Reading and Writing facts}
\par Reading and writing facts in Loci is very simple. 
\begin{itemize}
\item  Specify the appropriate Loci datatype 
(if they are not already defined in Loci framework).
\item  Specify Loci container (i.e. store, storeVec ...etc)
\item  Create the facts using the member function
\begin{verbatim}
   fact_db  facts;
   facts.create_fact(factname, container); 
\end{verbatim}
\item  Write the facts database in file using (.hdf5 will be appended
to the file)
\begin{verbatim}
   facts.write_hdf5( fact_database_filename ); 
\end{verbatim}
\item  Read the facts database from file using (.hdf5 will be appended
to the file)
\begin{verbatim}
   facts.read_hdf5( fact_database_filename ); 
\end{verbatim}
\item Using h5dump tool to see the contains of the file
\end{itemize}
%
\par Now we shall give three examples for writing facts
\begin{itemize}
\item  In the first example, we shall write facts with atomic datatype, which
are predefined in Loci. Using these datatype doesn't require anything extra 
from the user.
\item In the second example, we will write compound datatype, in which user
specify {\em Identity Converters} and insert them into Loci datatype. Once the
new datatype is build, using is as simple as example one.
\item In the third example, we will write class which uses {\em user defined conversion}
\end{itemize}
\section { Writing atomic datatype }
In the following example, we shall use {\em store} container and fill it with
some data. We will demonstrate how the facts database is called to write the
contains of the container in HDF5 data format.
\begin{verbatim}
  #include <Loci.h>
  using namespace std;
 
  store<int>      data;   // Container with integer datatype
  entitySet       eset;   // EntitySet
  fact_db         facts;  // facts database

  //*******************************************************************
  // Create a fact, fill it with some data and register it with facts
  // database
  //*******************************************************************
  void GenerateData()
  {
    int num = 10;
    eset = interval(1,num);

    data.allocate(eset);
    for( int j = 1; j <= num; j++) {
      data[j] =   100+j;
    }

    facts.create_fact("store",data) ;
  }
  //*******************************************************************

  int main(int argc, char *argv[])
  {
    GenerateData();
    facts.write_hdf5( "example");
  }
  //*******************************************************************
\end{verbatim}
The execution of the above program will produce {\em example\_p0.hdf5} file. 
\begin{verbatim}
 HDF5 "example_p0.hdf5" {
 GROUP "/" {
   GROUP "store" {
      DATASET "Interval Set" {
         DATATYPE { H5T_STD_I32BE }
         DATASPACE { SIMPLE ( 2 ) / ( 2 ) }
         DATA {
            1, 10
         }
      }
      DATASET "VariableData" {
         DATATYPE { H5T_STD_I32BE }
         DATASPACE { SIMPLE ( 10 ) / ( 10 ) }
         DATA {
            101, 102, 103, 104, 105, 106, 107, 108, 109, 110
         }
      }
   }
}
}
\end{verbatim}
\par Notice the output, HDF5 write all necessary information about the container. EntitySet is 
written under the {\em Interval Set} dataset and the data is written under {\em VariableData}
dataset. Each datatype and dataspace is also specified in the file. {\bf H5T\_STD\_I32BE} tells
us that data was 32 bit, and big-endian integer data.
%
\section { Writing Compound datatype}
Creating datatypes which contains another compound datatypes is fairly
simple. Just define each datatype individually and add these types in 
the parent datatype.

\begin{verbatim}
 #include <Loci.h>
 using namespace std;

 namespace Loci {
   struct Velocity {
     Array<double,3>  comp;
   };

   struct CellAttrib {
     int      local_id;
     double   density;
     double   pressure;
     Velocity vel;
   };

   template <>
   struct data_schema_traits<Velocity> {
     typedef IDENTITY_CONVERTER Schema_Converter ;
     static DatatypeP get_type() {
       return DatatypeP(comp);
     }
   };

   template <>
   struct data_schema_traits<CellAttrib> {
     typedef IDENTITY_CONVERTER Schema_Converter ;
     static DatatypeP get_type() {
       CompoundDatatypeP  cmpd = CompoundFactory( CellAttrib() );
       LOCI_INSERT_TYPE(cmpd, CellAttrib, local_id );
       LOCI_INSERT_TYPE(cmpd, CellAttrib, density  );
       LOCI_INSERT_TYPE(cmpd, CellAttrib, pressure );
       LOCI_INSERT_TYPE(cmpd, CellAttrib, vel );
       return DatatypeP(cmpd);
     }
   };


 }
 //*******************************************************************
 store<Loci::CellAttrib>  data;   // Container
 entitySet                eset;   // EntitySet
 fact_db                  facts;  // facts database

 //*******************************************************************
 // Create the facts and register with facts database
 //*******************************************************************

 void GenerateData()
 {
   int num = 2;
   eset = interval(1,num);

   data.allocate(eset);
   for( int j = 1; j <= num; j++) {
     data[j].local_id  =   j;
     data[j].density   =   0.25*j;
     data[j].pressure  =   1.25*j;
     data[j].vel.comp[0]  =   1.05*j;
     data[j].vel.comp[1]  =   2.05*j;
     data[j].vel.comp[2]  =   3.05*j;
   }
   facts.create_fact("store",data) ;
 }
 //*******************************************************************
 int main(int argc, char *argv[])
 {
   GenerateData();
   facts.write_hdf5( "example");
 }
 //*******************************************************************
\end{verbatim}
\begin{verbatim}
  HDF5 "example_p0.hdf5" {
  GROUP "/" {
    GROUP "store" {
       DATASET "Interval Set" {
          DATATYPE { H5T_STD_I32BE }
          DATASPACE { SIMPLE ( 2 ) / ( 2 ) }
          DATA {
             1, 2
          }
       }
       DATASET "VariableData" {
          DATATYPE {
             H5T_STD_I32BE "local_id";
             H5T_IEEE_F64BE "density";
             H5T_IEEE_F64BE "pressure";
             {
                H5T_IEEE_F64BE "Array"[3];
             } "velocity";
          }
          DATASPACE { SIMPLE ( 2 ) / ( 2 ) }
          DATA {
             {
                1,
                0.25,
                1.25,
                {
                   [ 1.05, 2.05, 3.05 ]
                }
             },
             {
                2,
                0.5,
                2.5,
                {
                   [ 2.1, 4.1, 6.1 ]
                }
             }
          }
       }
    }
 }
 }
\end{verbatim}

\section { Writing containers with user defined datatype}
All attribute containers have to store information from which state of an user defined
datatype should be reconstructed. Two different cases arises with user defined datatype
An user defined container may contain objects of varying size and shapes. All these
information must be stored in the file for later retrieval. 

\par All user defined datatypes requires
\begin{enumerate}
\item  input and output stream overloaded functions.
\item  Specialized template {\em data\_schema\_traits} class
\item  A class which can pack and unpack data of an object
\end{enumerate}

\begin{verbatim}
   #include <Loci.h>
   using namespace std;
 
   namespace Loci {
     struct My_Type {
      list<int>  alist;
      friend ostream& operator << (ostream &, const My_Type &);
      friend istream& operator >> (istream &, My_Type &);
    };

    //*******************************************************************
    inline std::ostream& operator << ( std::ostream &s, const My_Type &obj)
  {
    list<int> :: const_iterator ci;
    list<int> :: const_iterator begin = obj.alist.begin();
    list<int> :: const_iterator end   = obj.alist.end();
    for( ci = begin; ci != end; ++ci)
      s << *ci <<" ";
    return s;
  }
    //*******************************************************************
    inline std::istream& operator >> ( std::istream &s, My_Type &obj)
  {
    obj.alist.clear();
    int newval;
    s >> newval;
    obj.alist.push_back( newval );
    return s;
  }
  //*******************************************************************
  class My_Type_SchemaConverter;

  template <>
  struct data_schema_traits<My_Type> {
    typedef USER_DEFINED_CONVERTER Schema_Converter ;

    typedef int Converter_Base_Type ;
    typedef My_Type_SchemaConverter Converter_Type ;
  };

  class My_Type_SchemaConverter {
    // For the schema converter, we always store a reference to the object
    // we are converting schmata for.
    My_Type &RefObj ;
  public:
    explicit My_Type_SchemaConverter(My_Type &new_obj): RefObj(new_obj) {}
    int getSize() const {
      return RefObj.alist.size() ;
    }
    void getState(int *buf, int &size) {
      size = getSize() ;
      int ii=0;

      list<int> :: const_iterator ci;
      list<int> :: const_iterator begin = RefObj.alist.begin();
      list<int> :: const_iterator end   = RefObj.alist.end();
      for(ci = begin; ci != end; ++ci)  {
        buf[ii++] = *ci;
      }
    }
    void setState(int *buf, int size) {
      RefObj.alist.clear();
      list<int> :: iterator ci;
      for(int i=0;i<size;++i)
        RefObj.alist.push_back(buf[i]);
    }
  };
  }


  store<Loci::My_Type>   data;   // Container
  entitySet              eset;   // EntitySet
  fact_db                facts;  // facts database

  void GenerateData()
  {
    int num = 2;
    eset = interval(1,num);

    data.allocate(eset);
    int indx = 1;
    for( int j = 0; j < 3; j++) {
      data[1].alist.push_back(indx);
      indx++;
    }

    indx = 1;
    for( int j = 0; j < 5; j++) {
      data[2].alist.push_back(indx+100);
      indx++;
    }

    facts.create_fact("store",data) ;
 }
    
 //*******************************************************************
 int main(int argc, char *argv[])
 {
   GenerateData();
   facts.write_hdf5( "example");
 }
 //*******************************************************************
\end{verbatim}
\begin{verbatim}
HDF5 "example_p0.hdf5" {
GROUP "/" {
   GROUP "store" {
      DATASET "ContainerSize" {
         DATATYPE  H5T_STD_I32BE  
         DATASPACE  SIMPLE { ( 2 ) / ( 2 ) } 
         DATA {
            3, 5
         } 
      } 
      DATASET "Interval Set" {
         DATATYPE  H5T_STD_I32BE  
         DATASPACE  SIMPLE { ( 2 ) / ( 2 ) } 
         DATA {
            1, 2
         } 
      } 
      DATASET "VariableData" {
         DATATYPE  H5T_STD_I32BE  
         DATASPACE  SIMPLE { ( 8 ) / ( 8 ) } 
         DATA {
            1, 2, 3, 101, 102, 103, 104, 105
         } 
      } 
   } 
} 
} 
\end{verbatim}
\section {Distributed facts }
\begin{verbatim} 
      #include <Loci.h>
      using namespace std;
    
      int                 my_rank, numprocs;  // MPI information
      storeVec<int>       data;               // Container
      entitySet           eset;               // EntitySet
      fact_db             facts;              // facts database
    
    
      void GenerateData()
      {
        int num = 10;
        eset = interval(1,num);
        data.setVecSize(2);
    
        data.allocate(eset);
        for(int i = 1; i < num+1; i++) {
            data[i][0] = 10*i;
            data[i][1] = 100*i;
        }
        facts.create_fact("storeVec",data) ;
      }
    
      int main(int argc, char *argv[])
      {
        Loci::Init(&argc, &argv) ;
        GenerateData();
    
        std::vector<entitySet> partition(Loci::MPI_processes) ;
        partition = Loci::generate_distribution(facts, rdb, Loci::MPI_processes) ;
        Loci::distribute_facts(partition, facts, rdb) ;

        facts.write_hdf5( "example");
    
        Loci::Finalize();
      }
\end{verbatim} 
Execute the above program with 2 processors and the output will be written in {\em example\_p0,hdf5},
and {\em example\_p1.hdf5}
\begin{verbatim} 
     1  HDF5 "example_p1.hdf5" {
     2  GROUP "/" {
     3     GROUP "ProcessorID" {
     4        DATASET "Processor" {
     5           DATATYPE { H5T_STD_I32BE }
     6           DATASPACE { SIMPLE ( 2 ) / ( 2 ) }
     7           DATA {
     8              1, 2
     9           }
    10        }
    11     }
    12     GROUP "l2g" {
    13        DATASET "Interval Set" {
    14           DATATYPE { H5T_STD_I32BE }
    15           DATASPACE { SIMPLE ( 2 ) / ( 2 ) }
    16           DATA {
    17              0, 4
    18           }
    19        }
    20        DATASET "Map" {
    21           DATATYPE { H5T_STD_I32BE }
    22           DATASPACE { SIMPLE ( 5 ) / ( 5 ) }
    23           DATA {
    24              1, 2, 6, 9, 10
    25           }
    26        }
    27     }
    28     GROUP "storeVec" {
    29        DATASET "Interval Set" {
    30           DATATYPE { H5T_STD_I32BE }
    31           DATASPACE { SIMPLE ( 2 ) / ( 2 ) }
    32           DATA {
    33              0, 4
    34           }
    35        }
    36        DATASET "VariableData" {
    37           DATATYPE { H5T_STD_I32BE }
    38           DATASPACE { SIMPLE ( 10 ) / ( 10 ) }
    39           DATA {
    40              10, 100, 20, 200, 60, 600, 90, 900, 100, 1000
    41           }
    42        }
    43        DATASET "VecSize" {
    44           DATATYPE { H5T_STD_I32BE }
    45           DATASPACE { SIMPLE ( 1 ) / ( 1 ) }
    46           DATA {
    47              2
    48           }
    49        }
    50     }
    51  }
    52  }
\end{verbatim} 

\par In the above example line 3-11, {\em ProcessorID} group provides 
information about distribution. Line 6 says that number of entries in
this group is two, Line 8 specifies that this data was written on
process ID 1, and the maximum number of process on which
the simulation was carried out was 2. \\

\par In line 12-19, information about local to global numbering is
provided which is owned by process id 1 {specified by {em ProcessorID}
group} Line 17 says that there are 5 local entities with {0,1, ...4} ids. 

\par Line 20-27 provides mapping from local to global entitySet. The
mapping in one-to-one and for this example as follows.

\begin{center}
\begin{tabular}[h]{|c|c|} \hline
Local numbering & Global numbering \\ \hline
0   &  1         \\
1   &  2         \\
2   &  6         \\
3   &  9         \\
4   &  10        \\ \hline
\end{tabular}
\end{center}

\par Line 28-51 actually write the attribute data. In this example, in
the group {\em storeVec}
we write the storeVec container data. Any {\em Loci} datatypes could
be used to write data in
the in this group. (Refer to Loci Datatypes for more details )

\begin{center}
\begin{tabular}[h]{|c|c|} \hline
Local numbering & storeVec container Data \\ \hline
0   &  (10,100)         \\
1   &  (20,200)         \\
2   &  (60,600)         \\
3   &  (90,900)         \\
4   &  (100,1000)      \\ \hline
\end{tabular}
\end{center}

\par Line 43-49 group specifies that the size of storeVec
container. Line 47 inform that the size is 2



\chapter{Loci Helper Classes}
\section{Loci Helper Classes}

Loci also provides a few helper classes that are often useful in
numerical computations.  
One is the {\tt Array} template class which provides a
mechanism for creating Arrays as first class objects that are
appropriate for using as classes used in templated containers. ({\it
Never use C arrays in templated containers.  Their semantics are
different from other C++ objects and may break templated code in
unexpected ways.})  In addition to the {\tt Array} template class,
classes for three and two dimensional vectors are also provided.  See
the following example code to see how to use these helper classes.

\include{helpers_cc}

\chapter{Using Third Party Libraries}

As is often the case when writing high-performance numerical simulation
software, there may come a time when it is desireable to utilize some third
party's existing set of library routines rather than having to write your
own versions.  For example, the Portable, Extensible Toolkit for Scientific
Computing (PETSc), from Argonne National Laboratory, is a highly optimized
library for the numerical solution of partial differential equations on
parallel and serial architectures.  The library includes a large suite of
tools for solving a wide variety of linear and nonlinear systems of
equations.

As described in the preceeding chapters, Loci expects first class objects
to have methods defined to accomplish data serialization as well as input
and output.  Since most third party libraries will make use of their own
set of data structures, Loci will not be able to effectively manage them
as it does for first class objects which it is already familiar with.  So,
in order to utilize third party libraries in the context of a Loci
application, the user has a choice:  to write all of the supporting code
that Loci expects, or to utilize the {\tt blackbox} container to hide the
implementation details of the third party library from Loci.

\section{Blackbox Usage}
The following example shows how the blackbox container can be utilized
in order to make use of a third party library (in this case PETSc) with a
minimum of extra effort.

\subsection*{main.cc}
\input{petsc1_main_cc}

\subsection*{rules.cc}
\input{petsc1_rules_cc}


\end{document}

