\section{The {\tt options\_list} class}

In many circumstances the user needs to input complex optional
information to a solver.  These parameters are usually provided as a
part of the initial facts that are provided by the user and may express
several options to be used by a particular functional part of the
system.  To help standardize the input of this information {\it Loci}
provides a powerful system for inputting complex and hierarchical
information called the {\tt options\_list}.  This form of input allows
named assignment of data, specification of units that data is
presented, input of lists and other complex data forms.  A common use
of the {\tt options\_list} is the assignment of boundary conditions in
Loci solvers.  This section will discuss how to use the {\tt
  options\_list} type to allow for more powerful inputs to your {\it
  Loci} programs.

When a parameter type is an {\tt options\_list} type, then the input
in the ASCII version of the initial facts (the {\tt vars} file) will
have a variable that is delimited by less-than and greater-than
symbols.  For example the boundary conditions are input using an {\tt
  options\_list} type:
\begin{verbatim}
boundary_conditions:<
 BC_1=specified(Twall=500K), 
 BC_2=specified(Twall=300K),
 BC_3=adiabatic, BC_4=adiabatic  
>
\end{verbatim}
The basic function of the {\tt options\_list} is to give a way of
specifying multiple parameters.  In this case the parameters are the
boundary surface names which are assigned to the boundary condition.
The boundary conditions are represented by complex data types. For the
specified case the boundary conditions are represented by a named
function type where the arguments contain a named list of variables
(which can be thought of as another nested {\tt options\_list}).  The
general structure of an {\tt options\_list} variable will be as
follows:  First a '<' character which opens the options block, then a
comma separated list of {\it name} = {\it argument} where {\tt
  argument} can be the following:
\begin{enumerate}
  \item a floating point number
  \item a floating point number with units annotation
  \item a string value in ``quotes''
  \item a list of floating point numbers enclosed in `[' and `]'
  \item a list of name assignments enclosed in `[' and `]'
  \item a name
  \item a name with arguments enclosed in (parenthesis)
\end{enumerate}

The options list can be used in two modes: one where the names of the
variable names are limited to a preset list or the default mode where
any name can be used on the input. The first mode is enabled by
passing a string to the constructor that contains a colon separated
list of allowable variable names.   The simplest way to use options
list for inputs is to simply put the class in a {\tt param} and allow it to
be read in with a optional rule.  For example:

\begin{verbatim}
$type param<options_list> boundary_conditions ;

$rule optional(boundary_conditions) {}
\end{verbatim}

Once the option list is read in the following member functions can be
used to interrogate the values stored in the option list:
\begin{enumerate}
\item {\tt optionExists}: This member function will return true if the
  variable name given in the string argument is defined in the option
  list.
\item {\tt getOptionNameList}: This member function returns a list of
  the names stored in the option list using the type {\tt
    options\_list::option\_namelist}.
\item {\tt getOptionValueType}: This member function returns the value
  type associated with the option passed in as the argument.  The
  value type may be {\tt REAL}, {\tt NAME}, {\tt FUNCTION}, {\tt
    LIST},  {\tt STRING}, {\tt BOOLEAN}, or {\tt
    UNIT\_VALUE}. 
\item {\tt getOption}:  This member function retrieves the value
  associated with the named option. The second argument is the
  returned value and may be the types {\tt bool}, {\tt double}, {\tt
    string}, or {\tt options\_list::arg\_list}.
\item {\tt getOptionUnits}: This member function is passed the name, a
  requested unit, a double variable where the returned value will be
  stored, or alternatively the last argument may be a {\tt vector3d}
  for retrieving 3d vector values.
\end{enumerate}

For simple data input the {\tt options\_list} type is straightforward
to use.  Some examples:

\begin{verbatim}
   void extract_data(const options_list &ol,
                     double &alpha,
                     double &temperature,
                     string &name,
                     bool &adiabatic) {
      // for nondimensional data just use getOption
      ol.getOption("alpha",alpha) ; 
      // If we want a default value set it and check if the
      // option is available
      temperature = 300 ; // default value
      if(ol.optionExists("temperature")) {
        // for dimensional data specify what units you want
        // to retrieve the data in.  This will be the default
        // units for this input
        ol.getOptionUnits("temperature","kelvin",temperature) ;
      }
      // for other datatypes the input works the same
      ol.getOption("filename",name) ;
      ol.getOption("adiabatic",adiabatic) ;
   }
\end{verbatim}

For more complex types, for example lists of values which can occur in
options lists for cases such as:
\begin{verbatim}
valueList : < list = [ 1.0, 2.5, 1.0, 5.0 ],
              funclist = funcname(1.0, 2.5, 1.0, 5.0) >
\end{verbatim}
In these cases the list of values can be extracted by first extracting
the {\tt arg\_list} and then looping over the list and extracting the 
data item associated with each term.  For example, a generic function
that can extract a list of doubles assigned to a variable would be
implemented as:
\begin{verbatim}
  void getList(const options_list &ol, std::string vname,
               vector<double> &valuelist) {
    if(ol.getOptionValueType(vname) == Loci::LIST) {
      // It is a list so get list
      Loci::options_list::arg_list list ;
      ol.getOption(vname,list) ;
      int size = list.size() ;
      // loop over list and insert into valuelist
      for(int i=0;i<size;++i) {
        if(list[i].type_of() != Loci::REAL) {
          cerr << "improper list for " << vname << endl ;
          Loci::Abort() ;
        }
        double val = 0 ;
        // get each list item with get_value call
        list[i].get_value(val) ;
        valuelist.push_back(val) ;
      }
    }
  }
\end{verbatim}

For more powerful use of the option lists it is possible to consider
recursive specifications where the contents of a list or an argument
list of a function can be converted to an option list and then queried
further.  For example, in the boundary condition example above the
specified boundary condition had arguments in the same form as the
options list itself.  In this case it is possible to parse this
structure by using the {\tt arg\_list} to create another options list
for the purpose of parsing the structure.  For example, one way to
parse the options of the specified boundary condition is as follows:

\begin{verbatim}
   void parseBoundary(const options_list &ol, string bcname) {
     if(ol.optionExists(bcname)) {
       if(ol.getOptionValueType(bcname) == FUNCTION) {
         string bctype ;
         options_list.arg_list fvalues ;
         // get the function and the arguments
         ol.getOption(bcname,bctype,fvalues) ;
         // create options list for function arguments
         options_list fol ;
         fol.Input(fvalues) ;
         if(bctype == "specified") { // now it is a specified bc
           // now check argument options (e.g. specified(Twall=300K))
           if(fol.optionExists("Twall")) {
              double Twall ;
              fol.getOptionUnits("Twall","kelvin",Twall) ;
              // ...
\end{verbatim}
