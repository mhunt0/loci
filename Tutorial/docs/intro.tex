\chapter{An Introduction to Loci}

Loci is a both a C++ library and a programming framework specifically
designed for developing computational simulations of physical fields,
such as computational fluid dynamics.  One advantage the framework
provides is automatic parallelization.  Once an application is
described within the Loci framework, the application can be executed
in parallel without change, even though the description within the
framework included no explicit parallel directives.  A particular
advantage of the programming framework is that it provides a formal
framework for the development of simulation knowledge-bases using {\em
  logic-relational} abstractions.  While the approach will probably be
alien to most who begin to use it, the programming model is extremely
powerful and worth the patience required to adjust to a new way of
thinking about programming.

Several major components comprise the Loci framework and these include
facilities for managing sets of entities, containers that can
associate values with entities, a database for managing user provided
facts, a database for managing user provided rules, and finally, a
query system that can generate programs that satisfy specific user
requests.  Most of these facilities are provided as a C++ library.  In
addition to this library, Loci provides a preprocessor program that
translates high level descriptions into C++ code.  The main purpose of
this preprocessor is to automate the more mundane aspects of the C++
interface and is not actually needed to develop Loci programs.  This
tutorial will mainly focus on using the loci preprocessor.  Loci
programs are provided with files using the {\tt .loci} suffix to
indicate these are files that contain Loci preprocessor directives.

% Loci intends to help a developer manage the complexity of a software
% project.  Loci appeals to logic programming, and to ideas about
% database queries, to manage complexity.  The developer provides
% descriptions of objects and algorithms.  The user of a code developed
% in Loci sets the code going by calling the program with a particular
% query.  Backward chaining, Loci creates a schedule of computations.

% Loci offers parallelism to the scientist or engineer without requiring
% her to become a programmer or a computer scientist.  Whether in
% multiple threads on a shared-memory machine, or in multiple processes
% on a distributed-memory system, Loci schedules manage parallel
% computation.

% The user of Loci writes a program with three parts:  the definitions,
% the transformations, and the goals.  The definitions are collected in
% the facts database, the transformations are collected in the rules
% database, and the query, as already mentioned, is given when the
% program is called.

% Now let us begin to learn about Loci by looking at an example.  We
% shall develop code to numerically solve a two-dimensional heat
% equation with initial and boundary conditions.
