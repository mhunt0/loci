\chapter {Introduction}
The Loci\footnote{\scriptsize The name Loci is derived from the
  framework's ability to compute the locus of application component
  specifications.} system is an application framework that seeks to
reduce the complexity of assembling large-scale finite-difference or
finite-element applications, although it could be applied to many
algorithms that are described with respect to a connectivity network or
graph.  The design of the Loci system recognizes that a significant
portion of the complexity and bugs associated with developing large
scale computational field simulations is derived from errors in
control and data movement.  In other words, a significant number of
errors in these applications are caused by incorrect looping
structures, improper calling sequences, or incorrect data transfers.
Many of these problems are subtle and result from gradual evolution of
the application over time giving rise to inconsistencies between
various application components.  The Loci framework addresses these
problems by automatically generating the control and data movement
operations of an application from component specifications while
guaranteeing a level of consistency between components.  The approach
taken is similar to Strand \cite{Foster.90} except that it includes
semantics of unstructured mesh computations in rule specifications.
%
\begin{figure}[h]
\special{psfile=loci.eps vscale=25 hscale=25 voffset=-100 hoffset=75}
\vspace{1.5in}\caption {  A Loci framework }
\end{figure}
%
This tutorial is intended as a practical introduction to functional 
programming using Loci. We provide a set of simple codes, which 
describe some salient features of Loci. Although the programs are
intentionally kept simple, by the end of the tutorial, user should
have essential information required to develop programming in Loci. 
