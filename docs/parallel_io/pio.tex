\documentclass{article}



\usepackage{listings}
\usepackage{xcolor}
\definecolor{Darkgreen}{rgb}{0,0.4,0}
\definecolor{listinggray}{gray}{0.9}
\definecolor{lbcolor}{rgb}{0.9,0.9,0.9}
\lstset{
backgroundcolor=\color{lbcolor},
    tabsize=4,    
%   rulecolor=,
    language=[GNU]C++,
        basicstyle=\scriptsize,
%        upquote=true,
        aboveskip={1.5\baselineskip},
        columns=fixed,
        showstringspaces=false,
        extendedchars=false,
        breaklines=true,
        prebreak = \raisebox{0ex}[0ex][0ex]{\ensuremath{\hookleftarrow}},
        frame=single,
        numbers=left,
        showtabs=false,
        showspaces=false,
        showstringspaces=false,
        identifierstyle=\ttfamily,
        keywordstyle=\color[rgb]{0,0,1},
        commentstyle=\color[rgb]{0.026,0.112,0.095},
        stringstyle=\color[rgb]{0.627,0.126,0.941},
        numberstyle=\color[rgb]{0.205, 0.142, 0.73},
%        \lstdefinestyle{C++}{language=C++,style=numbers}’.
}

\lstset{
    backgroundcolor=\color{lbcolor},
    tabsize=4,
  language=C++,
  captionpos=b,
  tabsize=3,
  frame=lines,
  numbers=left,
  numberstyle=\tiny,
  numbersep=5pt,
  breaklines=true,
  showstringspaces=false,
  basicstyle=\footnotesize,
%  identifierstyle=\color{magenta},
  keywordstyle=\color[rgb]{0,0,1},
  commentstyle=\color{Darkgreen},
  stringstyle=\color{red}
  }
\pagestyle{myheadings}

\begin{document}


\title{
{\bf Paralle IO in Loci}
}
\maketitle
\section{Overview}
To improve I/O performance, the Loci framework was updated to take
advantage of the parallel I/O provided by the file system.
Originally hdf5 file format was used, serial I/O was performed. i.e.,
only process 0 will perform read/write.   After update, hdf5 file
format is still used, but each process will perform read/write to a
shared file if necessary. The update is made transparant to the users
as much as possible. However, the developer/user should read this
documentation to avoid errors.

To improve I/O performance, Loci also added MPI parallel file  I/O functions. Intead of
reading/writing to a self-described, structured, and portable hdf5
file, these MPI I/O
functions read/write  from multiple processes to a commom file which
is not portable.     

   
\section{Functions that are used to create, open and close a file for
serial/parallel io:}

Loci provided a set of functions to  create, open and close a file. Use
Loci-provided API listed in this section to make sure your code can
run with both serial and parallel I/O.

\begin{lstlisting}
  hid_t hdf5CreateFile(const char *name, unsigned flags,
                      hid_t create_id,
                      hid_t access_id, MPI_Comm comm,
                      size_t file_size_estimate=0);
 
 
  hid_t hdf5CreateFile(const char *name, unsigned flags,
                      hid_t create_id,
                      hid_t access_id,
                      size_t file_size_estimate = 0) ;
 

  hid_t hdf5OpenFile(const char *name, unsigned flags,
                     hid_t access_id,
		     MPI_Comm comm) ;


  hid_t hdf5OpenFile(const char *name, unsigned flags,
                     hid_t access_id) ;

 
  herr_t hdf5CloseFile(hid_t file_id) ;
  
  herr_t hdf5CloseFile(hid_t file_id, MPI_Comm comm);

 
  hid_t writeVOGOpen(std::string filename) ;
  
  hid_t readVOGOpen(std::string filename) ;
  
  void writeVOGClose(hid_t file_id) ;
 
  hid_t createUnorderedFile(const char * filename,
                            entitySet set,
                            fact_db &facts) ;
  
  
  hid_t createUnorderedFile(const char * filename,
                            entitySet set);
  
 
  void closeUnorderedFile(hid_t file_id) ;
  

\end{lstlisting}

\section{Functions that are used to manage parallel io:}

\begin{lstlisting}
 void set_parallel_io(bool io_type);

\end{lstlisting}
 If the user need setup the parallel I/O or serail I/O for the whole application, then --pio, --nopio options in the command line will be enough.   
 For example:
 \begin{lstlisting}

 mpirun -np 40 chem --pio case1.vog

\end{lstlisting}

 This function is used when the user need to switch between parallel I/O and serial I/O.  It should be called collectively, i.e. each process should call it, and the order of calls should be the same.  And this function should be called right before open/create a file or after close a file. it should never be called between open/create a file and close a file.

 \begin{lstlisting}
 set_parallel_io(true);

 file_id = Loci::hdf5OpenFile(testfile.c_str(),
                                 H5F_ACC_RDONLY, H5P_DEFAULT);

 Loci::readMultiStore(file_id, "test_array",test_array3,dom,*Loci::exec_current_fact_db) ;

 Loci::hdf5CloseFile(file_id) ;

 set_parallel_io(false);

 \end{lstlisting}

\section{Functions that are used to perform serial/parallel I/O:}
The following functions will perform I/O in serial or in parallel, depend on the settting in Loci.
  
\begin{lstlisting}
void read_store(hid_t group_id, storeRepP qrep,
                int &offset, MPI_Comm comm);

void write_store(hid_t group_id, storeRepP qrep, 
                 entitySet dom, int offset,
                 MPI_Comm comm);

void writeContainer(hid_t file_id,std::string vname,
                   Loci::storeRepP var, fact_db &facts);

void readContainer(hid_t file_id, std::string vname,
                   Loci::storeRepP var,
                   entitySet readSet, fact_db &facts);

void writeContainer(hid_t file_id,std::string vname,
                    Loci::storeRepP var);

void readContainer(hid_t file_id, std::string vname, 
                   Loci::storeRepP var,
                   entitySet readSet);

template<class T>
void writeUnorderedVector(hid_t group_id,
                          const char *element_name,
                          std::vector<T> &v,
                          MPI_Comm prime_comm);


void writeContainerRAW(hid_t file_id, std::string vname,
                         storeRepP var, MPI_Comm comm) ;

  
void readContainerRAW(hid_t file_id, std::string vname,
                        storeRepP var, MPI_Comm comm ) ;


template<class T> void readVectorSerial(hid_t group_id,
                                        const char *element_name,
                                        std::vector<T> &v ) ;

template< class T >
inline void writeMultiStore(hid_t file_id, std::string vname,
                            multiStore<T> &var, entitySet write_set,
                            fact_db &facts);


template< class T >
inline void writeMultiStore(hid_t file_id, std::string vname,
                            const const_multiStore<T> &var,
                            entitySet write_set, fact_db &facts);

template< class T >  
inline void readMultiStore(hid_t file_id,
                           std::string vname,
                           multiStore<T> &var,
                           entitySet read_set,
                           fact_db &facts); 

\end{lstlisting}

\section{Namespace pio:}
namespace pio is used to keep functions that has separate version for serial I/O and
parallel I/O, these functions are for Loci developers, not for Loci users.



For backward campatibilty, the following functions are supposed to be in namespace pio but are left outside. Users should use functions in the previous section. 

\begin{lstlisting}
 template< class T >
  inline void writeMultiStoreP(hid_t file_id, std::string vname,
                               multiStore<T> &var, entitySet write_set,
                               fact_db &facts) ;
template< class T >
   void writeMultiStoreP(hid_t file_id, std::string vname,
                         const const_multiStore<T> &var,
                         entitySet write_set, fact_db &facts); 

template< class T >
 void readMultiStoreP(hid_t file_id,
                      std::string vname,
                      multiStore<T> &var,
                      entitySet read_set,
                      fact_db &facts); 

\end{lstlisting}


 
\section{Functions that are obselete:}
The following functions can be replaced by functions in section 2. Should not be used in the future. 

\begin{lstlisting}
hid_t hdf5PCreateFile(const char *name, unsigned flags,
                      hid_t create_id, hid_t access_id, 
                      size_t file_size_estimate,MPI_Comm comm) ;

hid_t hdf5PCreateFile(const char *name, unsigned flags,
                      hid_t create_id, hid_t access_id, 
                      size_t file_size_estimate)
  
hid_t hdf5POpenFile(const char *name, unsigned flags,
                     hid_t access_id); 

hid_t hdf5POpenFile(const char *name,
 unsigned flags, 
 hid_t access_id,
 MPI_Comm comm) ;

herr_t hdf5PCloseFile(hid_t file_id);      
 
herr_t hdf5PCloseFile(hid_t file_id, MPI_Comm comm) ;
\end{lstlisting}


\section {Raw  MPI parallel file I/O }
Loci provided the following functions to perform raw MPI parallel file
I/O.
 \begin{lstlisting}
 template< class T >
 void pmpi_writeStoreP(std::string& filename,                              
                        store<T> &var,
                        const entitySet& write_set,
                        fact_db &facts,
                        int xfer_type,
                        bool ordered) ;
 template< class T >
 void pmpi_writeStoreVecP(std::string& filename,
                         storeVec<T> &var,
                         const entitySet& write_set,
                         fact_db &facts,
                         int xfer_type,
                         bool ordered);

 template< class T >
 void pmpi_writeStoreP(std::string& filename,                              
                        const_store<T> &var,
                        const entitySet& write_set,
                        fact_db &facts,
                        int xfer_type,
                        bool ordered) ;
 template< class T >
 void pmpi_writeStoreVecP(std::string& filename,
                         const_storeVec<T> &var,
                         const entitySet& write_set,
                         fact_db &facts,
                         int xfer_type,
                         bool ordered);

 template< class T >
 void pmpi_readStoreP(std::string& filename,
                      store<T> &var,
                      entitySet read_set,
                      fact_db &facts, int xfer_type);

 template< class T >
 void pmpi_readStoreVecP(std::string& filename,
                         storeVec<T> &var,
                         entitySet read_set,
                         fact_db &facts, int xfer_type);

template< class T >
void pmpi_writeMultiStoreP(std::string filename,                              
                          multiStore<T> &var,
                          entitySet write_set,
                          fact_db &facts, int xfer_type);
template< class T >
void pmpi_writeMultiStoreP(std::string filename,
                          const const_multiStore<T> &var,
                          entitySet write_set,
                          fact_db &facts, int xfer_type);  
template< class T >
void pmpi_readMultiStoreP(std::string& filename,
                          multiStore<T> &var,
                          entitySet read_set,
                          fact_db &facts, int xfer_type);

\end{lstlisting}




 Compared with parallel hdf file I/O version,  MPI I/O might has
better performance, depending on the data size. But is has limitations: 

\begin{itemize} 
\item each variable is read/written to a separate file.
\item the file is not portable.
\item the file has a fixed structure.
\item Loci MPI I/O interface can only be used for store, storeVec, and multiStore. Each
  type has its own function interface. Not like hdf file I/O function
  interface, writeConatiner() and readContainer() can be used for
  parameters, stores, and storeVecs.    
\item Loci MPI I/O interface only has parallel I/O version. 
\end{itemize}

The structure of MPI I/O file is a file header, a domain/fileID section
followed by a data section.
 
the file header is defined as:

 \begin{lstlisting}
 typedef struct {
   int ordered;  //specify if the container is reordered in file
                 //numbering before it is written out.
                 //It has two values 1 or 0
   int vec_size; //specify the vecSize of storeVecs;
                 //For stores, vec_size = 1.  
   int dom_size; //if ordered, specify how many intergers is used
                 //to define the domain. Not entitySet.size(), 
                 //but enritySet.num_intervals()*2.  
                 //if unordered, dom_size should be 0
 } store_header;
\end{lstlisting}  

 If the container is reordered in file
numbering before being written out, then the header is followed by the whole domain
in file numbering. Otherwise, the header is followed by the unordered
fileID from process 0, process 1, ...

The last part of the file is the actual data in the container.    
\section {Examples }
\subsection {Write containers(store, storeVec, not multiStore) in hdf5 format }

\begin{lstlisting}

 //setup a container to write out 
 //the container can be store, storeVec.
 //if multiStore, use writeMultiStoreP(), not writeContainer() 
 storeVec<real_t> test_storeVec ;


 //fill data in test_storeVec here
 ...

 //set parallel io here,
 //to use serial io, set to false; 
 //if the whole program use the same io type, 
 //command line option '--pio' '--nopio' will work,
 //this line is not necessary  
 Loci::set_parallel_io(true);
 
 string testfile = "./storeVec.hdf5" ;

 //the following code is the same for store, storeVec,
 //not for multiStore

 //open a file for writeonly. 
 //used for both serial and parallel io
 //can also use   hdf5PCreateFile() 
 file_id = Loci::writeVOGOpen(testfile);

 //for both serial and parallel io, 
 //for both store and storeVec, but no multiStore
 Loci::writeContainer(file_id, "test_storeVec",test_storeVec.Rep()) ;

 //close file, for both serial and parallel io
 Loci::writeVOGClose(file_id);
 ...

\end{lstlisting}

\subsection {Read containers(store, storeVec, not multiStore) in hdf5 format }

\begin{lstlisting}

 //setup a container to read in 
 //the container can be store, storeVec.
 //if multiStore, use readMultiStoreP(), not readContainer() 
 storeVec<real_t> test_storeVec2 ;

 //must allocate in advance, entitySet dom is from
 //the context of the code
 test_storeVec2.allocate(dom);
 
     
   


 //set parallel io here,
 //to use serial io, set to false; 
 //if the whole program use the same io type, 
 //command line option '--pio' '--nopio' will work,
 //this line is not necessary  
 Loci::set_parallel_io(true);
 
 string testfile = "./storeVec.hdf5" ;

 //the following code is the same for store, storeVec,
 //not for multiStore

 //open a file for readonly. 
 //used for both serial and parallel io
 //can also use:  file_id = Loci::hdf5OpenFile(testfile.c_str(),
 //                                H5F_ACC_RDONLY, H5P_DEFAULT);  
 file_id = Loci::readVOGOpen(testfile);

 //for both serial and parallel io, 
 //for both store and storeVec, but no multiStore
 Loci::readContainer(file_id, "test_storeVec",test_storeVec2.Rep(),
 dom, *Loci::exec_current_fact_db) ;


 //close file, for both serial and parallel io
 //can also use:  Loci::writeVOGClose(file_id);
 Loci::hdf5CloseFile(file_id); 
 
 ...

\end{lstlisting}


\subsection {Write multiStore in hdf5 format }

\begin{lstlisting}

 //setup a multiStore to write out 
 multiStore<real_t> test_multiStore ;


 //fill data in test_multiStore here
 ...

 //set parallel io here,
 //to use serial io, set to false; 
 //if the whole program use the same io type, 
 //command line option '--pio' '--nopio' will work,
 //this line is not necessary  
 Loci::set_parallel_io(true);
 
 string testfile = "./multiStore.hdf5" ;

 //open a file for writeonly. 
 //used for both serial and parallel io
 //can also use   hdf5PCreateFile() 
 file_id = Loci::writeVOGOpen(testfile);

 //for both serial and parallel io, 
  Loci::writeMultiStoreP(file_id, "test_multiStore",test_multiStore,
                         dom,  *Loci::exec_current_fact_db) ;

 //close file, for both serial and parallel io
 Loci::writeVOGClose(file_id);
 ...

\end{lstlisting}

\subsection {Read  multiStore  in hdf5 format }

\begin{lstlisting}

 //setup a multiStore to read in 
 multiStore<real_t> test_multiStore2 ;


 //set parallel io here,
 //to use serial io, set to false; 
 //if the whole program use the same io type, 
 //command line option '--pio' '--nopio' will work,
 //this line is not necessary  
 Loci::set_parallel_io(true);
 
 string testfile = "./multiStore.hdf5" ;

 //open a file for readonly. 
 //used for both serial and parallel io
 //can also use:  file_id = Loci::readVOGOpen(testfile);
 file_id = Loci::hdf5OpenFile(testfile.c_str(),
                             H5F_ACC_RDONLY, H5P_DEFAULT);  

 Loci::readMultiStoreP(file_id, "test_multiStore",test_multiStore2,dom, *Loci::exec_current_fact_db) ;

 //close file, for both serial and parallel io
 //can also use:  Loci::writeVOGClose(file_id);
 Loci::hdf5CloseFile(file_id); 
 
 ...

\end{lstlisting}



\subsection {Write storeVec with raw MPI I/O}
\begin{lstlisting}
 
 //setup a container to write out 
 //different type of container has different MPI I/O function 
 storeVec<real_t> test_storeVec ;
 
 //fill data in test_storeVec here
 ...
 

 //Notice: all Loci MPI I/O functions use parallel I/O only 
 //set parallel io here,
 //if the whole program use parallel I/O type, 
 //command line option '--pio'  will work,
 //this line is not necessary 
 Loci::set_parallel_io(true); 

 string mpi_testfile = "./storeVec.dat" ;
 
 //data transfer type: can be DXFER_COLLECTIVE_IO or DXFER_INDEPENDENT_IO
 int DXFER_TYPE = DXFER_COLLECTIVE_IO;
 
 //specify if the container will be reordered 
 //according to file number before writing
 bool ordered = true; 

 //different functions for different type of containers, 
 //store: pmpi_writeStoreP(...),
 //storeVec: pmpi_writeStoreVecP(...), 
 //multiStore: pmpi_writeMultiStoreP(...) 
 //this function will open a file for writeonly, 
 //write out container, and then close the file
 Loci::pmpi_writeStoreVecP(mpi_testfile,
                           test_storeVec, dom,  
                           *Loci::exec_current_fact_db, DXFER_TYPE, ordered); 
 ...
\end{lstlisting}

\subsection {Read storeVec with raw MPI I/O}
\begin{lstlisting}
 
 //setup a container to read 
 //different type of container has different MPI I/O function 
 storeVec<real_t> test_storeVec2 ;
 
 //must allocate in advance, entitySet dom is from
 //the context of the code
 test_storeVec2.allocate(dom);

 ...
 

 //Notice: all Loci MPI I/O functions use parallel I/O only 
 //set parallel io here,
 //if the whole program use parallel I/O type, 
 //command line option '--pio'  will work,
 //this line is not necessary 
 Loci::set_parallel_io(true); 

 string mpi_testfile = "./storeVec.dat" ;
 
 //data transfer type: can be DXFER_COLLECTIVE_IO or DXFER_INDEPENDENT_IO
 int DXFER_TYPE = DXFER_COLLECTIVE_IO;
 

 //different functions for different type of containers, 
 //store: pmpi_readStoreP(...),
 //storeVec: pmpi_readStoreVecP(...), 
 //multiStore: pmpi_readMultiStoreP(...) 
 //this function will open a file for readonly, 
 //read in container, and then close the file
 Loci::pmpi_readStoreVecP(mpi_testfile, test_storeVec2, dom, *Loci::exec_current_fact_db, DXFER_TYPE); 
 
 ...
\end{lstlisting}




\end{document}
